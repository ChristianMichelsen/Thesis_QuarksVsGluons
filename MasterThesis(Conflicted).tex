\documentclass[a4paper, twoside, nobib]{tufte-book}


\hypersetup{colorlinks}% uncomment this line if you prefer colored hyperlinks (e.g., for onscreen viewing)

%%
% Book metadata
\title[A Physicist's Approach To Machine Learning -- Understanding The Basic Bricks]{A Physicist's \newline \noindent 
       Approach to \newline \noindent 
       Machine Learning \newline \noindent
       --  \newline \noindent
       Understanding  \newline \noindent
       The Basic Bricks}
\author[Christian Michelsen]{\newline \noindent 
        Christian Michelsen \newline \noindent 
        Niels Bohr Institute \newline \noindent 
        University of Copenhagen \newline }
\publisher[]{Supervisor: \newline \noindent 
           Troels Petersen \newline \noindent
           Niels Bohr Institute \newline \noindent
           University of Copenhagen}

%%
% If they're installed, use Bergamo and Chantilly from www.fontsite.com.
% They're clones of Bembo and Gill Sans, respectively.
%\IfFileExists{bergamo.sty}{\usepackage[osf]{bergamo}}{}% Bembo
%\IfFileExists{chantill.sty}{\usepackage{chantill}}{}% Gill Sans

%\usepackage{microtype}

% when using custom bibliography
\usepackage[square, numbers, nonamebreak]{natbib}
\bibliographystyle{abbrvnat}


%%
% Just some sample text
\usepackage{lipsum}

% package for quotes
\usepackage{epigraph}
\setlength\epigraphwidth{.8\textwidth}
% \setlength\epigraphrule{0pt}

%%
% For nicely typeset tabular material
\usepackage{booktabs}

% maths packages
\usepackage{amsthm,amssymb,amsmath}
\usepackage[mathscr]{eucal}

%%
% For graphics / images
\usepackage{graphicx}
\setkeys{Gin}{width=\linewidth,totalheight=\textheight,keepaspectratio}
\graphicspath{{graphics/}{{figures/}}}

% allow for dots in filename
\usepackage{grffile}


\usepackage{tikz}
\usepackage[subpreambles=true]{standalone}

% subcaption does not work with tufte
\usepackage[caption=false]{subfig}
% \usepackage[export]{adjustbox}


% The fancyvrb package lets us customize the formatting of verbatim
% environments.  We use a slightly smaller font.
\usepackage{fancyvrb}
\fvset{fontsize=\normalsize}

%%
% Prints argument within hanging parentheses (i.e., parentheses that take
% up no horizontal space).  Useful in tabular environments.
\newcommand{\hangp}[1]{\makebox[0pt][r]{(}#1\makebox[0pt][l]{)}}

%%
% Prints an asterisk that takes up no horizontal space.
% Useful in tabular environments.
\newcommand{\hangstar}{\makebox[0pt][l]{*}}

%%
% Prints a trailing space in a smart way.
\usepackage{xspace}

% add numbers to chapters, sections, subsections
\setcounter{secnumdepth}{2}
\setcounter{tocdepth}{2}

% Feyman diagrams
% \usepackage{tikz-feynman}
% \usepackage{feynman}

% lemmas and corrollaryies
% \newtheorem{theorem}{Theorem}[section]
% \newtheorem{corollary}{Corollary}[theorem]
% \newtheorem{lemma}[theorem]{Lemma}
\newtheorem{lemma}{Lemma}
\newtheorem{sublemma}{Lemma}[lemma]
\newtheorem{corollary}{Corollary}
\newtheorem{theorem}{Theorem}
\newtheorem{axiom}{Axiom}

% bold math
\usepackage{bm}

% for vector norms and abs
\newcommand{\abs}[1]{\lvert#1\rvert}
\newcommand{\norm}[1]{\lVert#1\rVert}

% allow first, second and third, 1st 2nd 3rd
\usepackage[super]{nth}

% % for using Inkscape plots
\usepackage{import}
\usepackage{xifthen}
\usepackage{pdfpages}
\usepackage{transparent}
\newcommand{\incfig}[1]{%
    \def\svgwidth{\columnwidth}
    \import{./figures/}{#1.pdf_tex}
}


%  units
\usepackage[binary-units=true]{siunitx}
\sisetup{seperr}
\DeclareSIUnit{\kroner}{kr.}
\DeclareSIUnit{\kr}{kr.}
\DeclareSIUnit{\Mkr}{M.kr.}
\DeclareSIUnit{\year}{year}
\DeclareSIUnit{\years}{years}
\DeclareSIUnit{\yr}{yr}

%  alignment in matrix matrices
\usepackage{mathtools}

% custom subsubsection:
\titleclass{\subsubsection}{straight}
\titleformat{\subsubsection}%
  [hang]% shape
  {\normalfont\itshape}% format applied to label+text
  {\thesubsubsection}% label
  {1em}% horizontal separation between label and title body
  {}% before the title body
  []% after the title body


% custom centering of figure 
\makeatletter
\newcommand*{\centerfloat}{%
  \parindent \z@
  \leftskip \z@ \@plus 1fil \@minus \textwidth
  \rightskip\leftskip
  \parfillskip \z@skip}
\makeatother





% \titleformat{\chapter}%
%   [display]% shape
%   {\relax\ifthenelse{\NOT\boolean{@tufte@symmetric}}{\begin{fullwidth}}{}}% format applied to label+text
%   {\itshape\huge\thechapter}% label
%   {0pt}% horizontal separation between label and title body
%   {\huge\rmfamily\itshape}% before the title body
%   [\ifthenelse{\NOT\boolean{@tufte@symmetric}}{\end{fullwidth}}{}]% after the title body

\usepackage[dvipsnames]{xcolor}
\definecolor{red} {RGB}{227, 26, 28}
\definecolor{blue}{RGB}{31, 120, 180}
\definecolor{green}{RGB}{51, 160, 44}
\definecolor{orange}{RGB}{255, 127, 0}
\definecolor{purple}{RGB}{106, 61, 154}

\definecolor{light-blue}{RGB}{166, 206, 227}
\definecolor{light-green}{RGB}{178, 223, 138}
\definecolor{light-purple}{RGB}{202, 178, 214}

\definecolor{viridis-light}{RGB}{252, 226, 63}
\definecolor{viridis-dark}{RGB}{59, 18, 72}





\definecolor{light-gray}{gray}{0.95}


% make a code environment
\newcommand{\code}[1]{\colorbox{light-gray}{\texttt{\detokenize{#1}}}}

% make quote signs
\newcommand{\q}[1]{``#1''}

% make Figure quote
\newcommand{\figref}[1]{Figure ~\ref{#1}}

% avoid problems with autocite not properly defined
\newcommand{\autocite}[1]{\citep{#1}}

% bold vector
\renewcommand{\vec}[1]{\mathbf{#1}}

\DeclareMathOperator*{\argmin}{arg\,min} % Jan Hlavacek



% to have references in normal text and not sidenotes, see:
% https://tex.stackexchange.com/questions/247636/how-to-remove-sidenote-citations-in-tufte-latex



% \usepackage{titlesec, blindtext, color}
% \definecolor{gray75}{gray}{0.75}
% \definecolor{gray50}{gray}{0.50}
% \newcommand{\hsp}{\hspace{20pt}}
% \titleformat{\chapter}
%   [hang]
%   {\Huge\rmfamily\itshape}
%   {\thechapter\hsp\textcolor{gray50}{|}\hsp}
%   {0pt}
%   {\huge\bfseries}

% chapter format
\titleformat{\chapter}%
  {\huge\rmfamily\itshape}% format applied to label+text
  {\itshape\Huge\thechapter\hspace{1pt}.\hspace{4pt}} % label
  {0pt}% horizontal separation between label and title body
  {}% before the title body
  []% after the title body


%%
% Some shortcuts for Tufte's book titles.  The lowercase commands will
% produce the initials of the book title in italics.  The all-caps commands
% will print out the full title of the book in italics.
\newcommand{\vdqi}{\textit{VDQI}\xspace}
\newcommand{\ei}{\textit{EI}\xspace}
\newcommand{\ve}{\textit{VE}\xspace}
\newcommand{\be}{\textit{BE}\xspace}
\newcommand{\VDQI}{\textit{The Visual Display of Quantitative Information}\xspace}
\newcommand{\EI}{\textit{Envisioning Information}\xspace}
\newcommand{\VE}{\textit{Visual Explanations}\xspace}
\newcommand{\BE}{\textit{Beautiful Evidence}\xspace}

\newcommand{\TL}{Tufte-\LaTeX\xspace}

% Prints the month name (e.g., January) and the year (e.g., 2008)
\newcommand{\monthyear}{%
  \ifcase\month\or January\or February\or March\or April\or May\or June\or
  July\or August\or September\or October\or November\or
  December\fi\space\number\year
}


% % Prints an epigraph and speaker in sans serif, all-caps type.
% \newcommand{\openepigraph}[2]{%
%   %\sffamily\fontsize{14}{16}\selectfont
%   \begin{fullwidth}
%   \sffamily\large
%   \begin{doublespace}
%   \noindent\allcaps{#1}\\% epigraph
%   \noindent\allcaps{#2}% author
%   \end{doublespace}
%   \end{fullwidth}
% }

% Inserts a blank page
\newcommand{\blankpage}{\newpage\hbox{}\thispagestyle{empty}\newpage}

\usepackage{units}

% Typesets the font size, leading, and measure in the form of 10/12x26 pc.
\newcommand{\measure}[3]{#1/#2$\times$\unit[#3]{pc}}

% Macros for typesetting the documentation
\newcommand{\hlred}[1]{\textcolor{Maroon}{#1}}% prints in red
\newcommand{\hangleft}[1]{\makebox[0pt][r]{#1}}
\newcommand{\hairsp}{\hspace{1pt}}% hair space
\newcommand{\hquad}{\hskip0.5em\relax}% half quad space
\newcommand{\TODO}{\textcolor{red}{\bf TODO!}\xspace}
\newcommand{\ie}{\textit{i.\hairsp{}e.}\xspace}
\newcommand{\eg}{\textit{e.\hairsp{}g.}\xspace}
\newcommand{\na}{\quad--}% used in tables for N/A cells
\providecommand{\XeLaTeX}{X\lower.5ex\hbox{\kern-0.15em\reflectbox{E}}\kern-0.1em\LaTeX}
\newcommand{\tXeLaTeX}{\XeLaTeX\index{XeLaTeX@\protect\XeLaTeX}}
% \index{\texttt{\textbackslash xyz}@\hangleft{\texttt{\textbackslash}}\texttt{xyz}}
\newcommand{\tuftebs}{\symbol{'134}}% a backslash in tt type in OT1/T1
\newcommand{\doccmdnoindex}[2][]{\texttt{\tuftebs#2}}% command name -- adds backslash automatically (and doesn't add cmd to the index)
\newcommand{\doccmddef}[2][]{%
  \hlred{\texttt{\tuftebs#2}}\label{cmd:#2}%
  \ifthenelse{\isempty{#1}}%
    {% add the command to the index
      \index{#2 command@\protect\hangleft{\texttt{\tuftebs}}\texttt{#2}}% command name
    }%
    {% add the command and package to the index
      \index{#2 command@\protect\hangleft{\texttt{\tuftebs}}\texttt{#2} (\texttt{#1} package)}% command name
      \index{#1 package@\texttt{#1} package}\index{packages!#1@\texttt{#1}}% package name
    }%
}% command name -- adds backslash automatically
\newcommand{\doccmd}[2][]{%
  \texttt{\tuftebs#2}%
  \ifthenelse{\isempty{#1}}%
    {% add the command to the index
      \index{#2 command@\protect\hangleft{\texttt{\tuftebs}}\texttt{#2}}% command name
    }%
    {% add the command and package to the index
      \index{#2 command@\protect\hangleft{\texttt{\tuftebs}}\texttt{#2} (\texttt{#1} package)}% command name
      \index{#1 package@\texttt{#1} package}\index{packages!#1@\texttt{#1}}% package name
    }%
}% command name -- adds backslash automatically
\newcommand{\docopt}[1]{\ensuremath{\langle}\textrm{\textit{#1}}\ensuremath{\rangle}}% optional command argument
\newcommand{\docarg}[1]{\textrm{\textit{#1}}}% (required) command argument
\newenvironment{docspec}{\begin{quotation}\ttfamily\parskip0pt\parindent0pt\ignorespaces}{\end{quotation}}% command specification environment
\newcommand{\docenv}[1]{\texttt{#1}\index{#1 environment@\texttt{#1} environment}\index{environments!#1@\texttt{#1}}}% environment name
\newcommand{\docenvdef}[1]{\hlred{\texttt{#1}}\label{env:#1}\index{#1 environment@\texttt{#1} environment}\index{environments!#1@\texttt{#1}}}% environment name
\newcommand{\docpkg}[1]{\texttt{#1}\index{#1 package@\texttt{#1} package}\index{packages!#1@\texttt{#1}}}% package name
\newcommand{\doccls}[1]{\texttt{#1}}% document class name
\newcommand{\docclsopt}[1]{\texttt{#1}\index{#1 class option@\texttt{#1} class option}\index{class options!#1@\texttt{#1}}}% document class option name
\newcommand{\docclsoptdef}[1]{\hlred{\texttt{#1}}\label{clsopt:#1}\index{#1 class option@\texttt{#1} class option}\index{class options!#1@\texttt{#1}}}% document class option name defined
\newcommand{\docmsg}[2]{\bigskip\begin{fullwidth}\noindent\ttfamily#1\end{fullwidth}\medskip\par\noindent#2}
\newcommand{\docfilehook}[2]{\texttt{#1}\index{file hooks!#2}\index{#1@\texttt{#1}}}
\newcommand{\doccounter}[1]{\texttt{#1}\index{#1 counter@\texttt{#1} counter}}

% Generates the index
\usepackage{makeidx}
\makeindex


\pagenumbering{roman}

\begin{document}

% % gives the width of the current document in pts
% \showthe\textwidth
% \showthe\columnwidth

% % Front matter
% \frontmatter

% % r.1 blank page
% \blankpage

% % v.2 epigraphs
% \newpage\thispagestyle{empty}
% \openepigraph{%
% The public is more familiar with bad design than good design.
% It is, in effect, conditioned to prefer bad design, 
% because that is what it lives with. 
% The new becomes threatening, the old reassuring.
% }{Paul Rand%, {\itshape Design, Form, and Chaos}
% }
% \vfill
% \openepigraph{%
% A designer knows that he has achieved perfection 
% not when there is nothing left to add, 
% but when there is nothing left to take away.
% }{Antoine de Saint-Exup\'{e}ry}
% \vfill
% \openepigraph{%
% \ldots the designer of a new system must not only be the implementor and the first 
% large-scale user; the designer should also write the first user manual\ldots 
% If I had not participated fully in all these activities, 
% literally hundreds of improvements would never have been made, 
% because I would never have thought of them or perceived 
% why they were important.
% }{Donald E. Knuth}


% r.3 full title page
\maketitle


% v.4 copyright page
\newpage
\begin{fullwidth}
~\vfill
\thispagestyle{empty}
\setlength{\parindent}{0pt}
\setlength{\parskip}{\baselineskip}
Copyright \copyright\ \the\year\ \newline Christian Michelsen

% \par\smallcaps{Published by \thanklesspublisher}
\par\smallcaps{https://github.com/ChristianMichelsen}

\par Licensed under the Apache License, Version 2.0 (the ``License''); you may not
use this file except in compliance with the License. You may obtain a copy
of the License at \url{http://www.apache.org/licenses/LICENSE-2.0}. Unless
required by applicable law or agreed to in writing, software distributed
under the License is distributed on an \smallcaps{``AS IS'' BASIS, WITHOUT
WARRANTIES OR CONDITIONS OF ANY KIND}, either express or implied. See the
License for the specific language governing permissions and limitations
under the License.\index{license}

\par\textit{First printing, \monthyear}
\end{fullwidth}

% r.5 contents
\tableofcontents

\listoffigures

\listoftables

% % r.7 dedication
% \cleardoublepage
% ~\vfill
% \begin{doublespace}
% \noindent\fontsize{18}{22}\selectfont\itshape
% \nohyphenation
% Dedicated to those who appreciate \LaTeX{} 
% and the work of \mbox{Edward R.~Tufte} 
% and \mbox{Donald E.~Knuth}.
% \end{doublespace}
% \vfill
% \vfill


% r.9 introduction
\cleardoublepage
\chapter{Abstract}
\pagenumbering{arabic}

This sample book discusses the design of Edward Tufte's books
%  \citep{Tufte2001,Tufte1990,Tufte1997,Tufte2006} 
and the use of the \doccls{tufte-book} and \doccls{tufte-handout} document classes.


%%
% Start the main matter (normal chapters)
\mainmatter

\chapter{Introduction}
\label{ch:introduction}
\epigraph{\textit{``Begin at the beginning,'' the King said, gravely, ``and go on till you
come to an end; then stop.''}}{--- Lewis Carroll, \textit{Alice in Wonderland}}


\newthought{Not only} \marginnote{The background for this masters's thesis, is that it is part of a so-called 4+4 Ph.D. project (also known as an integrated Ph.D.). The Ph.D. dissertation is about the use of machine learning and deep learning in the field of ancient genomics. Here ancient DNA is sampled and analysed with the hope of finding patterns, structure, in the genome which were previously unknown. The overall goal is two-fold. On the big scale it is the better understand human history in the broadest sense of the word history. Where did we come from, where did we go. On a much smaller scale, the goal is to understand local history and migration patterns; how did we end up where we did. It is with this background that this project should be seen: as an introduction to the general use of applied machine learning.}
is the title of this project fairly broad, so are the subjects covered in this thesis. The overall goal of this project is to apply machine learning to different datasets and see how well these comparatively new tools might improve classical statistical methods. The project have dealt with two (seemingly) very different datasets: Danish housing prices and Quark-Gluon discrimination in particle physics, and the aim of this section is to provide an initial overview of the scope and relationship of the two sub-projects; two sub-projects which are covered in each part of this book. 



The first part of the thesis deals with the problem of estimating housing prices as precisely and accurately as possible. This was the sub-project that was worked on in the beginning of the overall project and worked as an initial introduction to the application of machine learning to real-life datasets. The housing prices dataset thus became the playground in which the subtleties of these new modern tools were examined, where the difference between real life datasets with all its quirks, outliers and bad formatting, and curated toy datasets that works out of the box (such as the famous Iris dataset \citep{andersonSpeciesProblemIris1936a,fisherUseMultipleMeasurements1936}) were experienced first hand. Since the project started the dataset changed due to a new collaboration with the Danish housing agency \href{www.boligsiden.dk}{Boligsiden} where the agreement was, stated shortly, that we would get their data and they would get our results. Boligsiden is a natural collaborator since they are the biggest on the market\sidenote{Due to being owned by the \q{Dansk Ejendomsmæglerforening}, The Danish Association of Chartered Estate Agents.} and have been very helpful the continuos process of providing data but it also should also be noted that they have had no say on the results presented in this thesis. 
During this initial stage, the author sparred with Simon Gudiksen\sidenote{Who afterwards went on to get a job at Boligsiden.} who also worked on the same dataset, however, both projects were done independently. Where Gudiksen focussed on the prediction of the time evolution of the housing prices using Recurrent Neural Networks (RNN), my work was mostly on the different levels and methods of hyperparameter optimization with some smaller detours into Natural Language Processing (NLP) as an example. \newline


The second part, the Quark-Gluon discrimination in particle physics, was the main part of the project. Not only was most of the time focussed on this sub-project, it was also the work that generated the highest academic output; an article based on this is in the making. This part dealt with data from the Large Electron Positron collider (LEP) which was an underground particle accelerator at CERN built in \num{1989} and was discontinued in \num{2000}, where the first phase (LEP1), from \num{1989}-\num{1995}, is the sole source of data. As the name suggests it collided electrons and positrons together in what is still the largest electron-positron accelerator ever built \citep{LargeElectronPositronCollider}. During LEP1 it was primarily the decay channels of the $Z$-boson that were probed where especially the $Z\rightarrow q\bar{q}g$ and $Z \rightarrow q\bar{q}gg$ were examined in this thesis. It is especially the distributions of these gluon jets and the difference between Data\sidenote{Where \q{Data} with capital D refers to the actual, measured data and \q{data} refers to any arbitrary selection of data.}   and MC that are of interests to the theoreticians that develop the MC-models. At first an improved $b$-tagging algorithm was developed based on only the vertex variables since the primary variables of interest to the theoreticians are the shape variables. Here methods and code developed in the hyperparameter optimization process from the housing prices part were used. After the improvement in the $b$-tagging model, an event-based $g$-tag model -- in comparison to the jet-based $b$-tagging model -- was implemented which (hopefully) allows one to extract useful events of interest. Having found these useful events, one can start looking at how the distributions in the relevant variables differ between Data and MC. Finally XXX \TODO.

The thesis is structured such that \autoref{ch:ML_theory} introduces the needed theoretical Machine Learning (ML) background needed for understanding the methods used throughout the thesis, \autoref{ch:housing_price_analysis} describes the housing prices part of the project as mentioned above, \autoref{ch:particle_physcis_LEP} introduces the basic physics in the standard model and the Lund string model which is used throughout the rest of the theses, \autoref{ch:quark_gluon_analysis} explains analysis of the main project in this thesis, i.e. the quark gluon analysis, and finally the two chapters \autoref{ch:discussion_outlook} and \autoref{ch:conclusion} discusses the overall work in this thesis and concludes on it. 

The work presented in this thesis is split up into two parts as presented above, however, it should be noted that during the analysis part of the project they were treated not as two different projects but rather as two different instances of same underlying problem: teaching computers how to find patterns in high-dimensional data automatically and should thus not be seen as two independent projects. This also highlight another key aspect of this project, that the author does not have any background in particle physics other than rudimentary knowledge stemming from an undergraduate education in general physics. 

All of the work presented here is performed by the author unless otherwise noted.

\chapter{Machine Learning Theory}
\label{ch:ML_theory}

\epigraph{\textit{``People worry that computers will get too smart and take over the world, but the real problem is that they're too stupid and they've already taken over the world.''}}{--- Pedro Domingos}

\newthought{Machine Learning} is the method of teaching computers how to automatically find patterns in (often high-dimensional) data. According to some sceptics machine learning (ML) is just glorified statistics, however, by the same logic physics is just glorified mathematics. In contrary, machine learning is a set of subject located somewhere along the hypothetical line from simple, classical statistics to futuristic artificial intelligence. It includes methods ranging from the well-known statistical methods such as linear regression to the modern, advanced zoo of different neural networks \citep{veenNeuralNetworkZoo2016} which has seen a plethora of use cases in recent years. 

\section{Statistical Learning Theory}
This chapter deals with the theory of ML which Statistical Learning Theory is a subcategory of. Many books are written on the subject where this thesis especially follows the overall notation used in the very accessible introduction in the book Learning From Data \citep{abu-mostafaLearningData2012} and the graduate course Advanced Topics in Machine Learning \citep{AdvancedTopicsMachine} at the computer science institute\sidenote{Datalogisk Institut Københavns Universitet, DIKU.} at the faculty of Science, University of Copenhagen. Statistical learning theory is the analysis of how to not only find the function, or \emph{hypothesis}, that matches the data best, but also bounding the difference in performance between this hypothesis and the hidden, underlying data generation distribution often only known by Nature. 

Overall there are two subfields within machine learning: supervised and unsupervised\sidenote{Also known as \q{self-supervised} or \q{predictive} learning.}. The difference depends on whether or not the data that is trained on is labelled or not. Classic linear regression is an example of the former and linear dimensionality reduction using PCA of the latter. Since unsupervised learning techniques are only used sparsely throughout this project, the main focus will be on supervised learning. 

\section{Supervised Learning}
\label{sec:ml:supervised_learning}
In supervised learning we are given a set of $N$ different samples of which we for each one knows $M$ different variables written as the column-vector $\vec{x}_i = [x_1, x_2, \dots, x_M] \in \mathcal{X}$ for the $i$th observation and $\mathcal{X}$ denotes the sample space. All of these samples as a whole is written as the matrix $\vec{X}$ with the individual observations $\vec{x}_i$ transposed and stacked on top of each other $\vec{X} = [\vec{x}_0^T, \vec{x}_1^T, \dots, \vec{x}_N^T]$ such that row $i$ in $\vec{X}$ corresponds to observation $i$. In the case of supervised learning we also have the label $y$ of each sample $y \in \mathcal{Y}$ where $\mathcal{Y}$ denotes the sample or label space. 
In the case where $y$ is a real number $\mathcal{Y} = \mathbb{R}$ the problem is said to a \emph{regression} problem, e.g. predicting the price of a house. On the contrary, if $\mathcal{Y}$ is binary such that $\mathcal{Y} = \{0, 1\}$ then the problem is said to be a (binary) \emph{classification} problem\sidenote{If $\mathcal{Y}\in \mathbb{Z}$ then it would be a multiclass classification problem.} e.g. predicting whether or not a particle is a quark or not. 

\begin{marginfigure}
  \includegraphics[width=0.99\textwidth, trim=5 5 10 5, clip]{figures/learning_problem/learning_problem.pdf}
  \caption[Overview of the learning problem.]
    {Schematic overview of the learning problem and how to find the optimal hypothesis $h^*$ to approximate $f$ given the training data $\mathcal{D}_\mathrm{train}$.
    }
  \label{fig:ml:learning_problem}
\end{marginfigure}

Without any loss of loss of generality let the focus for now be on classification. The goal is to find the underlying \q{true} function $f: \mathcal{X} \mapsto \mathcal{Y}$ that gives the correct label $y\in\mathcal{Y}$ for each observation $\vec{x}\in\mathcal{X}$. This function, however, is unknown and cannot perfectly be found. Although it is impossible to find $f$ it is possible to learn an approximation of it $h$ based on some training observations $\mathcal{D}_\mathrm{train} = \left \{(\vec{x}_1, y_1), \dots, \vec{x}_N, y_N) \right\}$. The optimal hypothesis, $h^*$, is chosen among a set of candidate hypotheses $\mathcal{H} = \left\{h_1, h_2, \ldots, h_M  \right\}$, and hopefully $h^*$ will be a good approximation of $f$, $h^* \approx f$. A schematic overview of this process can be seen in Figure~\ref{fig:ml:learning_problem}. 

How can one make sure that $h^*$ really is a good approximation of $f$? That is where Statistical learning theory comes into play. From a statistical standpoint we are interested in modelling the unknown joint probability $P(\vec{x}, y)$ over $\mathcal{X}$ and $\mathcal{Y}$. We assume that $\mathcal{D}_\mathrm{train}$ is independent and identically distributed (\emph{iid})\sidenote{This is one of the two key assumptions of statistical learning theory, the other being that future events are coming from the same distrubution as the one that generated the past events. These assumptions are sometimes called the PAC assumtions where PAC is an abbreviation for Probably, Approximatelly Correct.} from $P(\vec{x}, y)$ and thus want to find the hypothesis whose predictions $h(\vec{x})=\hat{y}$ matches the conditional probability distribution $P(y|\vec{x})$ as best as possible. 

To quantify the statement \q{as best as possible} in the previous paragraph we define the loss function $\ell$ which measures the loss for predicting $\hat{y}$ instead of $y$: $\ell(\hat{y}, y) = \ell(h(\vec{x}), y) \in \mathbb{R}^+$. Given $\ell$ we now introduce the method of (empirical) risk minimization \citep{vapnikPrinciplesRiskMinimization1991} and the expected loss\sidenote{Also called expected error or the out-of-sample error.}:
\begin{equation} 
  \label{eq:L}
  L(h) = \mathbb{E} \left[\ell(h(\vec{x}), y) \right] = \int \ell(h(\vec{x}), y)  \, dP(\vec{x}, y).
  \end{equation}
The optimal hypotheses $h^*$ is the hypothesis which minimizes the expected loss $L(h)$. However, the joint probability distribution $P(\vec{x}, y)$ is unknown and we are thus left with the empirical loss\sidenote{Also called empirical error.} of $h$ on $\mathcal{D}$:
\begin{equation}
  \label{eq:L_hat}
  \hat{L}(h, S) = \frac{1}{N} \sum_{i=1}^{N} \ell(h(\vec{x}_i), y_i), % \equiv \mathcal{L} 
\end{equation}
which is an approximation of $L(h)$ based on the training data available. 
% In the remainder of the text $\hat{L}(h, S)$ and $\mathcal{L}$ will be used interchangeably. 
Now the optimal hypothesis $h^*$ can defined:
\begin{equation}
  h^* = \argmin_{h\in\mathcal{H}} \hat{L}(h, S).
\end{equation}

\section{Generalization Bound}
\label{sec:generalization_bound}
In \autoref{sec:ml:supervised_learning} the method of selecting the optimal hypothesis $h^*$ out of the total set of candidate hypothesis $\mathcal{H}$ was sketched. However, there is still no guarantee that $h^*$ will work well, that is to say that the \emph{generalization error} $G(h)$ might be big:
\begin{equation}
  G(h) = \hat{L}(h, S) - L(h).
\end{equation}
The generalization error is thus the difference between the expected error $L(h)$ and the empirical error $\hat{L}(h, S)$. It describes the loss in performance of our chosen model compared to the optimal, yet hidden, model. Since $P(\vec{x}, y)$ is unknown, $G(h)$ cannot be computed, however, it is possible to bound this error using statistical learning theory. To do so, the union bound and Hoeffding's (one-sided) inequalities are introduced. 
\begin{lemma}[The Union Bound]
  For any finite or countably infinite sequence of events $E_1, E_2, \dots$ (not necessarily independent): 
  \begin{equation}
    \mathbb{P} \left\{\bigcup_{i \leq 1} E_i \right\} \leq \sum_{i \leq 1} \mathbb{P} \left\{E_i \right \}. 
  \end{equation}
\end{lemma}
The union bound, in simple terms, states that the probability of any one of $n$ events happening is less than or equal to the sum of the individual probabilities of the events happening. As an example, let $E_1=\{2, 4, 6\}$ be the event that a die rolls an even number and $E_2=\{4, 5, 6\}$ be the event that a die rolls a number larger than or equal to $4$. Then $\mathbb{P} \left\{E_1 \cup E_2 \right\} = \mathbb{P} \left\{ 2, 4, 5, 6 \right\} \leq \mathbb{P} \left\{E_1 \right \} + \mathbb{P} \left\{E_2 \right \} $. 
\begin{lemma}[The one-sided Hoeffding's inequalities]
  Let $Z_1, \dots, Z_n$ be independent random variables each belonging to the $[0, 1]$ interval such that $\mathbb{P}\left\{Z_i \in [0, 1] \right\} = 1$ and $\mathbb{E}[Z_i] = \mu$ for all i, then for every $\epsilon > 0$:
  \begin{equation}
    \mathbb{P} \left\{  \frac{1}{N}\sum_{i=1}^N Z_i - \mu \geq \epsilon \right\} \leq e^{-2n\epsilon^2} 
    \label{eq:hoeffding_onesided_a}
  \end{equation}
  and
  \begin{equation}
    \mathbb{P} \left\{ \mu - \frac{1}{N}\sum_{i=1}^N Z_i  \geq \epsilon \right\} \leq e^{-2n\epsilon^2}.
    \label{eq:hoeffding_onesided_b}
  \end{equation}
\end{lemma}
When using the union bound on equation \eqref{eq:hoeffding_onesided_a} and equation \eqref{eq:hoeffding_onesided_b} we arrive at Hoeffding's (two) sided inequality:
\begin{lemma}[The two-sided Hoeffding's inequality]
  \label{lemma:hoeffding}
  Let $Z_1, \dots, Z_n$ be independent random variables each belonging to the $[0, 1]$ interval such that $\mathbb{P}\left\{Z_i \in [0, 1] \right\} = 1$ and $\mathbb{E}[Z_i] = \mu$ for all i, then for every $\epsilon > 0$:
  \begin{equation}
    \mathbb{P} \left\{ \left| \frac{1}{N}\sum_{i=1}^n Z_i - \mu \right| \geq \epsilon \right\} \equiv \mathbb{P} \left\{ \left| \hat{\mu} - \mu \right| \geq \epsilon \right\} \leq 2 e^{-2N\epsilon^2},
    \label{eq:hoeffding_inequality}
  \end{equation}
  where we have defined the (empirical) average of $Z$ to be $\hat{\mu}$: $\hat{\mu}=\frac{1}{n}\sum_{i=1}^N Z_i $
\end{lemma}
Assuming that the loss $\ell(\hat{y}, y)$ is bounded in the $[0, 1]$ interval\sidenote{Which it is for classification, however, it can be extended in a similar fasion for regression.}, $\ell(\hat{y}, y) \in [0, 1]$ for all $\hat{y}, y$, we can bound the generalization error $G(h)$ by letting $Z_i = \ell(\hat{y}_i, y_i) = \ell(h(\vec{x}_i, y_i))$ be the loss of $h$ in sample $(\vec{x}_i, y_i)$. By comparing Lemma \ref{lemma:hoeffding} and equation \eqref{eq:L_hat} we see that $\hat{\mu} = \hat{L}(h, S)$, and similar for equation \eqref{eq:L}: $\mu = L(h)$. We then see that the generalization error is bounded:
\begin{equation}
  \label{eq:hoeffding_inequality_generalization_error}
  \mathbb{P} \left\{ \left| G(h) \right| \geq \epsilon \right\} = \mathbb{P} \left\{ \left| \hat{L}(h, S) - L(h) \right| \geq \epsilon \right\} \leq 2 e^{-2N\epsilon^2}.
\end{equation}
This equation provides a bound on the difference between the empirical loss and the expected loss. Say the probability of the generalization error being larger than $\epsilon = 0.1$ is needed for $N=100$ samples, we find that the this probability is $P=27\%$. The generalization bound  can be rewritten in terms of $\delta$:
\begin{equation}
  \delta = 2 e^{-2N\epsilon^2} \in (0, 1) \quad \Rightarrow \quad \epsilon = \sqrt{\frac{\ln \frac{2}{\delta}}{2N}} \quad \Rightarrow
\end{equation}
\begin{theorem}[Hoeffding's inequality for a single hypothesis]
  \label{theorem:hoeffding_single}
  Assume that $\ell$ if bounded in the $[0, 1]$ interval, then for a single hypothesis $h$ and any $\delta\in(0,1)$ we have:  
  \begin{equation}
    \label{eq:hoeffding_inequality_generalization_error_delta}
    \mathbb{P} \left\{ \left| \hat{L}(h, S) - L(h) \right| \geq \sqrt{\frac{\ln \frac{2}{\delta}}{2N}}  \right\} \leq \delta.
  \end{equation}
\end{theorem}
Equation \eqref{eq:hoeffding_inequality_generalization_error_delta} can be read as the probability of the generalization error being larger than $\sqrt{\frac{\ln \frac{2}{\delta}}{2N}}$ is $\delta$ or similarly that with probability greater than $1-\delta$:
\begin{equation}
  \label{eq:hoeffding_inequality_single_PAC}
  \left| \hat{L}(h, S) - L(h) \right| \leq \sqrt{\frac{\ln \frac{2}{\delta}}{2N}}.
\end{equation}
This is a powerful result relating the performance for a (fixed) hypothesis $h$ with the number of samples, $N$. We see that a higher $N$ yields a tighter bound on the generalization error, however, inversely\sidenote{Not to be understood as $1/x$ in this context.} related to on the certainty\sidenote{Technically the certainty of the model is $1-\delta$.} $\delta$ of this bound. 

There is a big assumption of this derivation although: that the hypothesis $h$ cannot depend on the sample $S$ and thus has to be chosen before seeing the data. We say that $h$ has to be \emph{fixed}. Of course the term machine learning indicates that some kind of learning is taking place: exactly as seen previously where we wanted to find the optimal hypotheses $h^*$ out of all the possible ones $\mathcal{H}$. For now assume that $\mathcal{H}$ is finite and consists of $M$ hypotheses: $|\mathcal{H}| = M$. We thus have $[h_1, h_2, \dots, h_M]$ hypotheses which we test simultaneously and where Hoeffding's inequality is true for each of them leading to the following theorem:
\begin{theorem}[Hoeffding's inequality for a finite set of hypotheses candidates]
  \label{theorem:hoeffding_finite}
  Assume that $\ell$ is bounded in the $[0, 1]$ interval and that $|\mathcal{H}| = M$. Then for any $\delta\in(0,1)$ we have:  
  \begin{equation}
    \label{eq:hoeffding_inequality_theorem_multiple}
    \mathbb{P} \left\{ \exists h \in \mathcal{H}: \left| \hat{L}(h, S) - L(h) \right| \geq \sqrt{\frac{\ln \frac{2M}{\delta}}{2N}}  \right\} \leq \delta.
  \end{equation}
\end{theorem}

\begin{proof}
  The proof begins by denoting $H_i$ as the event where: 
  \begin{equation*}
    \left| \hat{L}(h_i, S) - L(h_i) \right| \geq \sqrt{\frac{\ln \frac{2}{\delta'}}{2N}},
  \end{equation*}
   and then taking the union bound (the first inequality) followed by applying Hoeffding's inequality to each part in the sum (the second inequality): 
\begin{equation*}
  % \label{eq:hoeffding_inequality_generalization_error_delta}
  \begin{split}
    \mathbb{P} \Biggl\{ \exists h \in \mathcal{H}: \left| \hat{L}(h, S) - L(h) \right| &\geq \sqrt{\frac{\ln \frac{2}{\delta'}}{2N}}  \Biggr\} \\
    &= \mathbb{P} \left\{ \bigcup_{h \in \mathcal{H}} \left| \hat{L}(h, S) - L(h) \right| \geq \sqrt{\frac{\ln \frac{2}{\delta'}}{2N}}  \right\}  \\
    &\leq \sum_{h \in \mathcal{H}} \mathbb{P} \left\{\left| \hat{L}(h, S) - L(h) \right| \geq \sqrt{\frac{\ln \frac{2}{\delta'}}{2N}} \right\}  \\
    &\leq \sum_{h \in \mathcal{H}} \delta' \\
    &= M \delta'.
  \end{split}
\end{equation*}
By making the substitution $\delta = M \delta'$ we arrive at equation \eqref{eq:hoeffding_inequality_theorem_multiple}. 
\end{proof}

As we did in equation \eqref{eq:hoeffding_inequality_single_PAC}, equation \eqref{eq:hoeffding_inequality_theorem_multiple} can also be read as with probability greater than $1-\delta$ then for all $h\in\mathcal{H}$:
\begin{equation}
  \label{eq:hoeffding_inequality_multi_PAC}
  \left| \hat{L}(h, S) - L(h) \right| \leq \sqrt{\frac{\ln \frac{2M}{\delta}}{2N}}.
\end{equation}
This bound is looser than the one for only a single hypothesis by a factor $\ln M$, however, this holds for the optimal hypothesis $h^*$. 

\subsection{Generalization Bound for infinite hypotheses}
\label{subsec:generalization_bound_infinite}
Section \ref{sec:generalization_bound} dealt with the case of a single hypotheses $h$ and a finite set of candidate hypotheses $h\in\mathcal{H}, |\mathcal{H}| = M$. When $M$ goes towards $\infty$ the generalization bound goes to $\infty$ and the bound becomes useless. However, even simple models such as a linear classifier\sidenote{Also known as the perceptron. Often includes a constant offset $b$ as well which is omitted for brevity.} that predicts $\hat{y}=1$ when the dot product $\vec{w}^T\vec{x}$ is positive and $\hat{y}=0$ when it is negative has $|\mathcal{H}| = \infty$. Since there an infinite number of hypotheses $h(\vec{w})$, assuming we allow $\vec{w}$ to take any real values as is almost always the case, $\mathcal{H}$ is infinite. 

To solve this obvious problem with the Hoeffding inequality, we introduce\sidenote{For proof, see: \citet{abu-mostafaLearningData2012}} the Vapnik-Chervonenkis (VC) generalization bound. The VC-bound is based on the so-called VC-dimension of the hypothesis space $\mathcal{H}$: $d_\mathrm{VC}(\mathcal{H}) = d_\mathrm{VC}$. The VC-dimension is a measure of the complexity of the hypothesis space, the degrees of freedom of the model so to say. For example the VC-dimension of the $d$-dimensional linear classifier defined above\sidenote{In the general case when the offset $b$ is included: $d_\mathrm{VC}=d+1$.} is $d_\mathrm{VC}=d$ for $\{\vec{x}, \vec{w}\} \in \mathbb{R}^d$. 
\begin{theorem}[VC Generalization Bound]
  \label{theorem:VC_generalization_bound}
  Let $\mathcal{H}$ be a hypotheses class with VC-dimension: $d_\mathrm{VC}(\mathcal{H}) = d_\mathrm{VC}$. Then with probability at least $1-\delta$: 
  \begin{equation}
    \label{eq:VC_bound}
    L(h) \leq \hat{L}(h, S) + \sqrt{ \frac{8}{N} \ln \left( \frac{4}{\delta} \left( \left(2N \right)^{d_\mathrm{VC}} + 1 \right)  \right)} .
  \end{equation}
\end{theorem}
Equation \eqref{eq:VC_bound} states that the out of sample error $L(h)$ is bounded from above by the empirical error $\hat{L}(h, S)$ and the $\sqrt{ \boldsymbol{\cdot}}$ which is related to the complexity of the hypothesis space $\mathcal{H}$, the number of samples $N$ and the certainty $\delta$. We will call this model complexity penalty $\Omega(N, \mathcal{H}, \delta)$:
\begin{equation}
  \Omega(N, \mathcal{H}, \delta) = \sqrt{ \frac{8}{N} \ln \left( \frac{4}{\delta} \left( \left(2N \right)^{d_\mathrm{VC}(\mathcal{H})} + 1 \right)  \right)}.
\end{equation}
As the hypothesis space complexity $d_\mathrm{VC}$ grows, the generalization bound loosens but it is more likely that $\mathcal{H}$ contains a strong hypothesis. This relationship is called the approximation-estimation or the  \emph{bias-variance} tradeoff. When the model is too simple to properly fit the complexity in the data it is called \emph{underfitting}\sidenote{Here the error from $\hat{L}(h, S)$ dominates.}, when the model is so complex that it starts fitting the inherent noise in the data it is called \emph{overfitting}\sidenote{Here the error from $\Omega(N, \mathcal{H}, \delta)$ dominates.}. The loss as a function of model complexity gives the characteristic curve illustrated in Figure~\ref{fig:ml:empirical_risk}. As the model complexity increases the training loss decreases. Initially also the validation loss decreases, but at some point the behavior of model on the validation set worsens and the loss increases; overfitting happens. 

\begin{marginfigure}
  \includegraphics[width=0.98\textwidth]{figures/overfitting/overfitting_1.pdf}
  \caption[Approximation-Estimation tradeoff]
    {Illustration of the empirical loss as a function of model complexity. The \textcolor{blue}{training error} is shown in blue and \textcolor{red}{validation error} in red.
    }
  \label{fig:ml:empirical_risk}
\end{marginfigure}


\section{Avoiding overfitting}
\label{sec:ml:overfitting}
Avoiding overfitting is one of the most important issues in machine learning. By now, most modern machine learning algorithms have the inherent model complexity needed for overfitting and thus it has to be managed. Due to the importance of the issue, a number of different methods preventing or reducing overfitting exists. Most of them are not complementary of each other but can be taken advantage of in a combination. In this section model regularization will be introduced in \autoref{subsec:regularization}, cross validation in \autoref{subsec:cross_validation}, and early stopping in \autoref{subsec:early_stopping}. 

\subsection{Model Regularization}
\label{subsec:regularization}
One of the earliest methods developed for preventing overfitting was model regularization. A. N. Tikhonov \citep{tikhonovStabilityInverseProblems1943} was one of the first to describe this method in \num{1943}. In particular regularization was used to solve \emph{ill posed} linear regression problems. Regular linear regression problems refer to minimizing the residual sum of squares written in matrix form as:
\begin{equation}
  \begin{split}
    \hat{\bm{\beta}}_{\mathrm{LS}} = \argmin_{\bm{\beta}} \norm{\vec{y} - \vec{X} \bm{\beta} }_2^2 = \argmin_{\bm{\beta}} \norm{ \vec{y} - \vec{f}(\vec{X}) }_2^2 \\
    \vec{f}(\vec{X}) = \vec{X} \bm{\beta},
  \end{split}
\end{equation}
where $\vec{y}$ is vector of values we are trying to predict\sidenote{E.g. the prices of a collection of houses.}, $\vec{X}\in\mathbb{R}^{N\times M}$ is the matrix of input parameters such that each observation $\vec{x}_i \in \mathbb{R}^M$ is a $M$-dimensional vector stacked in $\vec{X}$ as rows with $N$ total observations, $\hat{\bm{\beta}}_{\mathrm{LS}}$ is the vector of unknown coefficients\sidenote{Here excluding the constant offset $\beta_0$ which can be included trivially.} of the linear least squares (LS) model $\vec{f}$, and $\norm{ \bm{\cdot} }_2$ is the normal Euclidean norm. In general, the $p$-norm is defined as:
\begin{equation}
  \norm{\vec{x}}_p = \left( \sum_{i=1}^N \abs{x_i}^p \right)^{1/p}.
\end{equation}
Differentiating the objective $\norm{\vec{y} - \vec{X} \bm{\beta} }_2^2$ with respect to (w.r.t.) $\bm{\beta}$ and setting the derivative equal to $0$ to find the minimum\sidenote{When checking the double derivative wrt. $\bm{\beta}$ it is seen that this really is a minimum and not a maximum (or saddle point).} yields the solution for $\bm{\beta}$:
\begin{equation}
  \begin{split}
    \frac{\partial}{\partial \bm{\beta}} \norm{\vec{y} - \vec{X} \bm{\beta} }_2^2 = -2 \vec{X}^T \left( \vec{y} - \vec{X} \bm{\beta} \right) = 0  \Rightarrow \\
    \vec{X}^T \vec{y} - \vec{X}^T \vec{X} \bm{\beta} \Rightarrow \\
    \hat{\bm{\beta}}_{\mathrm{LS}} =  \left( \vec{X}^T \vec{X} \right)^{-1} \vec{X}^T \vec{y}.
  \end{split}
\end{equation}
However, this solution for $\hat{\bm{\beta}}_{\mathrm{LS}}$ is only valid when $\vec{X}^T \vec{X}$ is invertible, i.e. $\vec{X}$ has to be full rank \citep{hastieElementsStatisticalLearning2009}. If this is not the case, the problem is said to be ill posed. Tikhonov solved this problem by adding an extra term to the minimization problem, which we will call $\Omega$ for simplicity. For a specific choice of $\Omega$, one gets:
\begin{equation}
  \label{eq:l2_norm}
  \hat{\bm{\beta}}_{\mathrm{L_2}} = \argmin_{\bm{\beta}} \left \{\norm{ \vec{y} - \vec{f}(\vec{X}) }_2^2 + \lambda \norm{ \bm{\beta} }^2_2 \right \},
\end{equation}
where $\lambda \geq 0$ is the regularization strength. This is the so-called $L_2$-regularization, also known as ridge regression for linear problems. For ridge regression\sidenote{Here $\hat{\bm{\beta}}_{\mathrm{L_2}}$ is the solution for any general function $\vec{f}$ and $\hat{\bm{\beta}}_{\mathrm{ridge}}$ is the specific solution for a linear function $\vec{f}$, where linear is w.r.t. to the model parameters $\bm{\beta}$.} the corresponding solution for $\bm{\beta}$ is:
\begin{equation}
  \label{eq:l2_norm_linear}
  \hat{\bm{\beta}}_{\mathrm{ridge}} =  \left( \vec{X}^T \vec{X} + \lambda \vec{I} \right)^{-1} \vec{X}^T \vec{y},
\end{equation}
where $\vec{I}$ is the identity matrix\sidenote{The $\lambda \vec{I}$ in equation \eqref{eq:l2_norm_linear} also acts as a conditioner on the problem in the sense that it reduces the condition number of the matrix to be inverted turning the ill-posed problem to a well-behaved one.}. This extra term, $\Omega =  \lambda \norm{ \bm{\beta} }^2_2$, acts as a shrinkage factor on the coefficients of $\bm{\beta}$. Looking at equation \eqref{eq:l2_norm}, we see that this is the Lagrangian form of the equivalent problem: 
\begin{equation}
  % \label{eq:ml:constrained_minimization_L2}
  \begin{split}
    \hat{\bm{\beta}}_{\mathrm{L_2}} = \argmin_{\bm{\beta}} \norm{ \vec{y} - \vec{f}(\vec{X}) }_2^2 \\
    \mathrm{subject~to:} \quad \sum_{i=1}^N \beta_i^2 = \norm{\bm{\beta}}^2_2 \leq t,
  \end{split}
  \label{eq:l2_norm_non_lagrangian}
\end{equation}

for some $t \geq 0$ with a one-to-one mapping between $\lambda$ and $t$. We thus see that $L_2$ regularizes the coefficients of $\bm{\beta}$ to have some maximal norm. The effect of the regularization is controlled by $\lambda$, where $\hat{\bm{\beta}}_{\mathrm{L_2}} \rightarrow \hat{\bm{\beta}}_{\mathrm{LS}}$ for $\lambda \rightarrow 0$ and $\hat{\bm{\beta}}_{\mathrm{L_2}} \rightarrow \vec{0}$ for $\lambda \rightarrow \infty$. An example of this can be seen in Figure~\ref{fig:ml:regularization_ridge}. 
\begin{marginfigure}
  \includegraphics[width=0.99\textwidth, trim=5 5 5 5, clip]{figures/ridge_regression/ridge.pdf}
  \caption[Regularization Effect]
    {Effect of tuning the regularization strength $\lambda$ in ridge regression.
    }
  \label{fig:ml:regularization_ridge}
\end{marginfigure}
Here $N=9$ datapoints were randomly generated such that the $x$-values are evenly spaced from \num{0} to \num{1} and $y \sim \mathcal{N}(\mu=0, \sigma=10)$. They were then fit with a \num{9}-order polynomial by minimizing equation \eqref{eq:l2_norm} for different values of $\lambda$. Here we see the regularizing effect of $\lambda$, going from $\lambda=0$ in blue which fits all points\sidenote{Since the order of the polynomial is the same as the number of datapoints.} with a high degree of variance and a wildly oscillatory pattern to $\lambda=10$ in orange which is mostly flat in most of the interval. Also note in the legend how the norm of the fit parameters also decrease with $\lambda$ as expected. This is a great example of the \emph{bias-variance} tradeoff mentioned in \autoref{subsec:generalization_bound_infinite}, where bias refers the error the model makes when it is not advanced enough to fit the overall trend in the data (underfitting) and variance refers to the error the model makes when it starts to fit spurious noisy fluctuations in the training set which are not present in the validation set (overfitting). In this example $\lambda=0$ is clearly overfitting the data whereas $\lambda=10$ is an example of underfitting. 

\begin{marginfigure}
  \includegraphics[width=0.7\textwidth]{figures/ridge_lasso_sparse/ridge.pdf}
  \caption[Regularization Effect of $L_2$ ]
    {Sketch of the minimization problem defined in equation \eqref{eq:l2_norm_non_lagrangian}, i.e. for a $L_2$-penalty. The \textcolor{red}{constrain region} shown in red is defined as $\beta_1^2 + \beta_2^2 \leq t$ for $L_2$ in $2D$-space and the contours of the unconstrained solution is shown with grey, dashed lines. 
    }
  \label{fig:ml:regularization_effect_ridge}
\end{marginfigure}
\begin{marginfigure}
  \includegraphics[width=0.7\textwidth]{figures/ridge_lasso_sparse/lasso.pdf}
  \caption[Regularization Effect of $L_1$]
    {Sketch of the similar minimization problem defined in Figure~\ref{fig:ml:regularization_effect_ridge} for the $L_1$-penalty. The \textcolor{red}{constrain region} shown in red is defined as $\abs{\beta_1} + \abs{\beta_2} \leq t$ for $L_1$ in $2D$-space and the contours of the unconstrained solution is shown with grey, dashed lines.
    }
  \label{fig:ml:regularization_effect_lasso}
\end{marginfigure} 

Other values of the regularization function $\Omega$ exists, for example the $L_1$-penalty:
\begin{equation}
  \Omega = \lambda \norm{\bm{\beta}}_1, 
\end{equation}
where the 1-norm, also known as Manhattan norm, is used. In the case of linear problems the $L_1$-penalty leads to Lasso regression introduced by \citet{tibshiraniRegressionShrinkageSelection1996} in \num{1996}. As with the $L_2$-penalty, the $L_1$-penalty also regularizes the coefficients of $\bm{\beta}$, however, this loss leads to sparse\sidenote{Meaning that a number of the $\beta_i$ coefficients are 0, the number depending on $\lambda$.} solutions. An illustration of this can be seen in Figure~\ref{fig:ml:regularization_effect_ridge} and Figure~\ref{fig:ml:regularization_effect_lasso} where the constraint regions of $\bm{\beta}$ os shown in red and the grey ellipses are the contours of the non-constrained problem. Notice how the intersection of the contour lines and the constrain region leads to $\beta_1 \neq 0, \beta_2 \neq 0$ for the $L_2$-penalty whereas it leads to the sparse solution $\beta_1=0, \beta_2 \neq 0$ for the $L_1$-penalty. This is a general pattern seen for $L_p$-penalties for $p \leq 1$.

Overall, model regularization is heavily used in modern machine learning algorithms. In general, the function or the so-called \emph{objective function} $\mathcal{L}$ they are trying to minimize is:
\begin{equation}
  \mathcal{L}(h) = \hat{L}(\ell, h, S) + \Omega(h)
\end{equation}
where $\hat{L}$ is the empirical loss and $\Omega$ is the regularization penalty. As can be seen from the above discussion, choosing the right value for the regularization strength is fundamental problem in model regularization. How to choose a suitable value for $\lambda$ is discussed in \autoref{subsec:cross_validation} and the choice of the training loss function $\ell$ in \autoref{sec:ml:loss_function}. 

\subsection{Cross Validation}
\label{subsec:cross_validation}
In general we want to  be able to estimate the performance\sidenote{\q{Performance} is used here as the word for the general metric to be optimized for, no matter whether or not this metric should be maximized or minimized.} of the developed model. Since evaluation the model on the data it was already trained on would give a biased estimate of the performance, we need an unbiased method of doing so. The easiest way of doing so would be to set a fraction of the data aside, e.g. \SI{20}{\percent}, train on the remaining part and then evaluate the performance on the data set aside. Splitting the data up like this would provide us with a training (data)set and a test set in a $\num{80}:\num{20}$ ratio. This would then yield an unbiased performance estimate when evaluation the performance on the test set. 
However, if one then needed to compare two different models and choose the best one, as is often the case, this method would not work since we would choose the model with the best performance on the test set which can be seen as training on the test set and thus it has \q{tainted} the purity of the test set. To avoid this, an additional split is made such that we get a training set, a validation set, and a test set, where you often see a $\num{80}:\num{10}:\num{10}$ ratio. The two models can then be compared on the validation set and the performance of the chosen model can be estimated from the test set. 

\begin{marginfigure}
  \includegraphics[width=0.98\textwidth, trim=5 5 5 5, clip]{figures/cross_validation/kfold.pdf}
  \caption[$k$-Fold Cross Validation]
    {$k$-fold cross validation. 
    }
  \label{fig:ml:cross_val_kfold}
\end{marginfigure} 

\begin{marginfigure}
  \includegraphics[width=0.98\textwidth, trim=10 10 10 10, clip]{figures/cross_validation/kfold_time.pdf}
  \caption[$k$-Fold Cross Validation for Time Series Data]
    {$k$-fold cross validation for time series data. 
    }
  \label{fig:ml:cross_val_kfold_time}
\end{marginfigure} 

This way of splitting up the data has some clear benefits and is thus also often used. There is a drawback, however, and that it that we are not fully utilizing a lot of the data in this way. Basically\SI{20}{\percent} of the data are only used to provide a single number of performance and does not necessarily allow an uncertainty or confidence interval of this measurement to calculated. Thus other methods of estimating model performance are developed where one of the most used and well-known are the $k$-fold cross validation (CV). Here the entire dataset are split up into $k$ chunks which are randomly drawn subsamples (without replacement). In the first iteration, the model is trained on the first $k-1$ subsamples and evaluated on the last $k$ subsample. In the second iteration the evaluation subsample is a new one. This process is continued $k$ times until all samples in the dataset have been trained and evaluated on \citep{hastieElementsStatisticalLearning2009}. For an illustration of this, see Figure~\ref{fig:ml:cross_val_kfold}. The process yields $k$ estimates of the performance of the model which can then be averaged to form a single performance number and the variability of the performance can even be gauged\sidenote{Special care has to be taken here since the $k$ different performance values are not independent.}. The disadvantage of $k$-fold CV is that the performance estimate is now slightly biased, however, this effect is generally very small. The biggest disadvantage is the computational burden related to doing $k$-fold CV where $k\gg 1$. A compromise often used in applied ML is $k=5$ which is also what is used in this project.  

Special care has to be taken when dealing with time series data. Here the problem of \q{data leakage} is often introduced inadvertently. Data leakage is when the model is exposed to information from the test set that it was not supposed to be exposed to. In the case of time series data, if the data is split by the usual $k$-fold CV, then each subsample contains events from all times and the model does not learn how to predict future events. To circumvent this problem, a special type of $k$-fold CV for time series data has to be used. Here all samples up to a specific time, eg. all houses sold before 2018, is used for training and then it is evaluated on the performance of samples after the event, e.g. houses sold in \num{2018}. For an illustration of this, see Figure~\ref{fig:ml:cross_val_kfold_time}.


\subsection{Early Stopping}
\label{subsec:early_stopping}
Most modern machine learning models are trained iteratively. This is the case for both (boosted) decision trees and neural networks, both of which are used in this project. Iteratively here means that the model starts off with an initial guess of the parameters of the model to learnt and then by looking at the data \q{learns} a new set of values for the parameters. The question then become: for how long should the model be allowed to continue training. 

This is a prime example of the bias-variance tradeoff. The model should be trained long enough to be able to capture the complexity inherent in the data but also should not train for so long that it starts to overfit the data. Even though Figure~\ref{fig:ml:empirical_risk} was just an illustration of the bias variance tradeoff, it is also something that is seen in real data and can taken advantage of through \emph{early stopping}. Early stopping is the process of monitoring the loss for training set and validation set $\mathcal{D}_\mathrm{train}$ and $\mathcal{D}_\mathrm{val}$. As mentioned in \autoref{subsec:cross_validation}, the model is only fitted on $\mathcal{D}_\mathrm{train}$ but the performance on $\mathcal{D}_\mathrm{val}$ is also measured. Whenever the validation loss starts to increase, the training of the model should be terminated. 

To avoid stopping just because of a single outlier due to noise that terminated the process, one often uses \emph{patience} in the early stopping process: if the loss has not decreased since the last minimum after \emph{patience} number of iterations, then terminate the process. Early stopping is thus an easy way of avoiding overfitting for iteratively trained models only requiring a validation set.  

\section{Loss functions}
\label{sec:ml:loss_function}
How we evaluate the performance of a model is of course very important since it defines the metric for the problem; what is good and what is bad? Obviously this depends on whether or not we a dealing with a classification problem or a regression problem. Let us for know focus on the latter. Say that a house is estimated to cost \num{2} million DKK (\si{\Mkr}) but was sold for \SI{4}{\Mkr}. Compare this to an a house that was estimated to cost \SI{8}{\Mkr} but was sold for \SI{6}{\Mkr} In both cases the price was \SI{2}{\Mkr} wrong, but does this mean that both predictions are equally good? The first case was a factor of 2 off, whereas the second was only $25\%$ off. The first case underestimated the price whereas the second overestimated. 

As it should be clear by now the choice of loss function $\ell$ is of utmost importance. It is also not a problem that can be solved by computers, is problem-specific, and has to be defined manually. The choice of loss function is what is called a \emph{hyperparameter}, the optimization of which is further discussed in \autoref{sec:ml:hyperparameter_optimization}. The most common choice of loss function is by far the Squared Error (SE): 
\begin{equation}
  \ell_\mathrm{SE}(y, \hat{y}) = \left( y-\hat{y} \right)^2,
\end{equation}
where $y$ is the true value and $\hat{y}$ is the predicted one. Squared Error has the advantage that it is differentiable everywhere, an effect that is both needed for many statistical derivations but also a requirement for some machine learning models. The disadvantage is that it gives too much weight to outliers since every deviation away from the truth is squared. In contrast to this, there is the Absolute Error (AE) defined as:
\begin{equation}
  \ell_\mathrm{AE}(y, \hat{y})  = \abs{y-\hat{y} }.
\end{equation}

For AE, outliers have a lot smaller weight since it deals with the absolute value of the deviation and not the squared deviation. However, this comes at a price; AE is not differentiable at every point: at $y-\hat{y} = 0$ the derivative of the absolute value function is un-defined. Many functions have been invented trying to deal with these problems. For a more general discussion of loss functions, see e.g. Barron \citep{barronGeneralAdaptiveRobust2017}. Six different loss functions (for regression problems) have been investigated in this project. In addition to SE and AE also the LogCosh, Cauchy  \citep{barronGeneralAdaptiveRobust2017}, Welsch  \citep{barronGeneralAdaptiveRobust2017} and Fair  \citep{AllstateClaimsSeverity} loss functions are used:
\begin{equation}
  \begin{split}
    \ell_\mathrm{LogCosh}(y, \hat{y})  &= \log\left( \cosh\left( y-\hat{y} \right) \right) \\
    \ell_\mathrm{Cauchy}(y, \hat{y})  &= \log\left( \frac{1}{2} \left(\frac{y-\hat{y}}{c}\right)^2 + 1   \right) \\
    \ell_\mathrm{Welsch}(y, \hat{y})  &=  1 - \exp\left( - \frac{1}{2} \left(\frac{y-\hat{y}}{c}\right)^2  \right)\\
    \ell_\mathrm{Fair}(y, \hat{y})  &= c^2  \left( \frac{\abs{y-\hat{y}} }{c}  - \log \left(\frac{\abs{y-\hat{y}}}{c} +1 \right )   \right). 
  \end{split}
\end{equation}
The above loss functions share some similarities with AE, and in addition to this they are all (twice) differentiable functions. They are shown in Figure~\ref{fig:ml:objective_funcs}. They are shown for only positive values of $y-\hat{y}$ since they are symmetric in $y-\hat{y}$. Notice how SE quickly grows very large compared to the others. Absolute Error has a kink at $y-\hat{y}=0$ as the only one of the functions. Welsch is bounded in the interval $[0, 1)$. The derivative of both LogCosh and Fair goes toward $1$ when $y-\hat{y}$ goes towards $\infty$, whereas it goes to 0 for the Cauchy loss. A priori it is almost impossible to know which one of these loss functions performs best for a specific data set, so they have to be treated as hyperparameters. 

\begin{figure}
  \includegraphics[width=0.9\textwidth]{figures/objective_functions/objective_functions.pdf}
  \caption[Comparison of different objective functions.]
    {Comparison of the the six loss functions SE, AE, LogCosh, Cauchy, Welsch, and Fair as a function of $y-\hat{y}$. In the plot \textcolor{blue}{SE} is shown in blue, \textcolor{red}{AE} in red, \textcolor{green}{LogCosh} in green, \textcolor{orange}{Cauchy} in orange, \textcolor{purple}{Welsch} in purple, and \textcolor{light-blue}{Fair} in light blue. For the Cauchy, Welsch, and Fair functions $c$ is set to 1. For a zoom in of the inner region 
    where $y-\hat{y}<2$ see \figref{fig:ml:objective_funcs_zoom}. All six graphs are symmetric in $y-\hat{y}$ which is why they are only shown for positive values of $y-\hat{y}$.
    }
  \label{fig:ml:objective_funcs}
\end{figure}

\begin{marginfigure}
  \includegraphics[width=0.98\textwidth]{figures/objective_functions/objective_functions_zoom.pdf}
  \caption[Comparison of different objective functions zoom in.]
    {Zoom in of \figref{fig:ml:objective_funcs}. 
    }
  \label{fig:ml:objective_funcs_zoom}
\end{marginfigure}

\subsection{Evaluation Function}
Since some machine learning models require an analytic expression for the derivative and second derivative, it not always possible to use a custom objective function if it is non-differentiable. The Area Under Curve (AUC) of the Receiver Operating Characteristic (ROC) is such a measure. Another example would be the width of the distribution of all $y_i-\hat{y}_i$. In this thesis this overall performance metric will be called the \emph{evaluation function} $f_\mathrm{eval}$ compared to the differentiable proxy for this function, the objective function. To sum up: the loss function $\ell(y, \hat{y})$ measures the loss for an individual prediction and the objective function $\mathcal{L}$ is the aggregated version of the individual losses. The objective function is assumed to be a good proxy for the evaluation function. 

For the loss functions defined in \autoref{sec:ml:loss_function} the objective function is based on the mean of the individual losses:
\begin{equation}
  \mathcal{L} = \frac{1}{N} \sum_{i=1}^N \ell(y_i, \hat{y}_i) + \Omega. 
\end{equation}

\section{Decision Trees}
\label{sec:ml:decision_trees}

\begin{marginfigure}[-3cm]
  \includegraphics[width=0.99\textwidth, trim=10 10 10 10, clip]{figures/decision_tree/tree_example.pdf}%     without .tex extension
  % or use \input{mytikz}
  \caption[Decision Tree Cuts In Feature Space]{Illustration of the cuts a decision tree model make for \textcolor{blue}{signal} in blue circles and \textcolor{red}{background} in red squares. This is an visualization in the feature space of the decision tree seen in Figure~\ref{fig:ml:decision_tree}.}
  \label{fig:ml:decision_tree_feature_space}
\end{marginfigure}

Decision Trees are a simple machine learning method that works by partitioning the feature space into smaller subspaces, high-dimensional rectangles basically, and then fit each subspace with a constant for regression problems or a single label for classification problems \citep{hastieElementsStatisticalLearning2009}. A simple example of this can be seen in Figure~\ref{fig:ml:decision_tree_feature_space} and Figure~\ref{fig:ml:decision_tree}. In the first figure we see an illustration of how the signal and background distributions look in the 2D feature space. The dashed lines in the figure indicate the cuts made by the decision tree (DT), cuts which are shown on the second figure as a typical DT plot. 
\begin{figure}
  \centering
  \includestandalone[width=0.6\textwidth]{figures/decision_tree/decision_tree}%     without .tex extension
  % or use \input{mytikz}
  \caption[Decision Tree]{
    Illustration of a simple decision tree. Here the tree partitions the input feature space consisting of the two variables $p_\perp$ and $E_\mathrm{jet}$ into two categories; either signal or background. A visualization of the cuts in the feature space can be seen in Figure~\ref{fig:ml:decision_tree_feature_space}.
  }
  \label{fig:ml:decision_tree}
\end{figure}
Here the first \q{box} is called the \emph{root} of the tree, any subsequent boxes are \emph{nodes} except the final ones that are not split any further which are \emph{leaves}.

At first the DT partitions the space according the the value of the transverse momentum\sidenote{All units in this project are in natural units such that both momentum and energy are in units of \si{\eV}.} $p_\perp$: if it is lower than (or equal to) \SI{20}{\GeV} then it classified as background. If the value is higher than \SI{20}{\GeV} then an extra split is made, this time on the energy of the jet $E_\mathrm{jet}$: if it is higher than \SI{10}{\GeV} it is classified as signal and otherwise as background. Training a DT on this data allows us to predict a new unseen event $(p_\perp, E_\mathrm{jet}) = (\SI{24}{\GeV},\SI{11}{\GeV})$ to be a signal-like event. This DT is said to be a shallow tree since it only has a depth of \num{2}. The maximum depth allowed for the model is an important hyperparameter since it clearly controls under- and overfitting by changing how many cuts and partitions in the feature space are allowed; the deeper the tree, the more complex the model becomes. Single DTs are very prone to overfitting\sidenote{With a solution to this problem given in \autoref{subsec:ml:multiple_decision_trees}.}, however, they are also extremely inspectable. They are even referred to as \q{white-box models} compared to black-box models such as neural networks. For a more thorough introduction to decision trees and how they are internally optimized (for finding the best cut values), see \citet{hastieElementsStatisticalLearning2009}. 

\subsection{Ensembles of Decision Trees}
\label{subsec:ml:multiple_decision_trees}
Single decision trees are prone to overfitting and generally suffer from high variance. Today especially two different methods exist to alleviate these problems: Random Forests (RFs) and Boosted Decision Trees (BDTs). Both methods are examples of so-called ensemble methods where a finite set of ML methods are combined into a single model. Typically ensemble methods are based on \emph{weak learners}: simple, often fast, methods that individually show relatively poor generalization performance typically due to variance. 

\subsubsection{Random Forests}
\label{subsubsec:ml:random_forest}
Random Forests were first introduced in 2001 by \citet{breimanRandomForests2001}. Random Forests are a collection of $B$ decision trees where each tree is trained on bootstrapped versions of the training data and the average\sidenote{In case of classification, it is the majority vote which is taken.} the individual trees' predictions $T_b$: 
\begin{equation}
  f_\mathrm{RF}(\vec{x}) = \hat{y}_\mathrm{RF} = \frac{1}{B}  \sum_{b=1}^B T_b(\vec{x}).
\end{equation}
The method of making artificial extra samples and training on them is in general called bootstrap aggregation or \emph{bagging}\autocite{hastieElementsStatisticalLearning2009}. It works by averaging out noisy estimates of the individual models hence reducing variance. 
\begin{theorem}[Variance of average of correlated i.d. variables]
  Given $B$ identically distributed (but not necessarily independent) variables each with variance $\sigma^2$ and positive pairwise correlation $\rho$, the variance of the average is:
  \begin{equation}
    \label{eq:ml:variance_of_correlated_variables}
    \mathrm{Var}(\bar{\mu}_B) = \frac{1-\rho}{B} \sigma^2 + \rho \sigma^2,
  \end{equation}
  where $\bar{\mu}_B$ is the average of the i.d. variables \autocite{hastieElementsStatisticalLearning2009}. % XXX myabe write proof here myself: https://stats.stackexchange.com/questions/320083/calculating-the-variance-of-the-average-of-b-dependent-random-variable
\end{theorem}
Equation \eqref{eq:ml:variance_of_correlated_variables} is the main idea behind RFs. As the number of trees $B$ in the forest increases, the first term goes towards zero. Thus, the more the correlation $\rho$ between the individual trees is reduced, the more the variance is reduced (which is the main problem in the case of DTs to begin with). 

In addition to training the trees on bagged samples, one more technique is used to further decrease the correlation between the individual trees: each bootstrapped sample of the dataset not only contains a only subset of the observations (rows), but also only a subset of the variables (columns)\sidenote{Throughout this project all data is assumed to be \emph{tidy data} unless otherwise explicitly mentioned. Tidy data was a concept formalized in 2003 by \citet{JSSv059i10} which basically states that each variable forms a columns, each observation forms a row, and that each type of observation unit forms a table. This also means that the term variable and column will be used interchangeable (together with \emph{feature}), and likewise with observation and row.}. This called column-subsampling (in contrast to row subsampling).

% Until very recently RF has suffered from the lack of a fast, optimized implementation in Python that could deal with big data sets. 

\subsubsection{Boosted Decision Trees}
\label{subsubsec:ml:boosted_decision_trees}
In the overall family of ensemble models, \emph{boosting} might be the most successful of them all where especially the specific algorithm called XGBoost \autocite{chenXGBoostScalableTree2016} revolutionized the ML world by winning numerous Kaggle\sidenote{XXX \TODO} competitions \autocite{DmlcXgboost} including the Higgs Machine Learning competition in \num{2014} \autocite{HEPMeetsML}. 

Boosting is the process of sequentially applying weak learner models to repeatedly modified versions of data \autocite{hastieElementsStatisticalLearning2009}. In an iterative fashion this combines many weak learners into a single strong learner. The final prediction is thus a weighted sum over the $M$ different weak learners $F_m$: 
\begin{equation}
  f_\mathrm{BDT}(\vec{x}) = F(\vec{x}) = \sum_{m=1}^M \alpha_m F_m(\vec{x})
\end{equation}
Boosting thus works by reducing both bias and variance since it iteratively fits weak learners. The variance is not reduced as much as for RFs since the weak learners are more correlated, however, their bias is lower. 

The term gradient boosting comes from the observation that repeatedly minimizing the residuals of the current model is similar to minimizing the gradient of the loss function for a specific choice of the loss function.

Imagine that we start off with an imperfect model $F_m(x)=\hat{y}$. In boosting we want the next iteration to be a better model than the previous one, so image that the perfect addition to the model that we needed to make was $h(x)$. For it to be perfect, the following would have to be true:
\begin{equation}
  \label{eq:ml:boosted_decision_trees_residual}
  F_{m+1}(x) = F_m(x) + h(x) = y \Leftrightarrow h(x) = y - F_m(x).
\end{equation}
The r.h.s. is the residual of the model, so at each stage the model is trying to fit the residuals of the current iteration of model. The \q{gradient} in \q{gradient boosting} comes from the following. Assume the loss function: $L(y, F_m(x)) = \frac{1}{2}  (y-F_m(x))^2$. The gradient of the loss w.r.t. to $F_m(x)$ is:
\begin{equation}
  \frac{\partial L(y, F_m(x))}{\partial F_m(x)} = F_m(x) - y. 
\end{equation}
This is exactly the negative of the r.h.s. of equation \eqref{eq:ml:boosted_decision_trees_residual}. Instead of looking at the model as trying to minimize the residual at each iteration, it can instead be generalized as trying to fit the negative gradients of the loss function:
\begin{equation}
  h(x) = - \frac{\partial L}{\partial F_m}.
\end{equation}
We thus end up with the following iterative model which is basically a \emph{gradient descent} algorithm.
\begin{equation}
  F_{m+1}(x) = F_m(x) + h(x) = F_m - \frac{\partial L}{\partial F_m}.
\end{equation}


AdaBoost \autocite{freundDesiciontheoreticGeneralizationOnline1995} was the first major algorithm to make use of boosting. It was seen as a way of iteratively giving wrongly predicted observations higher weight, however, this is just a result of a \q{lucky} choice of loss function\sidenote{Exponential loss for classification.} which was not realized until much later. AdaBoost could be used with many different weak learners, however mostly DTs were used to form BDTs. XGBoost \autocite{chenXGBoostScalableTree2016} is a fast, computationally efficient implementation of gradient boosting with DTs as base learners. It implements several model regularization techniques which makes it less prone to overfitting than other BDTs. 

\section{Hyperparamater Optimization}
\label{sec:ml:hyperparameter_optimization}
By now linear models with $L_1$ and $L_2$ regularization have been introduced along with decision trees (DTs), random forests (RFs) and gradient boosted trees (GBTs). All of these ML models tries to optimize their parameters according to some objective function. In addition to the parameters of the model, each one has a specific set of \emph{hyperparameters} that cannot directly be optimized in internal optimization process. This could be amount of regularization $\lambda$ for linear models, the maximum tree depth for DTs, the number of trees for RFs, or the column (or row) subsampling fraction for BTDs. 

In general we say that we have the ML model $\mathcal{A}$ with $K$ hyperparameters. Each of these hyperparameters have a domain $\Lambda_n$. The domain can be either real numbers $\Lambda_k \in \mathbb{R}$, integers $\Lambda_k \in \mathbb{Z}$, binary $\Lambda_k \in \{0, 1\}$, or categorical\sidenote{Could e.g. be the choice of loss function.}. We define the hyperparameter configuration space as: $\bm{\Lambda} = \Lambda_1 \times \Lambda_2 \times \dots \times \Lambda_K$. Within this space we are searching for a vector of hyperparameters $\bm{\lambda} \in \bm{\Lambda}$ which defines the optimal model $\mathcal{A}_{\bm{\lambda}^*}$. Here the optimal model is the defined as the model which gives the best generalization performance according to some evaluation function. The goal of finding the best hyperparameter $\bm{\lambda^*}$ is known as \emph{hyperparameter optimization} (HPO.)

The first naive approach would simply to manually\sidenote{Also known as jokingly as \q{Grad Student Descent}.} try out different combinations of $\bm{\lambda}$ and see performance on the validation set\sidenote{Remember only to use the test set on the final model!}. This is of course too cumbersome for advanced ML models, but it should be noted that it is a good place to start. In \autoref{subsec:ml:grid_search} the HPO method called Grid Search is introduced which is further generalized and optimized in \autoref{subsec:ml:random_search} with Random Search. Both these methods are easily parallelizable since they do not have any inherent history in its guesses. This is in contrary to Bayesian Optimization introduced in \autoref{subsec:ml:bayesian_optimization} which allows for \q{smart} guesses.

\subsection{Grid Search}
\label{subsec:ml:grid_search}

Grid Seach (GS) is a HPO method also known as full factorial design. It is called this because it tries out all possible combinations of the hyperparameter configuration space: the so-called cartesian product of $\bm{\Lambda}$. Imagine a 2D space where the two domains are respectively $x=\{1, 2, 3\}$ and $y=\{1, 2, 3,4\}$. Then GS runs through all \num{12} combinations of these two sets: 

\begin{marginfigure}
  \includegraphics[width=0.8\textwidth]{figures/gridsearch/grid.pdf}
  \caption[Grid Search]{Visualization of grid search run on the two hyperparameters $x$ and why $y$ with the domains $x=\{1, 2, 3\}$ and $y=\{1, 2, 3,4\}$.}
  \label{fig:ml:gridsearch}
\end{marginfigure}

\begin{equation}
  \{(1, 1), (1, 2), \dots, (x_i, y_i), \dots, (3, 4)\},
\end{equation}
as visualized in Figure~\ref{fig:ml:gridsearch}. The advantage of GS is that it is an exhaustive search over all combinations\sidenote{Note that the user has to provide the values for each hyperparamater to be tried out manually.} of hyperparameters, however, the total number of combinations grows exponentially and GS as a method thus suffers the curse of dimensionality\sidenote{Not the dimensionality of the input feature space, but of the hyperparameter configuration space.}.



\subsection{Random Search}
\label{subsec:ml:random_search}

To circumvent the problems of grid search, \citet{bergstraRandomSearchHyperparameter2012} developed the Random Search (RS) algorithm in 2012. Regarding the effect of the curse of dimensionality on grid search they wrote: \emph{``This failure of grid search is the rule rather than the exception in high dimensional hyper-parameter optimization''} \citep{bergstraRandomSearchHyperparameter2012}. Instead of searching through all possible values of $\bm{\lambda}$ like in GS, RS makes $B$ runs where each $\bm{\lambda}_i$ is given by:
\begin{equation}
  \label{eq:ml:random_search}
  \bm{\lambda}_i \sim \sum_{j=1}^K  \mathrm{PDF}_j(\Lambda_j) \cdot \bm{\hat{e}}_j  .
\end{equation}
Equation \eqref{eq:ml:random_search} should be understood in the following way. For each hyperparameter draw a random number from a user-defined Probability Density Function (PDF) and then let $\bm{\lambda}$ be the vector of those $N$ random numbers. In a 2D-space $\bm{\lambda}_i$ could thus be: 
\begin{align}
  \bm{\lambda}_i \sim &= \begin{bmatrix}
      \mathcal{N}(100, 4) \\
      \mathcal{U}(0, 1)
       \end{bmatrix},
\end{align}
where $\mathcal{N}(100, 2)$ is normal (Gaussian) distribution with mean $\mu=100$ and standard deviation $\sigma^2=4$ and $\mathcal{U}(0, 1)$ is the uniform distribution in the interval $[0, 1]$. Of course the PDF can be a PMF in the case of discrete hyperparameter domains.

\begin{marginfigure}
  \includegraphics[width=0.8\textwidth]{figures/randomsearch/random.pdf}
  \caption[Random Search]{Visualization of the difference between grid search and random search. Adapted from \citet{bergstraRandomSearchHyperparameter2012}.}
  \label{fig:ml:random_search}
\end{marginfigure}

The reason why random search is so powerful is not only because the number of function evaluations $B$ is easily tunable\sidenote{Compared to grid search which tries \emph{all} possible combinations.}, but also due to the fact that often some hyperparameter dimensions are more important than other. Even though the hyperparameter configuration space $\bm{\Lambda}$  might be high-dimensional, it often exhibits \emph{low effective dimensionality} \autocite{bergstraRandomSearchHyperparameter2012}. In the simplest 2D-case this can be written as the following example. Imagine that we want to maximize some evaluation function, e.g. accuracy of the predictions, and the model depends on the two independent hyperparameters $x$ and $y$: $f(x, y)$. In this example assume that $f$ is almost insensitive to $y$ and thus has an effective dimensionality of 1. Then $f(x,y) = g(x) + h(y) \approx g(x)$. For a visualization of this example, see Figure~\ref{fig:ml:random_search}.

Here GS is run with a grid of $3\times3 = 9$ points and RS is similarly run with 9 points drawn from uniform PDFs in the same interval. It easy to see that when the hyperparameter configuration space has a lower effective dimensionality than the actual dimensionality RS is far better at probing the space due to the projections into the sensitive dimensions cover more of these axes than for GS. In general in ML the hyperparameter configuration space has lower effective dimension than its actual dimension, but the different hyperparameters matter in different datasets

In general only a fraction of all hyperparameters matter for any dataset but different hyperparameters matter in different datasets and thus generally RS performs as well as GS or better \autocite{bergstraRandomSearchHyperparameter2012}. 

Note that RS can be seen as a generalization of GS, where GS is the specific example of RS if one uses a multidimensional hypergeometric distribution as PDF where the PDF is reevaluated after each run. 

\subsection{Bayesian Optimization}
\label{subsec:ml:bayesian_optimization}
When optimizing the hyperparameters it often takes a lot of time to evaluate the individual hyperparameters. Remember, that each evaluation consists of fitting $\mathcal{A}_{\bm{\lambda}}$ to the training data and then measure the performance on the validation set. Fitting the model on the training set can often take minutes, if not hours. This process is even slower when using cross validation. The idea behind Bayesian Optimization is that when the ML model, or any other black box function, is expensive\sidenote{With respect to time.} to evaluate then \q{smart} guesses are worth spending a bit of time on developing. The hope is that the time taken to come up with smart guesses is negligible compared to the overall function evaluation time. This is contrary to both GS and RS where each new set of hyperparameters $\bm{\lambda}$ is independent of the value of the evaluation performance.

In Bayesian Optimization (BO) \autocite{brochuTutorialBayesianOptimization2010}, the (unknown) evaluation function is iteratively fitted with a probabilistic surrogate model, most often by Gaussian processes (GPs). Given the fitted surrogate model, an acquisition function is computed. This function is cheap to evaluate and is a measure of where in the hyper-dimensional hyperparameter space there is a highest chance of finding a new good value of $\bm{\lambda}$. The acquisition function has to be chosen manually and especially the tradeoff between \emph{exploitation versus exploration} is particularly important. This value decides how \q{adventurous} or conservative the BO algorithm should be when exploring the evaluation space. 

\begin{marginfigure}
  \includegraphics[width=0.99\textwidth]{figures/bayesian_optimization/bo.pdf}
  \caption[Bayesian Optimization]{XXX \TODO. Adapted from \citet{brochuTutorialBayesianOptimization2010}.}
  \label{fig:ml:bayesian_optimization}
\end{marginfigure}

Bayesian Optimization is better explained by looking at Figure~\ref{fig:ml:bayesian_optimization}. First look at the top plot. This is a plot of the surrogate function in black with uncertainties shown in blue. This is a result of fitting GPs to the two previous points in black, $t=2$. This surrogate function is supposed to fit the evaluation function (called objective in the figure) shown as a dashed black line. Below we see the acquisition function in green. This is a function of the blue curve and the position of its maximum decides where the next guess of $\bm{\lambda}$ should be. With the chosen acquisition function and exploration willingness, we see that the next guess should be slightly to the left of the right-most point. This is a simple 1D toy problem, but one should imagine this happening in a high-dimensional space. After making a new guess, $t=3$ in the middle plot, the acquisition function changes since it learnt that this gave a worse evaluation value than the right-most point. Therefore, the next proposal for $\bm{\lambda}$ is slighty to the right of the right-most point. The process continues like this in an iterative fashion: first fitting GPs to the previous evaluation values and then choosing the next $\bm{\lambda}$ according the acquisition function given the GPs. 

Gaussian Processes provide a posterior distribution given some prior distribution and the datadependent ikelihood. The process of BO is quite technical and mathematical, especially if GPs are new material. For a more in-depth explanation of the topic, see \citet{brochuTutorialBayesianOptimization2010}. The important thing to note is that GPs return not only a posterior mean $\mu(\vec{x})$ but also an uncertainty $\sigma(\vec{x})$, as seen in Figure~\ref{fig:ml:bayesian_optimization} described above. The acquisition function used in this project is the Upper Confidence Bound (UCB):
\begin{equation}
  \mathrm{UCB}(\vec{x}) = \mu(\vec{x}) + \kappa \sigma(\vec{x}),
\end{equation}
where $\kappa \geq 0$ is the parameter\sidenote{Here $\kappa$ can thus be regarded as a hyper-hyperparameter since it controlls how the other hyperparamaters are optimized.} controlling the exploration-exploitation tradeoff. 

Bayesian Optimization has the great benefit of slowly learning the hyperparameter space and making smarter and more educated guesses over time. However, it also comes with the cost of being harder to numerically implement compared to GS and RS\sidenote{Which are basically plug-and-play with Scikit-Learn \autocite{scikit-learn}.}, and parallelization is non-trivial to implement since it by definition is a sequential process. The performance boost is also not guaranteed. 

\section{Feature Importance}
\label{sec:ml:feature_importance}
Having first established in \autoref{sec:ml:supervised_learning} that machine learning algorithms are indeed able to not only learn from data but also to generalize well without overfitting (\autoref{sec:ml:overfitting}), modern ML algorithms such as decision trees, random forests and boosted decision trees were introduced in \autoref{sec:ml:decision_trees} and they were hyperparameter optimized in \autoref{sec:ml:hyperparameter_optimization}, one would expect that one would have a well performing model by now. 

Now comes one of the most important issues in ML today: model inspection. Actually trying to make sense of the learnt model. Why does it predict as it does? Which features or variables are most important according to the model? Model interpretation is still very much active research today with no universally accepted methods. Some methods are model dependent and accurate, others might be model agnostic but slow or only approximations. In this project the focus will be on the so-called \emph{SHAP} values. 

In 2017 \citet{Lundberg:2017} showed that six different previously used methods were all specific instances of a universal underlying method\sidenote{The class of \emph{additive feature attribution methods} \citep{Lundberg:2017}.} and propose SHaplay Additive exPlanation (SHAP) values as a unified measure of feature importance. In 2018-2019 they developed a fast algorithm for computing SHAP values for tree ensemples and showed that previous measures of feature importance heavily used for trees, e.g. \emph{gain}, were \emph{inconsistent} \autocite{lundbergConsistentIndividualizedFeature2019}. That a measure for feature importance is inconsistent means that a model could rely more on feature $A$ than $B$, however, the feature importance would indicate opposite. SHAP values and \emph{permutation} feature importance are both consistent feature importance measure, however, only SHAP allows for individualized\sidenote{Meaning that you can get the feature importances for a single observation compared to only the global, overall feature importances as seen across the entire data set.}, or local, feature importances.  

SHAP values are within the class of additive feature attribution methods, which are functions where the explanation model $g$ is a linear combination of binary variables:
\begin{equation}
  \label{eq:ml:additive_feature_attribution_method}
  g(z') = \phi_0 + \sum_{i=1}^M \phi_i z_i'.
\end{equation}
Here $\phi_i \in \mathbb{R}$ are the feature importances and $z'$ is a binary variable such that $z_i'=1$ if the feature is present and otherwise $z_i' = 0$. SHAP values are based on Shapley regression values known from cooperative game theory \autocite{Shapley1953}. These values are based on the three axioms:
\begin{axiom}[Local Accuracy]
  Local accuracy says that the sum of the feature importances should equal the total reward:
  \begin{equation}
    f(x) = g(z') = \phi_0 + \sum_{i=1}^M \phi_i z_i'.
  \end{equation}
\end{axiom}
Here $f$ is the ML model, $g$ is the explanation model, $x$ is an observation in input feature space, $z'$ is an observation in the binary space as described above, and $\phi_i$ is the feature importance. 
\begin{axiom}[Missingness]
  Missingness means that features missing in the original input feature space (such that $z_i'=0$) should be attributed no importance:
  \begin{equation}
    z_i' = 0 \Rightarrow \phi_i = 0.
  \end{equation} 
\end{axiom}
\begin{axiom}[Consistency]
  \label{axiom:ml:shap_consistency}
  Consistency states if a model is changed such that it relies more on a certain feature, the feature importance of that feature should never decrease. 
\end{axiom}
% \begin{description}
%   \item[Local Accuracy] 
%   Local accuracy says that the sum of the feature importances should equal the total reward:
%   \begin{equation}
%     f(x) = g(x') = \phi_0 + \sum_{i=1}^N \phi_i x_i'.
%   \end{equation}
%   Here $f$ is the ML model, $g$ is the explanation model, $x$ is an observation in input feature space, $x'$ is an observation in the binary space as described above, and $\phi_i$ is the feature importance. 

%   \item[Missingness] Missingness means that features missing in the original input feature space (such that $x_i'=0$) should be attributed no importance:
%   \begin{equation}
%     x_i = 0 \Rightarrow \phi_i = 0
%   \end{equation}
%   \item[Consistency] Consistency states if a model is changed such that it relies more on a certain feature, the feature importance of that feature should never decrease.
% \end{description}
Given these three axioms, \citet{Lundberg:2017} show that the only solution to equation \eqref{eq:ml:additive_feature_attribution_method} is:
\begin{equation}
  \label{eq:ml:shap_feature_importance}
    \phi_i = \sum_{S \subseteq \widetilde{M} \backslash \{i\}} \frac{|S|!(M-|S|-1)!}{M!} \left[ f_x(S \cup \{i\}) - f_x(S) \right] ,
\end{equation}
which one can simplify to:
\begin{equation}
  \label{eq:ml:shap_feature_importance_simplification}
  \begin{split}
    \phi_i        &= \sum_{S \subseteq \widetilde{M} \backslash \{i\}} w(S) \cdot \Delta_{f_x}(S) \\
    w(S)             &\equiv \frac{|S|!(M-|S|-1)!}{M!} \\
    \Delta_{f_x}(S)  &\equiv \left[ f_x(S \cup \{i\}) - f_x(S) \right].
  \end{split}
\end{equation}


In equation \eqref{eq:ml:shap_feature_importance} $\widetilde{M}$ is the \emph{set} of all input features\sidenote{Compared to $M=|\widetilde{M}|$ which is the \emph{number} of all input features.}, $S \subseteq \widetilde{M} \backslash \{i\}$ means a subset $S$ of $\widetilde{M}$ without feature $i$, $S \cup \{i\}$ means the set $S$ with feature $i$ and $f_x(S) = f(h_x(z'))$ where $h_x(z')$ is the mapping function from the binary $z'$ space to the input feature space $x$. In equation \eqref{eq:ml:shap_feature_importance_simplification} the function is simplified to its basic constituents: the difference in performance  between including feature $i$ and not including it, $\Delta_{f_x}$, and its weight $w$. 

To get a better understanding of the different sets in the summation, one could look at the decision tree shown in Figure~\ref{fig:ml:decision_tree}. Here $\widetilde{M}$ would be $\widetilde{M}=\{p_\perp, E_\mathrm{jet} \}$. For the feature $i=p_\perp $ one would thus have:
\begin{equation}
  \begin{split}
    \phi_{p_\vert} = & \sum_{S \subseteq \widetilde{M} \backslash \{p_\perp \}} w(S)  \cdot\Delta_{f_x}(S)  \\
                   = & \sum_{S \in \left[\{\}, \{E_\mathrm{jet}\} \right]} w(S)  \cdot\Delta_{f_x}(S)  \\
                   = & \frac{0! (2-0-1)!}{2!}  [ f_x(\{p_\perp\}) - f_x(\{\})]  \\ 
                     & +    \frac{1! (2-1-1)!}{2!}  [f_x(\{ E_\mathrm{jet}, p_\perp \}) - f_x(\{ E_\mathrm{jet} \})].
  \end{split}
\end{equation}
Whereas $w(S)$ are easily calculated, $\Delta_{f_x}$ depends on the data. As the number of features grows, the number of terms in the sum grows exponentially. What \citet{lundbergConsistentIndividualizedFeature2019} did was to develop an efficient algorithm that could solve this for trees in polynomial time\sidenote{Specifically they managed to improve the time complexity from $\mathcal{O}(TL2^M)$ to $\mathcal{O}(TLD^2)$ where $T$ is the number of trees, $L$ is the maximum number of leaves in any tree, $M$ is te number of features, and $D$ is the maximum depth of any tree (where $D\approx \log L$ for balanced trees).}. 

SHAP values allows one to explain for a single prediction why it got the prediction that it got. When applied to the entire data set $\vec{X} \in \mathbb{R}^{N \times M}$ with $N$ observations each of with $M$ features, one gets the matrix $\bm{\Phi}$. When summing over the absolute value of each column, one gets the global impact $\phi_i^\mathrm{tot}$ of the $i^{\mathrm{th}}$ feature \citep{lundbergConsistentIndividualizedFeature2019}:
\begin{equation}
  \phi_i^\mathrm{tot} = \sum_{j=1}^N \abs{ \bm{\Phi}_{i, j} }.
\end{equation}
The global feature importance $\phi_i^\mathrm{tot}$ is thus a measure of the overall importance of feature $i$. 

Note that if one introduces a new feature to the dataset, correlated\sidenote{Not necessarily linearly correlated.} to an already existing feature, the feature importance of the previous feature will decrease. 
This is due to the axiom of symmetry:
\begin{axiom}[Symmetry]
  \label{axiom:ml:shapley_symmetry}
  If for all subsets S that do not contain $i$ or $j$: 
  \begin{equation}
    f_x(S \cup \{i\}) = f_x(S \cup \{j\})
  \end{equation}
  then $\phi_i = \phi_j$.
\end{axiom}
It can be shown that axiom \ref{axiom:ml:shapley_symmetry} is implied by axiom \ref{axiom:ml:shap_consistency}  \citep[Supp. Material]{Lundberg:2017}. Two identical features, as seen by the model, will thus \q{share} the feature importance.













\chapter{Danish Housing Prices}
\label{ch:housing_price_analysis}
\epigraph{\textit{``Buy land, they’re not making it anymore.''}}{--- Mark Twain}
% \epigraph{\textit{``It’s tangible, it’s solid, it’s beautiful. It’s artistic, from my standpoint, and I just love real estate.''}}{--- Donald Trump}
\newthought{Housing markets} have always been a playing field for economists, property speculators, real estate agents, and realtors. Personally, I am by no metric any of these, not even close to. Yet, the issue of estimating housing prices purely on a data driven basis was too interesting to ignore. Estimating housing prices is a classical economical discipline as seen in papers from the Danish National Bank developing a regional model of the Danish housing market \autocite{hviidWorkingPaperRegional2017} to an analysis of the financial crisis in \num{2008}-\num{2009} and its effects on the Copenhagen metropolitan area \autocite{mulalicFinancialCrisisDiverging2017}. 

\begin{figure*}
  \includegraphics[width=0.98\textwidth, trim=10 15 10 10, clip]{figures/housing_price_index_dst/housingindex_wide.pdf}
  \caption[Price Index of the Danish Housing Market]
          {Prince index of Danish one-family houses and owner-occupied apartments where \textcolor{red}{houses} are shown in red and \textcolor{blue}{apartments} in blue, where full lines are for the entirety of Denmark and dashed lines are only for Copenhagen. Errorbars (scaled up with a factor of 2) are shown as colored bands. The price index and its uncertainty is based on numbers from \citet{dstPriceIndexEJ14}, however, rescaled to 100 in 2000 (instead of 2006 as it was in the data).}
  \label{fig:h:price_index}
\end{figure*}

If one takes a look at the time development of the Danish housing market, the Danish governmental organization for statistics, Statistics Denmark, releases a price index \citep{dstPriceIndexEJ14} for both one-family houses (OFH) and owner-occupied apartments (OOA) quarterly, see Figure~\ref{fig:h:price_index}. Here it is easy to see the effect of the financial crisis around 2008, but also the steady increase in the housing market in both Copenhagen and the entire country since then. Housing in this context means both actual houses and privately owned apartments, and will be called residences in general in this project. 

The goal of this subproject is not to predict any future collapse in the financial markets as we saw upwards of 10 years ago. Instead, it is to learn patterns in the price of houses in steady times. The goal is training a computer to automatically\sidenote{In contrary to \citet{hviidWorkingPaperRegional2017, mulalicFinancialCrisisDiverging2017} and others who base their models on macro-economic principles.} be able to find these patterns and see if we can improve this model.

This chapter is organized as follows. The data will be introduced in  \autoref{sec:h:data_cleaning} where also some initial plots are shown. The data set is feature augmented in \autoref{sec:h:feature_augmentation} and the different evaluation functions will be discussed in \autoref{sec:h:evaluation_function}. The choice of evaluation function will be based on an initial hyperparameter optimization process in \autoref{sec:h:initial_hyperparameter_optimization}. The model is fitted and optimized in \autoref{sec:h:hyperparamater_optimization} and the results from the final model are presented in \autoref{sec:h:results}. The forecasting performance of the model will be tested in \autoref{sec:h:forecasting}. To better understand the model, \autoref{sec:h:model_inspection} deals with the issue of model inspection. The model is extended by combining multiple ML models into an ensemble, a process that is described in \autoref{sec:h:multiple_models}. Finally, the project concerning Danish housing prices will be discussed in \autoref{sec:h:discussion} and concluded in \autoref{sec:h:conclusion}.

\section{Data Preparation and Exploratory Data Analysis}
\label{sec:h:data_cleaning}
\epigraph{\textit{``\SI{80}{\percent} of data science is cleaning the data and \SI{20}{\percent} is complaining about cleaning the data.''}}{--- Anthony Goldbloom, Kaggle}


The first part of any data science project is actually getting the data and being able to read it. This has been an iterative process that has improved over time. The last data transfer we got was September \nth{3} 2019 which consisted of a \SI{522.4}{\mega\byte} CSV file with dimensions $(\num{711212}, \num{171})$. This section will go through the data cleaning process.

Before any further data analysis is performed, all of the data is loaded, except columns which only contain internal information for Boligsiden\sidenote[][-1cm]{The variables \code{Sag_Kvhx}, \code{Bygning_GOP_Matrikelnr}, and \code{Enhed_GOP_BoligtypeKode}.}. To get a better understanding of all the original input variables, histograms of the one-dimensional distributions of each variable are shown in Figures\sidenote[][-5mm]{Where the A indicates that the figures are in the the \hyperref[appendix:housing]{appendix}.}~\ref{fig:h:variable_overview_all_1} to \ref{fig:h:variable_overview_all_14}.

Four particularly interesting distributions are shown in Figure~\ref{fig:h:variable_overview}: the date of the sale \code{SalgsDato} in subfigure~\subref{fig:h:variable_overview_date}, the type of residence \code{SagtypeNr} in subfigure~\subref{fig:h:variable_overview_type},
the longitude of the residence \code{GisX_WGS84} in subfigure~\subref{fig:h:variable_overview_longitude}, and the area of the residence \code{ArealBolig} in subfigure~\subref{fig:h:variable_overview_area}.

\begin{margintable}[0.5cm]
  \centering
  \begin{tabular}{@{}rl@{}}
  % \toprule
  Code & Name           \\ 
  \midrule
  100  & Villa          \\ 
  200  & Rækkehus       \\
  300  & Ejerlejlighed  \\
  400  & Fritidsbolig   \\
  401  & Kolonihave     \\
  500  & Andelsbolig    \\
  600  & Landejendom    \\
  700  & Helårsgrund    \\
  800  & Fritidsgrund   \\
  900  & Villalejlighed \\
  1000 & Kvæggård       \\
  1100 & Svinegård      \\
  1200 & Planteavlsgård \\
  1300 & Skovejendom    \\
  1400 & Lystejendom    \\
  1500 & Specialejendom \\ 
  % \bottomrule
  \end{tabular}
  \vspace{2mm}
  \caption[Mapping between the Code in \code{SagTypeNr} and the Type of Residence]{Mapping between the code in \code{SagTypeNr} and the type of residence. The two important types of residences are \q{villa} (one-family houses) and \q{ejerlejlighed} (owner-occupied apartments).}
  \label{tab:h:salgstype}
\end{margintable}

The distribution of the date of sale, Figure~\ref{fig:h:variable_overview} \subref{fig:h:variable_overview_date}, is an interesting variable because it shows how Boligsiden has been collecting more and more data over time. Here \num{2007} and \num{2019} are clear outliers since their current database only contains sales from the end of \num{2007}, and \num{2019} only contains data from the first eight months of the year. The \code{SagTypeNr} is a discrete code that Boligsiden uses to differentiate between different types of residences. The mapping between code and description is shown in Table~\ref{tab:h:salgstype}. 

\begin{figure*}
  \centering
  \subfloat[\label{fig:h:variable_overview_date}]{\qquad}
  \includegraphics[width=0.45\textwidth, page=2, trim=15 15 15 15, clip]{figures/housing/overview_fig.pdf}\hfil
  \subfloat[\label{fig:h:variable_overview_type}]{\qquad}
  \includegraphics[width=0.45\textwidth, page=6, trim=15 15 15 15, clip]{figures/housing/overview_fig.pdf}
  \subfloat[\label{fig:h:variable_overview_longitude}]{\qquad}
  \includegraphics[width=0.45\textwidth, page=20, trim=15 15 15 15, clip]{figures/housing/overview_fig.pdf}\hfil
  \subfloat[\label{fig:h:variable_overview_area}]{\qquad}
  \includegraphics[width=0.45\textwidth, page=23, trim=15 15 15 15, clip]{figures/housing/overview_fig.pdf}
  \caption[Distributions of the Variables in the Housing Prices]{Distributions of four out of the \num{168} input variables. 
           Subplot ~\protect\subref{fig:h:variable_overview_date} shows the date of the sale, 
           Subplot ~\protect\subref{fig:h:variable_overview_type} shows the type of residence,
           Subplot ~\protect\subref{fig:h:variable_overview_longitude} shows the longitude,
           Subplot ~\protect\subref{fig:h:variable_overview_area} shows the area of the house.}
  \label{fig:h:variable_overview}
\end{figure*}

% \vspace{-2cm}
In this project only one-family houses (\q{Villas}) with code \num{100} and owner-occupied apartments (\q{Ejerlejlighed}) with code \num{300} are considered. In Figure~\ref{fig:h:variable_overview} \subref{fig:h:variable_overview_type} it is shown that these two types of residences are also the most frequent sales with close to \num{400000} and \num{150000} sales in total. The longitude distribution, Figure~\ref{fig:h:variable_overview} \subref{fig:h:variable_overview_longitude}, is mostly interesting due to fact that it clearly shows how the Great Belt and especially the Baltic Sea separates Denmark into three parts; the Western part, the Eastern part, and then Bornholm. Note that more than \SI{5}{\percent} of the residences' locations are unknown values, so-called \q{Not A Number}s (NANs). The distribution of the area, Figure~\ref{fig:h:variable_overview} \subref{fig:h:variable_overview_area}, shows that most residences are between \SI{50}{\meter^2} and \SI{200}{\meter^2}, as expected in Denmark. However, a relatively large part of the residences, \SI{2.5}{\percent}, are listed as having an area of \SI{0}{\meter^2} which are obviously erroneous entries.

\begin{figure*}
  \centering
  \subfloat[\label{fig:h:geo_overview_sqm_price}]{\,}
  \includegraphics[draft=false, width=0.4\textwidth]{figures/housing/Denmark_Overview_SqmPrice.png}\hfil
  \subfloat[\label{fig:h:geo_overview_sales_price}]{\,}
  \includegraphics[draft=false, width=0.4\textwidth]{figures/housing/Denmark_Overview_SalesPrice.png}
  \caption[Geographic Distribution of the Sold Residences]{Geographic distribution of the sold residences. 
           In subplot ~\protect\subref{fig:h:geo_overview_sqm_price} the sales are colored according to their square meter price and in subplot ~\protect\subref{fig:h:geo_overview_sales_price} according to the sales price. 
           }
  \label{fig:h:geo_overview}
\end{figure*}

The geographic distribution of sales are shown in Figure~\ref{fig:h:geo_overview}. The residences are coloured according to the square meter price in Figure~\ref{fig:h:geo_overview} \subref{fig:h:geo_overview_sqm_price} and according to the sales price in Figure~\ref{fig:h:geo_overview} \subref{fig:h:geo_overview_sales_price}. Notice the strong correlation between the distance to water and the square meter price, a correlation that is less visible when looking at the sales price. Since these plots each contain \num{674647} points\sidenote{Only sales with a valid GPS-coordinate and area of residence are shown}, over-plotting quickly becomes an issue. To circumvent this, the software package called DataShader \autocite{bednarDatashaderRevealingStructure2019} was used which in a simple, consistent, and not at least computationally efficient manner allows one to plot big data.



The most important of the features is the sales price, called \code{SalgsPris} in the dataset. Its distribution is shown in Figure~\ref{fig:h:price_overview_price}. This is a positively skewed distribution that shares visual similarities with a log-normal distribution. The mode of the all sales prices\sidenote{Measured in millions DKK, \si{\Mkr}} is \SI{1.1}{\Mkr} and the median is \SI{1.6}{\Mkr} The mean is \SI{2.0}{\Mkr} but this value is heavily influenced by a few very high values. 

\begin{figure}
  \includegraphics[width=0.98\textwidth, page=1, trim=15 15 15 15, clip]{figures/housing/overview_fig.pdf}
  \caption[Histogram of Prices of Houses and Apartments Sold in Denmark]
          {Histogram of prices of houses and apartments sold in Denmark.}
  \label{fig:h:price_overview_price}
\end{figure}

\vspace{-0.5cm}
\subsection{Correlations}
\label{subsec:h:correlations_lin_mic}

Having shown the \num{1}D-distributions of all the different variables in the previous section, the next step would be to look at the correlations between the variables. Since there are \num{168} input variables, it is almost impossible to understand every inter-variable correlation, however, it is tried in Figure~\ref{fig:h:correlations_all_lin}. In this figure, the correlations $\rho$ between all numerical variables that are not obviously related to other variables (like the GPS-coordinates that are in both latitude-longitude and ETRS89\sidenote[][-5mm]{European Terrestrial Reference System \num{1989}.} format), are plotted as a $(86 \times 86)$-dimensional heatmap\sidenote{With the condition that each plotted variable has to have at least one inter-variable correlation higher than $|\rho| > \SI{30}{\percent}$}.

Even though inter-variable correlations are important in the exploratory data analysis (EDA) phase, what is more important is to get a better understanding of how the input variables correlate with the output variable; the sales price. This is shown in Figure~\ref{fig:h:corr_lin} for the variables where $\abs{\rho} > \SI{10}{\percent}$. It is the previous property evaluation, \code{EjdVurdering_EjendomsVaerdi} that is correlated the most with the sales price, which does not come as any huge surprise. Other positively correlated variables are the cost of ownership\sidenote{This a great example of the fact that correlation does not imply causation as simply increasing the cost of ownership does not increase the sales price.}, area, number of bathrooms, longitude, and distance to nearest wind mill. In the other end, the local income tax, \code{Kommune_SkatteProcent}, is the variable that is the most negatively correlated to the sales price, followed by the geographical variables related to province, municipality, and postcode.

\begin{figure*}
  \centerfloat
  \includegraphics[width=0.99\textwidth, trim=0 10 0 40, clip]{figures/housing/lin_correlation.pdf}
  \caption[Linear Correlation Between Variables and Price]
          {Linear correlation $\rho$ between each variable and the sales price for variables where $\abs{\rho} > \SI{10}{\percent}$.}
  \label{fig:h:corr_lin}
\end{figure*}

The correlation $\rho$ used above is the linear correlation which only captures linear relationships between variables. All modern machine learning algorithms, however, are also able to capture higher-order correlations and thus a higher-order correlation measure is needed. The maximal information coefficient (MIC), which is a value between \num{0} and \num{1}, is such a nonlinear measure\sidenote{Note, that contrary to the linear correlation, MIC is always positive.}. It finds correlation based on the intuition that if two variables are correlated, it should be possible to split the data up into smaller grids where, if they are correlated, the grid that contain points should contain many points and the rest of the grids should be (relatively) empty \autocite{reshefDetectingNovelAssociations2011}. This is in comparison to two uncorrelated variables which would simply display noisy behavior and only have few grids with many points in. \citet{albanesePracticalToolMaximal2018a} extended on this idea and developed the computationally efficient algorithm called MICtools which computes the estimator $\mathrm{MIC}_e$ for MIC. An example of this nonlinear correlation is seen in Figure~\ref{fig:h:MIC_example_small}. Here the relationship between the normal linear correlation $\rho$ and $\mathrm{MIC}_e$ are shown for four synthetic datasets. Notice that $\mathrm{MIC}_e$ performs particularly well for the sine wave, and decent for the line and parabola, but only slightly captures the relationship for the exponential growth. The influence of noise on $\rho$ and $\mathrm{MIC}_e$ is shown in Figure~\ref{fig:h:MIC_example}. 

\begin{figure}
  \centerfloat
  \includegraphics[width=0.99\textwidth, trim=10 10 10 10, clip]{figures/housing/MIC_test_small.pdf}
  \caption[Comparison of the Linear Correlation $\rho$ and the Nonlinear MIC]
          {Comparison of the linear correlation $\rho$ and the nonlinear MIC for a straight line, a parabola, an exponential, and a sine wave, all with noise added. See also Figure~\ref{fig:h:MIC_example}.}
  \label{fig:h:MIC_example_small}
\end{figure}

Using $\mathrm{MIC}_e$ as the correlation measure between the numerical variables and the sales prices, the variables with a $\mathrm{MIC}_e$-score higher than \SI{10}{\percent} are shown in Figure~\ref{fig:h:corr_MIC}. Again, the previous property evaluations are the most correlated features to the sales price followed by the parish code, \code{SogneKode}. In general the geographical variables score high here with also the post code and municipality number. 

\begin{figure*}
  \centerfloat
  \includegraphics[width=0.99\textwidth, trim=0 10 0 40, clip]{figures/housing/MIC_plot.pdf}
  \caption[Nonlinear Correlation Between Variables and Price]
          {Nonlinear correlation between variables and price using Maximal Information Coefficient (MIC) for variables where $\text{MIC}>10\%$.}
  \label{fig:h:corr_MIC}
\end{figure*}

\subsection{Validity of Input Variables}

The fact that some of the variables contain considerable amounts of invalid values, NANs, requires this to be taken into account before any further analysis. The validity, defined as the percentage of valid observations, of every variable is shown in Figure~\ref{fig:h:nans}. Here the \num{168} variables are grouped together into \num{25} groups where each group share the same validity. An example of this is all of the \num{16} different distance-variables\sidenote{Distance to: prison, heath, high-voltage transmission line, industry, visible railroad, church, churchyard, coast, landing strip, motorway, access to motorway, recreational area, border, sports centre, beach, and windmill.}. We see that most of the variables have validities around more than 
\SI{85}{\percent}, however, a few of the variables, especially information about the building, \code{BygningsInfo}, have validities less than \SI{20}{\percent}. 

% \sidenote{Distance to: Fængsel, Hede, Højspaendingsledning, Industri, JernbaneSynlig, Kirke, Kirkegård, Kyst, Landingsbane, Motorvej, MotorvejTilFraKørsel, RekreativtOmråde, Rigsgrænse, Sportsanlæg, Strand, Vindmølle}

\begin{figure*}
  \centerfloat
  \includegraphics[width=\textwidth, trim=0 0 0 0, clip]{figures/housing/missing_bar.pdf}
  \caption[Validity of Input Features]
          {Percentage of valid counts for each variable grouped together in categories.}
  \label{fig:h:nans}
\end{figure*}

To see how closely related the different validity groups are, one can look at the dendrogram in Figure~\ref{fig:h:nans_dendrogram}. The dendrogram is based on a hierarchical clustering algorithm \citep{virtanenSciPyFundamentalAlgorithms2019} where the different groups are clustered according to the linear correlation of their validity. This diagram is supposed to be read in a top-down approach, where it can be seen that the name of the street, \code{Vejnavn}, and the number of the residence, \code{HusNr}, correlate a lot and are thus clustered very early. The year of the last time the residence was rebuilt or greatly modified, \code{OmbygningsAAr}, does not correlate strongly with any of the other variables and is thus the last variable to be clustered together. A heatmap of the validity inter-variable correlations is shown in Figure~\ref{fig:h:nans_heatmap}. 

\begin{figure}
  \includegraphics[width=0.98\textwidth, trim=35 10 0 10, clip]{figures/housing/missing_dendrogram.pdf}
  \caption[Validity Dendrogram]
          {Validity Dendrogram based on hierarchical clustering of the linear correlation of validity for the housing price variables clustered together.}
  \label{fig:h:nans_dendrogram}
\end{figure}


\subsection{Selection Criteria}
Given the 1D input variable distributions and their validity, we apply some very basic selection criteria before any further analysis. These selection criteria are shown in Table \ref{tab:h:initial_cuts}. 

\begin{table}[h]
  \begin{tabular}{@{}llcr@{}}
  % \toprule
               & Description                                          & Remaining & Removed \\ 
  \midrule
  Area         & \SI{20}{\meter\squared} $\leq$ Area $\leq$ \SI{500}{\meter\squared} & \num{689140}              & \num{23666}     \\
  Price        & \SI{0.1}{\Mkr} $\leq$ Price $\leq$ \SI{100}{\Mkr}                   & \num{687546}              & \num[group-minimum-digits=3]{1594}      \\
  Type         & Has sales type \num{1}                                              & \num{605415}              & \num{82131}     \\
  GPS          & Has valid GPS coordinates                                           & \num{578860}              & \num{26555}     \\
  Private      & Only non-business sales                                             & \num{549140}              & \num{29720}     \\
  Time         & Sold in \num{2009} or later                                         & \num{520548}              & \num{28592}     \\
  \bottomrule
  \end{tabular}
  \caption[Basic Selection Criteria]{Overview of the basic selection criteria which define the minimum information needed to predict the price of a sale.}
  \label{tab:h:initial_cuts}
\end{table}

The sales type, \code{OISSalgsType}, is an OIS\sidenote{OIS is short for \q{Den Offentlige Informationsserver}, the Danish public information server, and it collects information about Danish residences \autocite{oisWwwOISDk}.} code which describes what type of sale it is: when it is \num{1} it is considered a normal sale, compared to e.g. forced sales. These cuts are seen as the minimum requirements for what constitutes a curated dataset with no obvious outliers. The reason why the time requirement is applied is to reduce the effect of the financial crisis to creep into the model. 


\section{Feature Augmentation}
\label{sec:h:feature_augmentation}

The analysis have dealt with different types of residences all together until now. The rest of the analysis will be applied on single-family houses and owner-occupied apartments independently from this point on. 

\begin{margintable}[3cm]
  % \begin{tabular}{llr}
  \begin{tabular*}{\textwidth}{l @{\extracolsep{\fill}} lr}
  % \toprule
  String & Explanation  & Code \\ \midrule
  NAN    & No side door & \num{0}    \\
  TH, TV & Right, left  & \num{11}   \\
  MF     & Center       & \num{12}   \\
  --     & The rest     & \num{15} 
  \end{tabular*}
  \vspace{1mm}
  \caption[Side Door Mapping.]{Side door mapping. If the side door string contains e.g. \q{TH} this gets the code \num{11}.}
  \label{tab:h:sidedoor_code}
  \vspace{3mm}
\end{margintable}

First, invalid counts are dropped such that variables which contain more than \SI{10}{\percent} NANs are dropped, and duplicate rows are also removed. Then some manual features are added based on existing features. The day of the month, the month, and the year are extracted from the sales date. The sales date is converted to the numbers of days since January \nth{1}, \num{2009}. From the number of the house, \code{HusNr}, the value is extracted along with a boolean flag indicating whether or not it includes a letter (eg. \q{27B}). The number of the side door, \code{SidedoerNr}, is formatted according to Table~\ref{tab:h:sidedoor_code} and the road name according to  Table~\ref{tab:h:road_code}. The age of the house is added\sidenote[][-1cm]{In addition to only having the year the house was was built.} and the amount of time (in years) since last major modification. The energy rating label, \code{EnergiMaerke}, is also converted from strings to values according to Table~\ref{tab:h:energy_code}.

\begin{margintable}
  \begin{tabular*}{\textwidth}{l @{\extracolsep{\fill}} lr}
  Contains   & Explanation  & Code \\ \midrule
  Vej        & Road       & \num{0}    \\
  Gade       & Street     & \num{1}    \\
  Alle, Allé & Avenue     & \num{2}    \\
  Boulevard  & Boulevard  & \num{3}    \\
  --         & The rest   & \num{-1}  
  \end{tabular*}
  \vspace{1mm}
  \caption[Street Mapping]{Street mapping. If the street name contains e.g. \q{Vej} this gets the code \num{0}.}
  \label{tab:h:road_code}
  \vspace{3mm}
\end{margintable}

Finally, some of the variables in the dataset are not suitable for machine learning (or are simply transformations of other variables) and are thus dropped. These are variables such as the ID, when the house was deleted at Boligsiden, or the cash price.

\subsection{Time-Dependent Price Index}

In addition to the manual data augmentation, a time-dependent price index is also added. We make use of the open source package called Prophet made by \citet{taylorForecastingScale} at Facebook. It is based on a decomposable time series model \autocite{harveyEstimationProceduresStructural1990} with two\sidenote{In their paper, \citet{taylorForecastingScale} include a holiday component in their analysis as well which is not included in this project.} components; the trend $g(t)$ and the seasonality $s(t)$:
\begin{equation}
  y_\mathrm{Prophet}(t) = g(t) + s(t) + \epsilon_t,
\end{equation}
where $\epsilon_t$ is a normally distributed error term. \citet{taylorForecastingScale} fit this equation with a generalized additive model (GAM) \autocite{hastieGeneralizedAdditiveModels1987} which they argue has several practical advantages compared to ARIMA\sidenote{AutoRegressive Integrated Moving Average.} models which are commonly used in economics \autocite{nla.cat-vn1067782}. 

We fit the Prophet model on the weekly median price pr. square meter (PPSM) up until (and including) \num{2017}. The results of the prophet model fitted on (owner-occupied) apartments are seen in Figure~\ref{fig:h:prophet_forecast} and Figure~\ref{fig:h:prophet_trends}. In Figure~\ref{fig:h:prophet_forecast} the weekly median price pr. square meter for apartments are shown as black dots with the fitted Prophet model shown in blue. The $1\sigma$ uncertainty intervals are shown as the transparent blue band. The Prophet model not only allows predicting previous and future PPSMs, it also return the uncertainty of this prediction. The future predictions for the PPSM are the values after 2018. The trend and seasonality of the model are shown in Figure~\ref{fig:h:prophet_trends} where the top plot is the overall trend $g(t)$ and the bottom plot is the seasonality $s(t)$. Whereas the trend just continues to rise, the seasonality shows that residences are generally sold for a higher price in the Summer months compared to the Winter months. 
The Prophet model plots for one-family houses are shown in Figure~\ref{fig:h:prophet_forecast_villa} and \ref{fig:h:prophet_trends_villa}. 

\begin{figure}
  \includegraphics[draft=false, width=0.9\textwidth, trim=15 15 15 15, clip]{figures/housing/Ejerlejlighed_v18_cut_all_Ncols_all_prophet_forecast.png}
  \caption[Prophet Forecast for Apartments]
          {The predictions of the Facebook Prophet model trained on square meter prices for owner-occupied apartments sold before January 1st, 2018. The data is down-sampled to weekly bins where the median of each week is used as in input to the Prophet model, shown seen as black dots in the figure. The \textcolor{blue}{model's forecasts} for 2018 and 2019 are shown in blue with a light blue error band showing the $1\sigma$ confidence interval.
          }
  \label{fig:h:prophet_forecast}
\end{figure}

\begin{figure}
  \includegraphics[draft=false, width=0.9\textwidth, trim=15 15 15 15, clip]{figures/housing/Ejerlejlighed_v18_cut_all_Ncols_all_prophet_trends.pdf}
  \caption[Prophet Trends]
          {The trends of the Facebook Prophet model trained on square meter prices for owner-occupied apartments sold before January 1st, 2018. In the top plot is the overall trend as a function of year and in the bottom plot is the yearly variation as a function of day of year. It can be seen that the square meter price is higher during the Summer months compared to the Winter months, however, compared to the overall trend this effect is minor ($<10\%$). 
          }
  \label{fig:h:prophet_trends}
\end{figure}

Using the Prophet model, we define the price index (PI) to be the Prophet-predicted PPSM, $y_\mathrm{Prophet}$, for each residence normalized by the mean to give values around \num{1}:
\begin{equation}
  % \mathrm{PI}(t) = \frac{y_\mathrm{Prophet}(t)}{\mathrm{mean}(y_\mathrm{Prophet}(t))}.
  \mathrm{PI}(t) = \frac{y_\mathrm{Prophet}(t)}{\langle y_\mathrm{Prophet}(t) \rangle},
\end{equation}
where $\langle \boldsymbol{\cdot} \rangle$ refers to the average. The price index thus works as a measure of the national price for houses or apartments at a given time and is added as a variable to the dataset. 

% \begin{figure}
%   \includegraphics[width=0.9\textwidth]{figures/housing/tinglysning_fig.pdf}
%   \caption[Registration of property]
%           {Overview of registration of property as a function of amount of owners (\code{AntEjere}), lowest share (\code{MindsteAndel}) and biggest share (\code{StoersteAndel}) written as \code{[AntEjere, MindsteAndel, StoersteAndel]}.
%           }
%   \label{fig:h:tinglysning}
% \end{figure}

\section{Evaluation Function}
\label{sec:h:evaluation_function}

The choice of evaluation function $f_\mathrm{eval}$ is an important decision. The evaluation function will be based on the relative prediction $z_i$: 
\begin{equation}
  z_i = \frac{\hat{y}_i-y_i}{y_i},
\end{equation}
where $y_i$ is the true price and $\hat{y}_i$ the predicted one. The relative prediction is defined such that it is positive when $\hat{y}_i>y_i$. Due to the outlier selection criteria applied earlier, the denominator is made sure to always be positive (and never \num{0}), and $\vec{z}$ is expected to approximately follow a normal distribution\sidenote{Where $\vec{z}$ is the vector of all relative predictions $\vec{z} \in \mathbb{R}^N$.}. Initially the mean of $\vec{z}$ was considered as the choice of evaluation function $f_\mathrm{eval}\stackrel{?}{=} \mathrm{mean}(\vec{z})$ though this only ensures a minimum of bias, not necessarily a low spread. This lead the discussion on to look at the standard deviation of $z$ as $f_\mathrm{eval}\stackrel{?}{=} \mathrm{std}(\vec{z})$. The mean and the standard deviation, however, are not very robust estimators since they are heavily influenced by outliers. The mean (and thus also the standard deviation) has an \emph{asymptotic breakdown point} at \SI{0}{\percent}, where the breakdown point is defined as the smallest fraction\sidenote{Where the \q{asymptotic} in \q{asymptotic breakdown point} refers to when the number of samples goes to infinity.} of bad observations that can cause an estimator to become arbitrarily small or large: a single outlier with an arbitrarily large value may cause the mean to diverge to that large value \autocite{huber2011robust}. In comparison, the median has an asymptotic breakdown point of \SI{50}{\percent} and is thus a more robust estimator of centrality. A robust measure of the variability or dispersion of a sample $\vec{x}$ (compared to e.g. the standard deviation $\sigma$) is the median absolute deviation (MAD) written as:
\marginnote{Here $\abs{\vec{x}}$ is the absolute value of $x_i$ applied element-wise and not the norm of the vector $\vec{x}$.}
\begin{equation}
  \mathrm{MAD}(\vec{x}) = c \cdot \mathrm{median} \left( \abs{\vec{x} - \mathrm{median}(\vec{x})} \right), \quad c = \frac{1}{\Phi^{-1}(\frac{3}{4})},
  \label{eq:h:MAD}
\end{equation}
where $c$ is a normalization constant to make MAD a consistent estimator of the standard deviation $\sigma$ assuming normally distributed data and $\Phi^{-1}$ is the percent point function\sidenote{Inverse of the cumulative distribution function.} \autocite{leysDetectingOutliersNot2013}.
The MAD is thus the median of the absolute differences between the data and the median of the data. We are, however, not just interested in having the distribution of $\vec{z}$ as narrow as possible, we also want it centered around \num{0}. We thus continue with the following evaluation function:
\begin{equation}
  \begin{split}
    \mathrm{MAD}_0(\vec{x})  &\equiv c \cdot  \mathrm{median} \left( \abs{\vec{x} - 0} \right) = c \cdot \mathrm{median} \left( \abs{\vec{x}} \right) \\
    f_\mathrm{eval}(\vec{z}) &\equiv \mathrm{MAD}_0(\vec{z}) = c \cdot  \mathrm{median} \left( \abs{\vec{z}} \right).
  \end{split}
\end{equation}
To get an intuition about the size of a \q{good} value of $\mathrm{MAD}_0$, one could calculate it comparing the asking price with the actual sales price. Doing so, one finds: $f_\mathrm{eval}(\vec{z}_\mathrm{OFH}) = \SI{11.35}{\percent}$ for houses and $f_\mathrm{eval}(\vec{z}_\mathrm{OOA}) = \SI{5.72}{\percent}$ for apartments. In some cases $f_\mathrm{eval}$ will still be referred to as MAD and it will be mentioned explicably if the form in equation \eqref{eq:h:MAD} is meant. 

\marginnote[-4cm]{The robust estimator MAD is derived assuming symmetric distributions. Robust measures of the variability of non-symmetric samples have also been developed, see \citet{rousseeuwAlternativesMedianAbsolute1993} for more details.}


\section{Initial Hyperparameter Optimization}
\label{sec:h:initial_hyperparameter_optimization}

With the initial cleaning and feature adding done, the shapes of the ML-ready datasets are: (\num{291317}, \num{144}) for houses (OFHs) and (\num{114166}, \num{144}) for apartments (OOA), both sharing the same variables. All of the variables which are used from this point on are shown in Table~\ref{tab:h:all_variables}. 
The data are split into training and test sets such that training is defined as every sale from before \num{2018}, the test set is every sale from \num{2018}, and, since more data came after the project started, \num{2019} is a small extra test set. 

\begin{margintable}[-8.5cm]
  \begin{tabular}{lrr}
              & Houses       & Apartments   \\ \midrule
   Train      & \num{240070} & \num{93115}  \\   
   Test       & \num{34628}  & \num{14183}  \\   
   \num{2019} & \num{16619}  & \num{6868} 
  \end{tabular}
  \vspace{1.5mm}
  \caption[Number of Observations in the Housing Dataset]{\label{tab:h:train_test_split}Number of observations for houses and apartments in the training, test, and \num{2019} set.}
  \vspace{3mm}
\end{margintable}

\begin{margintable}[-3cm]
  \begin{tabular}{lrr}
% \begin{table}
  % \begin{tabular}{lll}
   Tight      & Houses        & Apartments  \\ \midrule
   Train      & \num{143179}  & \num{57795} \\   
   Test       & \num{20338}   & \num{8376}  \\   
   \num{2019} & \num{9683}    & \num{4030} 
  \end{tabular}
  \vspace{1.5mm}
  \caption[Number of Observations in the Housing Dataset for the Tight Selection]{\label{tab:h:train_test_split_tight}Number of observations for houses and apartments in the training, test, and \num{2019} set for the tight selection.}
  \vspace{3mm}
\end{margintable}

The number of observations for the different sets are shown in Table~\ref{tab:h:train_test_split}. Since the dataset has been shown to be quite noisy with a lot of invalid counts a \emph{tight} selection of the data is also applied. The tight selection is defined as residences which are within the \SI{1}{\percent} to \SI{99}{\percent} quantiles of all\sidenote[][-1.5cm]{Except the variables that contains the words: \q{aar} (year), \q{dato} (date), and  \q{prophet}.} numeric variables with more than 3 unique values. The number of observations for the different tight sets are shown in Table~\ref{tab:h:train_test_split_tight}. 

A small study into the effect of some various hyperparameters was performed before any further fitting. This study investigated the effect of the old sales by assigning them a lower weight depending on time. It was investigated whether or not the model would perform better if samples got the time-dependent weight $w(t)$ given by:

\begin{marginfigure}[-0.5cm]
  \includegraphics[width=0.99\textwidth]{figures/housing/Villa_v18_cut_all_Ncols_all_half_life_weights.pdf}
  \caption[Sample Weight as a Function of Time for Different Half-Lives.]
    {The sample weight $w(t)$ as a function of time $t$ where the time is in years after January \nth{1}, 2009, Here seen plotted for different values of the half-life $T_{\frac{1}{2}}$.}
  \label{fig:h:half-life}
\end{marginfigure}

\begin{equation}
  \begin{split}
    w'(t) &= e^{ k \cdot t}, \quad k = \frac{\log 2}{T_{\frac{1}{2}}} \, ,\\
    w(t) &= \frac{w'(t)}{\langle w'(t) \rangle} \, ,
  \end{split}
  \label{eg:h:sample_weight}
\end{equation}
% TODO: add parts about what weights are.

\noindent where $T_{\frac{1}{2}}$ is the half-life. This is illustrated in Figure~\ref{fig:h:half-life} for the different values of $T_{\frac{1}{2}}$ used in the study.  In addition to the weight, it was also investigated whether or not a $\log_{10}$-transformation of the price would increase performance. Finally the choice of loss function was also added to the study for the five different loss functions defined in \autoref{sec:ml:loss_function}. A grid search\sidenote[][-2cm]{Grid search was acceptable since the parameter space is small and two of the three dimensions are non-numerical.} was performed for:
\begin{equation}
  \begin{split}
    T_{\frac{1}{2}} &\in \{2.5,\, 5,\, 10,\, 20,\, \infty \}~\mathrm{ years} \\
    \log_{10} &\in \{\mathrm{True},\, \mathrm{False} \} \\
    \ell &\in \{ \ell_\mathrm{Cauchy},\, \ell_\mathrm{Fair},\, \ell_\mathrm{LogCosh},\, \ell_\mathrm{SE},\, \ell_\mathrm{Welsch}\}.
  \end{split}
\end{equation}

\begin{margintable}[-1cm]
  \begin{tabular}{@{}ccrc@{}}
    %\toprule
    Half-life & $\log_{10}$ & $N_\mathrm{trees}$ & $f_\mathrm{eval}$ \\
    \midrule
    \num{2.5} & True & \num{293} & \num{0.1598} \\
    \num{2.5} & False & \num{814} & \num{0.1466} \\
    \num{5} & True & \num{304} & \num{0.1610} \\
    \num{5} & False & \num{923} & \num{0.1468} \\
    \num{10} & True & \num{266} & \num{0.1610} \\
    $\mathbf{10}$ & \textbf{False} & $\mathbf{770}$ & $\mathbf{0.1450}$ \\
    \num{20} & True & \num{288} & \num{0.1613} \\
    \num{20} & False & \num{967} & \num{0.1467} \\
    $\infty$ & True & \num{340} & \num{0.1601} \\
    $\infty$ & False & \num{807} & \num{0.1480} \\
    %\bottomrule
  \end{tabular}
  \vspace{1.5mm}
  \caption[Results from the Initial Hyperparameter Optimization for Apartments]{\label{tab:h:HPO_initial_Cauchy-ejerlejlighed}Results of the initial hyperparameter optimization for apartments for the best loss function $\ell_\mathrm{Cauchy}$. The best hyperparameter is shown in bold.}
\end{margintable}


\begin{margintable}[0.5cm]
  \begin{tabular}{@{}ccrc@{}}
    %\toprule
    Half-life & $\log_{10}$ & $N_\mathrm{trees}$ & $f_\mathrm{eval}$ \\
    \midrule
    \num{2.5} & True & \num{434} & \num{0.1991} \\
    \num{2.5} & False & \num{1007} & \num{0.1872} \\
    \num{5} & True & \num{350} & \num{0.1999} \\
    \num{5} & False & \num{1130} & \num{0.1858} \\
    \num{10} & True & \num{436} & \num{0.1992} \\
    \num{10} & False & \num{1183} & \num{0.1850} \\
    \num{20} & True & \num{397} & \num{0.2003} \\
    $\mathbf{20}$ & \textbf{False} & $\mathbf{1514}$ & $\mathbf{0.1833}$ \\
    $\infty$ & True & \num{449} & \num{0.1992} \\
    $\infty$ & False & \num{1351} & \num{0.1844} \\
    %\bottomrule
  \end{tabular}
  \vspace{1.5mm}
  \caption[Results from the Initial Hyperparameter Optimization for Houses]{\label{tab:h:HPO_initial_Cauchy-villa}Results of the initial hyperparameter optimization for houses for the best loss function $\ell_\mathrm{Cauchy}$. The best hyperparameter is shown in bold.}
\end{margintable}


The grid search is performed for two boosted decision tree models based on XGBoost \autocite{chenXGBoostScalableTree2016} (XGB), one for apartments and one for houses. They were each training with \num{5}-fold cross validation and early stopping was applied with a patience of \num{100}. 

For apartments, the loss function with the lowest $f_\mathrm{eval}$ was the Cauchy loss with $T_{\frac{1}{2}}=10$ years and no $\log_{10}$-transformation. This model terminated by early stopping after \num{770} trees, see Table~\ref{tab:h:HPO_initial_Cauchy-ejerlejlighed}. 
For houses the loss function with the lowest $f_\mathrm{eval}$ was (also) the Cauchy loss with $T_{\frac{1}{2}}=20$ years and (also) no $\log_{10}$-transformation. This model terminated by early stopping after \num{1514} trees, see Table~\ref{tab:h:HPO_initial_Cauchy-villa}. 
It is interesting to note, that the best results for both houses and apartments share the same loss function and no $\log_{10}$-transformation. 

The $\log_{10}$-transformation was included as some machine learning methods assume that the dependent variable, $y$, is normally distributed. Initially, it also showed better results than no transformation, however, this turned out to be a numerical consequence which was alleviated by dividing all prices with a million, such that $y$ had units of \si{\Mkr} instead of \si{\kr} All of the results for the apartments are shown in Tables~\ref{tab:h:HPO_initial_Rmse-ejerlejlighed-appendix} to \ref{tab:h:HPO_initial_Fair-ejerlejlighed-appendix} along with all of the results for the houses in Tables~\ref{tab:h:HPO_initial_Rmse-villa-appendix} to \ref{tab:h:HPO_initial_Fair-villa-appendix}. 

\begin{figure}[h!]
  \centerfloat
  \includegraphics[width=0.99\textwidth, trim=0 0 0 0, clip]{figures/housing/Ejerlejlighed_v19_cut_all_Ncols_all_CV_viz_initial_HPO.pdf}
  \caption[Parallel Coordinate Plot of the Initial Hyperparameter Optimization for Apartments]
          {Hyperparameter optimization results for apartments. The results are shown as parallel coordinates with each hyperparameter along the $x$-axis and the value of that parameter on the $y$-axis. Each line is an iteration of the grid search, colored according to the performance of that hyperparameter as measured by the MAD. The \textcolor{red}{single best hyperparameter} is shown in red. For the hyperparameter \code{log10} \code{0} means False and \code{1} means True, for \code{Halftime} $\infty$ is mapped to \code{30}, and for \code{objective} the functions Cauchy (0), Fair (1), LogCosh (2) SquaredError (3), and Welsch (4) are mapped to the integers in the parentheses. See Figure~\ref{fig:h:initial_CV_res_parallel_coords_ejer_appendix_big} for a bigger version of this figure.} 
  \label{fig:h:initial_CV_res_parallel_coords_ejer}
\end{figure}

The results the hyperparameter optimization are shown in the parallel coordinate plot in Figure~\ref{fig:h:initial_CV_res_parallel_coords_ejer}. The hyperparameters (along the time taken and the number of trees) are plotted along the $x$-axis and the value of the hyperparameter on the $y$-axis. Every iteration of the grid search is thus a line on the plot. The lines are colored according to their evaluation score; the best hyperparameter is shown in red. For the hyperparameter \code{log10}, \code{0} means False and \code{1} means True, for \code{Halftime} $\infty$ is mapped to \code{30}, and for \code{objective} the functions Cauchy (0), Fair (1), LogCosh (2) SquaredError (3), and Welsch (4) are mapped to the integers in the parentheses. The results of the grid search for houses are shown in Figure~\ref{fig:h:initial_CV_res_parallel_coords_villa}.


What can be concluded from Figure~\ref{fig:h:initial_CV_res_parallel_coords_ejer} is that A) there is a clear preference to not $\log$-transform the data, B) the models with many trees (selected by early stopping) generally performed better than the ones with fewer trees, C) it is difficult to see a clear pattern for the halflife, and that D) there seems to be a tendency for the Cauchy loss to be the best. What is not seen in the figure, however, are how the uncertainties of the different iterations also matter, where the uncertainties are the standard deviations (not of the mean) of the \num{5} folds in the cross validation.
\begin{marginfigure}
  \centerfloat
  \includegraphics[width=0.95\textwidth, trim=0 0 0 0, clip]{figures/housing/Ejerlejlighed_v19_cut_all_Ncols_all_MAD_gridsearch_half.pdf}
  \caption[Initial HPO Results for the Weight Half-life $T_{\frac{1}{2}}$]
          {Evaluation score as a function of the weight half-life $T_{\frac{1}{2}}$ with the standard deviation over the \num{5} folds as errorbars for apartments.}
  \label{fig:h:hpo_gridsearch_objective}
\end{marginfigure}

\begin{marginfigure}
  \centerfloat
  \includegraphics[width=0.95\textwidth, trim=0 0 0 0, clip]{figures/housing/Ejerlejlighed_v19_cut_all_Ncols_all_MAD_gridsearch_obj.pdf}
  \caption[Initial HPO Results for the Loss Function]
          {Evaluation score as a function of the loss function with the standard deviation over the \num{5} folds as errorbars for apartments.}
  \label{fig:h:hpo_gridsearch_halflife}
\end{marginfigure}

For the half-life and the choice of objective function these are shown in Figure~\ref{fig:h:hpo_gridsearch_objective} and \ref{fig:h:hpo_gridsearch_halflife}. In the first of the two figures it is easily seen that even though $T_{\frac{1}{2}}=\SI{10}{\yr}$ is the minimum, the uncertainties are so large that nothing can be concluded definitely with regards to the half-life parameter related to the weights $w(t)$. On the contrary, in the second figure there is a clear performance difference between the different loss functions where the Cauchy loss achieves the best (lowest) value of $f_\mathrm{eval}$ and especially the Squared Error is disregarded. 

The rest of the machine learning models thus continue with the following hyperparameters:

\begin{table}[h]
  \centerfloat
  \begin{tabular}{@{}llcl@{}}
               & $\log_{10}$  & Half-life & Loss function \\ \midrule
  Apartments   & False & \SI{10}{\yr} & Cauchy \\
  Houses       & False & \SI{20}{\yr} & Cauchy
  \end{tabular}
  \label{tab:h:initial_hpo}
\end{table}



\FloatBarrier
\section{Hyperparameter Optimization}
\label{sec:h:hyperparamater_optimization}
With the initial hyperparameter optimization finished, the actual training of the models was started. Again, two XGBoost models were fitted (one for apartments and one for houses) to the training set the hyperparameters were optimized using both random search and Bayesian optimization. These were each run for \num{100} iterations with \num{5}-fold cross validation and early stopping\sidenote{This takes a good \num{4} hours for the random search, \num{7} hours for the Bayesian optimization, and \num{90} minutes to optimize the number of estimators with early stopping for apartments when run on the local computing cluster, HEP. For the houses this takes a good \num{24} hours for the random search, \num{34} hours for the Bayesian optimization, and almost \num{5} hours to optimize the number of estimators with early stopping.}. The hyperparameters to be optimized were the following:
\begin{itemize}
  \item[] The \code{subsample} controls the fraction of row subsampling and is a number between \num{0} and \num{1}. 
  \item[] The \code{colsample_bytree} controls the fraction of column downsampling for each tree, so how many columns (or variables) each tree are allowed to fit to. Is a number between \num{0} and \num{1}.
  \item[] The \code{max_depth} controls the maximum depth of every tree. Is a positive integer (negative values means no maximum depth).  
  \item[] The \code{min_child_weight} variable controls when the decision tree algorithm should stop splitting a node into further nodes (and will thus turn it into a leaf). Is a positive integer.
  \item[] \code{reg_lambda} controls the L2 regularization term. Is a positive number. 
  \item[] \code{reg_alpha} controls the L1 regularization term. Is a positive number.
\end{itemize}

The ranges of the hyperparameters were chosen by a manual, iterative process of fitting a subset of the data (\SI{1}{\percent}--\SI{10}{\percent}) and making sure that the best hyperparameter is not sufficiently close to the boundary of the range; if it is, then the range is extended. The final ranges chosen are shown in Table~\ref{tab:h:hpo_ranges}. Here $\mathcal{U}(a, b)$ refers to a uniform distribution from $a$ to $b$, and $\mathcal{U}_\mathrm{int}(a, b)$ is the same only an integer distribution. 
\begin{margintable}
  \centerfloat
  \begin{tabular}{@{}ll@{}}
  Hyperparameter          &  Range                      \\ \midrule
  \code{subsample}        & $\mathcal{U}(0.5, 0.9)$           \\
  \code{colsample_bytree} & $\mathcal{U}(0.1, 0.99)$           \\
  \code{max_depth}        & $\mathcal{U}_\mathrm{int}(1, 20)$ \\
  \code{min_child_weight} & $\mathcal{U}_\mathrm{int}(1, 30)$ \\
  \code{reg_lambda}       & $\mathcal{U}(0.1, 4)$  \\
  \code{reg_alpha}        & $\mathcal{U}(0.1, 4)$
  \end{tabular}
  % \vspace{\abovecaptionskip}
  \vspace{3mm}
  \caption[PDFs Used in the Random Search]{\label{tab:h:hpo_ranges} Probability Density Functions used in the random search to draw new sets of values for the hyperparameters. Each hyperparameter is drawn from the distribution seen in the table.}
\end{margintable}

The fitting pipeline from this point on consists of first running both random search and Bayesian optimization. Their results are compared and the best of the two models is chosen. Finally, this model is re-fitted with early stopping where the learning rate $\eta$ is reduced from $\eta=0.1$ to $\eta=0.01$. In the end, one ends up with a model that has been hyperparameter optimized for preprocessing optimizations, loss functions, sample weights, normal XGB hyperparameters and finally the number of estimators. I have manually implemented this pipeline in Python since no other packages provide the same flexibility as a custom implementation that works fully automated. The results of the random search and the Bayesian optimization are shown in Figure~\ref{fig:h:CV_res_RS_parallel_coords_ejer_non_appendix} and \ref{fig:h:CV_res_BO_parallel_coords_ejer}. The corresponding plots for houses are shown in Figure~\ref{fig:h:CV_res_RS_parallel_coords_villa} and \ref{fig:h:CV_res_BO_parallel_coords_villa}. 

\begin{figure*}
  \centerfloat
  \includegraphics[width=0.99\textwidth, trim=0 0 0 0, clip]{figures/housing/Ejerlejlighed_v19_cut_all_Ncols_all_CV_viz_HPO_RS.pdf}
  \caption[Parallel Coordinate Plot of the Random Search Hyperparameter Optimization Results of XGBoost for Apartments]
          {Hyperparameter optimization results of XGBoost parameters of the housing model for apartments shown as parallel coordinates. Here shown for random search. } 
  \label{fig:h:CV_res_RS_parallel_coords_ejer_non_appendix}
\end{figure*}

As with the initial grid search, it is important to compare the results in Figure~\ref{fig:h:CV_res_RS_parallel_coords_ejer_non_appendix} with their uncertainties. Figure~\ref{fig:h:CV_res_RS_uncertainties_ejer} shows the value of the evaluation function along with its \num{1}$\sigma$  and \num{2}$\sigma$ uncertainties are plotted as a function of the iteration along the $x$-axis. The minimum value of $f_\mathrm{eval}$ is shown in red. Even though this is the minimum value, notice how most of the other iterations are within \num{1}$\sigma$. The plot for the Bayesian optimization in Figure~\ref{fig:h:CV_res_BO_uncertainties_ejer} shows the exploration phase in the first half of the iterations where it afterwards converge to a more flat minimum, however, this minimum was still worse than the one found with random search. 

\begin{figure}
  \centerfloat
  \includegraphics[width=0.95\textwidth, trim=10 20 10 10, clip]{figures/housing/Ejerlejlighed_v19_cut_all_Ncols_all_xgb_score_over_time_random.pdf}
  \caption[Hyperparameter Optimization: Random Search Results]
          {The results of running random search on apartments using the XGB-model. The \textcolor{red}{minimum (mean) loss} along with its uncertainty is shown in red, the \textcolor{blue}{means} for the different iterations of random search in blue, and as light blue bands are the \textcolor{blue}{one (and two) standard deviation(s)}, all as a function of iteration number.} 
  \label{fig:h:CV_res_RS_uncertainties_ejer}
\end{figure}

% The best of the random search and the Bayesian optimization models is chosen for the subsequent analysis by first reducing the learning rate to $\eta=0.01$ and then find the best number of estimators by early stopping. 
The evaluation function as a function of number of estimators, also known as the \emph{learning curve}, is shown in Figure~\ref{fig:h:CV_res_ES_learning_curve_ejer} for the combined early stopping model. This curve is the realization of the theoretical Figure~\ref{fig:ml:empirical_risk} in real data, where it first improves a lot and then finds a stable plateau, although it slowly increases. The minimum and its uncertainty are shown in red. To reduce the model complexity (with the further advantage of resulting in a faster model at prediction time) we keep the model with the lowest number of estimators that are still within $1\sigma$ of the minimum: see the orange point in the figure. This results in a model that contains less than a fifth of the number of trees and is thus also significantly simpler and faster at inference time. This is the final hyperparameter optimized model that will be used for the further analysis. 

\begin{figure}
  \centerfloat
  \includegraphics[draft=false, width=0.99\textwidth, trim=10 20 10 10, clip]{figures/housing/Ejerlejlighed_v19_cut_all_Ncols_all_xgb_early_stopping_fig.png}
  \caption[Early Stopping Results]
          {The results of early stopping on apartments using the XGB-model. The \textcolor{red}{minimum (mean) loss} along with its uncertainty is shown in red, the \textcolor{blue}{means} for the different iterations of random search in blue, and as light blue bands are the \textcolor{blue}{one (and two) standard deviation(s)}, all as a function of number of estimators (trees). In orange the \textcolor{orange}{\q{best} number of estimators} is shown, defined as the lowest number of estimators which are still within $1\sigma$ of the minimum value.} 
  \label{fig:h:CV_res_ES_learning_curve_ejer}
\end{figure}

\vspace{-0.5cm}


\FloatBarrier
\section{Results}
\label{sec:h:results}

The performances of the three different models (the RS-optimized, the BO-optimized and the best of the two optimized with early stopping) are shown in Figure~\ref{fig:h:CV_res_performance_ejer}. Here the distribution of the relative price prediction $\vec{z}$ of the test set is shown for the three models. In addition to the distributions, also the performance metrics are shown: the value of the evaluation function\sidenote{Written as \code{MAD}.} along with the fraction of relative price predictions that are within the specified percentage. In this particular instance it is seen that \SI{41.3}{\percent} of the predictions by the final early stopping model are less than $\pm\SI{5}{\percent}$ wrong, \SI{69.8}{\percent} within $\pm\SI{10}{\percent}$, and \SI{91.9}{\percent} within $\pm\SI{20}{\percent}$. Note that the differences between the three models are minor and that the distributions almost cannot be distinguished from each other. 

\begin{figure*}[h!]
  \includegraphics[width=0.95\textwidth, trim=0 10 10 5, clip]{figures/housing/Ejerlejlighed_v19_cut_all_Ncols_all_xgb_z_hist_metrics.pdf}
  \caption[Performance of XGB-model for Apartments]
          {Histogram of $z$-values of the XGB-model trained on apartments. The performance after hyperparameter optimization using \textcolor{blue}{random search} is shown in blue, for \textcolor{red}{Bayesian optimization} in red, and the final model which is optimized with \textcolor{green}{early stopping} in green.
          } 
  \label{fig:h:CV_res_performance_ejer}
\end{figure*}

An interesting observation from the performance metrics of the test set is the low value of $f_\mathrm{eval}=\SI{9.289}{\percent}$ (\code{MAD}). By looking at Figure~\ref{fig:h:CV_res_ES_learning_curve_ejer}, one would expect a test loss of around \SI{12}{\percent} assuming iid. samples. However, this assumption does not seem to be fulfilled for the test set. The performances of the realtors are also better for the test set than the training set, and by comparing the test set with the extra \num{2019} set, it seems to be \num{2018} that was an extra homogenous year. The realtors' performance is shown in Table~\ref{tab:h:realtor_mad_train_test_2019}. The \q{Tight} in the the table corresponds to the realtors performance on the tight versions of the different datasets. The performance of the XGB models on the tight test set is shown in Figure~\ref{fig:h:CV_res_performance_ejer_tight}, where it can be seen that the final model has $f_\mathrm{eval}=\SI{8.383}{\percent}$. 
\begin{margintable}
  \centerfloat
  \begin{tabular}{@{}rlll@{}}
          & Train               & Test                & \num{2019}          \\ \midrule
  Normal  & \SI{5.80}{\percent} & \SI{4.97}{\percent} & \SI{6.19}{\percent} \\
  Tight   & \SI{5.69}{\percent} & \SI{4.94}{\percent} & \SI{6.19}{\percent} 
  \end{tabular}
  \vspace{2mm}
  \caption[Realtors' MAD]{The MAD of the realtors' predictions for the normal and tight selections in the training, test, and \num{2019} datasets.}
  \label{tab:h:realtor_mad_train_test_2019}
\end{margintable}

The results for the final model are shown in Table~\ref{tab:h:results_ejer} for owner-occupied apartments and in in Table~\ref{tab:h:results_villa} for one family houses. The MAD is around \SI{9}{\percent} for apartments and \SI{16}{\percent} for houses, which is still worse than the realtors' predictions, yet still acceptable for a model that does not have any variables describing the indoor conditions. For apartments around \SI{40}{\percent} of all the predictions are within $\pm \SI{5}{\percent}$ and more than \SI{90}{\percent} are within $\pm \SI{20}{\percent}$ which is similar to the performance of the professional, automated property evaluations from e.g. \citet{bolighedBolighedUsikkerhedDatavurderingen}. The performances on the tight cuts are shown in Table~\ref{tab:h:results_ejer_tight} and \ref{tab:h:results_villa_tight}. 

\begin{table}
  \centerfloat
  \begin{tabular}{@{}lcccc@{}}
    {} &      MAD (\%) & $\leq 5\% (\%)$ &  $\leq 10\% (\%)$ &   $\leq 20\% (\%)$   \\
    \midrule
    Train & \num{7.83} & \num{47.90} & \num{75.74} & \num{93.97} \\
    Test  & \num{9.29} & \num{41.33} & \num{69.77} & \num{91.91} \\
    2019  & \num{9.89} & \num{38.85} & \num{66.76} & \num{90.04}    
  \end{tabular}
  \caption[Performance Metrics for Apartments]{Performance metrics for the final housing model trained and evaluated on apartments.}
  \label{tab:h:results_ejer}
\end{table}


\begin{table}
  \centerfloat
  \begin{tabular}{@{}lcccc@{}}
    {} &      MAD (\%) & $\leq 5\% (\%)$ &  $\leq 10\% (\%)$ &   $\leq 20\% (\%)$    \\
    \midrule
    Train & \num{14.12} & \num{28.95} & \num{51.89} & \num{78.04}  \\
    Test  & \num{15.79} & \num{25.61} & \num{47.52} & \num{75.14}  \\
    2019  & \num{16.50} & \num{24.01} & \num{45.89} & \num{74.22}     
  \end{tabular}
  \caption[Performance Metrics for Houses]{Performance metrics for the final housing model trained and evaluated on houses.}
  \label{tab:h:results_villa}
\end{table}



\section{Forecasting}
\label{sec:h:forecasting}

To gauge the predictive power of the final model over time, we predicted prices of the following month's sales and compared the predictions to the actual, sold prices. This is done on a month-by-month basis for all the months in the test set (\num{2018}) and \num{2019}. 
We applied two different methods of forecasting: \emph{static} forecasting where the model is only trained once, and \emph{dynamic} forecasting where the model is retrained after each month on all of the previous sales. These predictions allow the relative predictions $\vec{z}$ to be calculated and the MAD and the standard deviation (SD) of $\vec{z}$ are shown in Figure~\ref{fig:h:forecast_MAD_SD}. 
\begin{figure}
  \centerfloat
  \includegraphics[width=0.99\textwidth, trim=10 15 10 10, clip]{figures/housing/Ejerlejlighed_v19_cut_all_Ncols_all_xgb_forecast_prediction_MAD.pdf}
  \caption[Standard Deviation and MAD of the Static and Dynamic XGB Forecasts]
          {Performance of 1-month forecasts for apartments sold in \num{2018} and \num{2019}. For both plots the XGB model is trained on data up to (but excluding) 2018. Top) The performance of the static model's prediction for both the \textcolor{blue}{standard deviation (SD)} and \textcolor{red}{MAD} of the $z$-scores. Bottom) Same as above, however, based on a dynamic model, i.e. a model which is retrained after every month to include the previous month's sales.} 
  \label{fig:h:forecast_MAD_SD}
\end{figure}

In the top subplot the results for the static forecast are shown, whereas the dynamic results are shown in the bottom subplot. The errorbars are calculated using the usual variance of the standard deviation:
\begin{equation}
  \sigma_\mu  = \frac{\sigma}{\sqrt{N}}, \quad \sigma_\sigma = \frac{\sigma}{\sqrt{2N}},
\end{equation}
where $\sigma_\mu$ is the standard deviation of the mean and $\sigma_\sigma$ is the standard deviation of the standard deviation\sidenote{That this estimator is biased does not matter since we are in the large $N$ limit, $N \sim 1000$.} \autocite{Barlow:0471922951}.
Notice the large fluctuations in the standard deviations over time compared to the MADs which is an effect of MAD being a robust estimator. What is also interesting to note is that the MAD seems to increase over time for the static model, albeit slowly. In comparison, for the dynamic model this seem less pronounced. This figure not only shows the time dependence of the performance of the model, but also that the model is quite stable over time, at least for the dynamic model. 

Using the relative price predictions $\vec{z}$, we construct the Market Index, $\mathrm{MI}$. This is an index which measures the overall level of the Danish housing market based on the assumption that if the houses sold in a month are generally sold at a higher price than was predicted by the model, there can be two reasons for it: either the model was wrong or the market simply changed in the time span. With the latter assumption, the ratio between the prediction and the actual price of a residence is thus a measure of the market index:
\begin{equation}
  \begin{split}
    z_\mathrm{mi} &\equiv \frac{\hat{y}}{y} = 1 - \frac{\hat{y}-y}{y}  = 1-z \\
    \mathrm{MI}_\mathrm{mean} &= \mathrm{mean}(\vec{z}_\mathrm{mi}) \\
    \mathrm{MI}_\mathrm{median} &= \mathrm{median}(\vec{z}_\mathrm{mi}).
    \label{eq:h:market_index}
  \end{split}
\end{equation}
Here $\vec{z}_\mathrm{mi}$ is the vector of ratios between prediction and actual price, and the market index can then be estimated using either the mean $\mathrm{MI}_\mathrm{mean}$ or the median $\mathrm{MI}_\mathrm{median}$. 

Figure~\ref{fig:h:forecast_MarketIndex} shows the market indices on a monthly basis for the static and the dynamic models. In the top panel the market index is shown for the static model. Here the median $\mathrm{MI}_\mathrm{median}$  is consistently higher than \num{1}, around \SI{1}{\percent}. Compare this to the dynamic $\mathrm{MI}_\mathrm{median}$ in the lower plot which fluctuates around \num{1}. The mean $\mathrm{MI}_\mathrm{mean}$ is added to show that its fluctuations are higher than the medians. 
\begin{figure}
  \centerfloat
  \includegraphics[width=0.99\textwidth, trim=10 15 10 10, clip]{figures/housing/Ejerlejlighed_v19_cut_all_Ncols_all_xgb_forecast_prediction_MarketIndex.pdf}
  \caption[Market Index based on the Static and Dynamic XGB Forecasts]
          {The Market Index as defined in equation \eqref{eq:h:market_index} for the static and dynamic 1-month forecasts for \num{2018} and \num{2019}. The \textcolor{red}{mean} is plotted in red and the \textcolor{blue}{median} in blue.} 
  \label{fig:h:forecast_MarketIndex}
\end{figure}

\FloatBarrier
\section{Model Inspection}
\label{sec:h:model_inspection}

As mentioned previously, one of the most important aspects of advanced machine learning methods (in addition to getting accurate predictions) is to understand the model. As mentioned in \autoref{sec:ml:feature_importance}, it is possible to use SHAP values to inspect the trained model for both local predictions and for global feature importances $\phi_i^\mathrm{tot}$. An example of how to use SHAP to better understand a local prediction is shown in Figure~\ref{fig:h:shap_single_apartment}. In this figure, I visualize the SHAP values for a particular sale as a bar chart and compare it to the final prediction $\hat{y}$. Remember, for SHAP values the prediction is the sum of all of the SHAP values. The expectation value is denoted as \code{Bias} in the plot. To show all \num{143} variables would make the figure excessively large, so two extra bins have been added: the \code{Overflow} bar which is the sum of the remaining positive SHAP values and likewise with \code{Underflow} for negative values. The sum of all the green and red bars adds up to $\hat{y}=\SI{6.86}{\Mkr}$ in this particular instance and the actual sold value was $y=\SI{6.35}{\Mkr}$ Thus, in cases where there is a large discrepancy between the predicted and actual sales prices, one can use this tool to better understand why the prediction was estimated as it was. 

\begin{figure*}[ht!]
  \centerfloat
  \includegraphics[width=0.99\textwidth, trim=15 15 10 40, clip]{figures/housing/Ejerlejlighed_v19_cut_all_Ncols_all_SHAP_fig_loc=489266.pdf}
  \caption[SHAP Prediction Explanation for Apartments]
          {Model explanation for XGB model for a specific apartment. The bars are the variables in the dataset that the model found most important sorted after their importance for this particular apartment. The bias bar refers to the expected value of the model, which is simply the mean of the training set which acts as the naive prediction baseline. The \q{cutoff positive (negative)} bars are the sum of the remaining positive (negative) values that are not shown. On the right hand side of the plot is the model prediction shown. The model prediction is the sum of all of the bars in the left par (\SI{6.86}{\Mkr} in this example). The \textcolor{red}{negative} values are shown in red, \textcolor{green}{positive} ones in green, and the \textcolor{blue}{prediction value} in blue. 
          } 
  \label{fig:h:shap_single_apartment}
\end{figure*}

Figure~\ref{fig:h:shap_overview} shows which variables are most important on a global scale as measured by $\phi_i^\mathrm{tot}$. Here the variables are sorted according to the normalized $\phi_i^\mathrm{tot}$ which is shown in parentheses after each variable name. In the center of the plot is shown a dot-plot of the dataset plotted with the SHAP value on the $x$-axis and colored according to the feature value. 

The way to interpret this plot is as follows. Take a variable of interest, e.g. the area of the apartment \code{ArealBolig} with $\phi_\mathrm{ArealBolig}^\mathrm{tot}=\SI{5.35}{\percent}$. Every dot is a sale plotted as a function of their SHAP value with a spread such that the height corresponds to the SHAP distribution of that specific feature. For the area, it can be seen that there is a long tail towards high SHAP values, however, most of the samples have slightly negative SHAP values. The dots are colored according to their feature value and it can thus be seen that large apartments (red) are given a higher SHAP value than small apartments (blue); precisely as expected from the model. In contrary, when the total days on market (\code{LiggetidSamlet}) is large it pushes the prediction in the negative direction. 

\begin{figure}[ht!]
  \centerfloat
  \includegraphics[width=0.9\textwidth, trim=0 0 0 1, clip]{figures/housing/Ejerlejlighed_v19_cut_all_Ncols_all_xgb_tight_SHAP_vals_summary.pdf}
  \caption[Feature importance of Apartments Prices]
          {Feature importance of apartment prices using the XGB-model. The feature importance is measured using SHAP values. The variables are sorted top to bottom according to their overall feature importance, i.e. the previous public property valuation \code{EjendomdsVaerdi0} is the most important single feature. Along the $x$-axis is the impact on model output, in this example the price in \si{\Mkr} This axes is colored by the value of the feature, from \textcolor{blue}{low} (blue) to \textcolor{red}{high} (red). In this particular example we see that high values of the previous public property valuation has high, positive impact on the model prediction -- exactly as expected. This is exactly opposite the total days on market described by the variable \code{LiggetidSamlet} where a high value has a negative impact.  
          } 
  \label{fig:h:shap_overview}
\end{figure}

The SHAP software \citep{lundbergConsistentIndividualizedFeature2018} not only allows for 1D dependencies to be gauged, it also features the so-called \emph{interaction plots}. These plots show the 2D-dependence between the variable and the SHAP value colored according to a second variable. Since the previous public property valuation (PPPV) (\code{EjendomdsVaerdi0}) is the most important of the features, the interaction plot of this variable is seen in Figure~\ref{fig:h:shap_overview_interaction}. Here the SHAP value (of the \code{EjendomdsVaerdi0}) is plotted as a function of \code{EjendomdsVaerdi0}. This plot shows that the higher the PPPV, the higher the model output. The colors show how this trend depends on time by the variable \code{SalgsDato_siden0} which is the number of days since January \nth{1} 2009 that the apartment was sold. The plots shows that newer sales with a high PPPV has even higher SHAP values than older sales with the same high PPPV. This agrees with the fact that the market has gone up since 2009. On the other hand, for low PPPV apartments the relationship with time is opposite, however, the effect is much smaller here. The SHAP software chose to color by \code{SalgsDato_siden0} since this is the variable which explains most of the variation for a given PPPV. 

\begin{figure}
  \centerfloat
  \includegraphics[draft=false, width=0.98\textwidth, trim=15 15 40 40, clip]{figures/housing/Ejerlejlighed_v19_cut_all_Ncols_all_xgb_tight_SHAP_vals_interaction_Vaerdi0.pdf}
  \caption[Feature Importance Interaction Plot for Apartments]
          {Interaction plot of the feature importances for the XGB model on apartments. Here shown for the previous public property valuation (\code{EjendomdsVaerdi0}) colored according to the sales date (\code{SalgsDato_siden0}).} 
  \label{fig:h:shap_overview_interaction}
\end{figure}

What can be concluded from this plot is that apartments with a high PPPV that were sold recently were sold for more than similar apartments (with the same PPPV) which were sold a longer time ago. This is effectively the time dependence of the sales prices. These interaction plots serves as helpful sanity checks to see that the model is actually learning reasonable relationships between the variables. 

\FloatBarrier
\section{Multiple Models}
\label{sec:h:multiple_models}

In addition to the XGBoost model, several other models were also tested. During the project, the LightGBM package \autocite{keLightGBMHighlyEfficient2017} (LGB), was released and started gaining traction in the ML community, especially for large-scale data analysis. LightGBM is also a type of boosted decision tree, however, in comparison to XGBoost, LightGBM implements some extra binning and categorical assumptions that greatly speeds up the fitting process. The LightGBM model was trained in the same way as the XGBoost model with the same hyperparameter optimization process and range for the hyperparameters. 

To compliment the boosted decision trees, a simple linear model (LIN) with $L_2$ loss (ridge regression) was fit. Before it was fit, all of the NANs were median-imputed. This means that all invalid values were replaced with the median for each variable\sidenote[][-6mm]{Since the linear model cannot deal with NANs naturally like XGBoost and LightGBM can.}. All the features were scaled\sidenote[][3mm]{Scaling of input features is generally an important preprocessing step for non-tree based ML models.} with a robust scaler from Scipy \autocite{virtanenSciPyFundamentalAlgorithms2019} in the $(25, 75)$ quantiles range and the regularization parameter was hyperparameter optimized. This linear model is quick and easy to both implement and fit, and can be seen as the simplest baseline model.

The $K$-nearest neighbors (KNN) algorithm was fitted to the data with a data preprocessing pipeline similar to the linear model with median imputation and robust scaling of the input features. The number of neighbors, $K$, was hyperparameter optimized together with the $p$-norm of the metric, $p \in \left\{ 1, 2\right\}$ (Manhattan, Euclidean, see also \autoref{subsec:regularization}).

Finally, support vector machines\sidenote{Known as support vector regression when applied to regression problems \autocite{awadSupportVectorRegression2015}.} (SVR) were used with the same preprocessing pipeline as the previous two methods and hyperparameter optimized for the $L_2$ regularization parameter $C$ and the kernel coefficient $\gamma$ for the radial basis function (RBF) kernel. 

\begin{figure*}[ht!]
  \centerfloat
  \includegraphics[draft=false, width=0.98\textwidth, trim=10 130 40 10, clip]{figures/housing/Ejerlejlighed_v19_cut_all_Ncols_all_all_models.pdf}
  \caption[Performance Comparison of Multiple Models for Apartments]
          {Distribution of the relative predictions $\vec{z}$ for apartments for the five different models and the ensemble model. The left plot shows the normal histogram (with a linear scale), whereas the right one has a log-scaled $y$-axis to better see the tails of the distributions. The MAD scores for the different models is further shown in the left plot.} 
  \label{fig:h:multiple_models}
\end{figure*}

The results for the five different models are shown in Figure~\ref{fig:h:multiple_models}. The two histograms are the same, the only difference is the logarithmic $y$-axis in the right subplot to better visualize the tails of the distributions. The figure shows how XGBoost and LightGBM are the best-performing models together with the ensemble (\code{ENS}) of the different models. This combined model is explained below. The evaluation score of the test set, \code{MAD}, is also shown on the figure. The scores confirm the visual clue: that the ensemble model performs the best, followed by the XGBoost model.


The five different models -- XGB, LGB, LIN, KNN, SVR -- should be able to each capture different parts of the hyperdimensional phase space and an ensemble of these models would thus be expected to be as good as or better than the best of the individual models. This kind of ensemble model is sometimes called a meta learner or a \emph{super learner} in the statistics community \autocite{vanSuperLearner2007}. 

To make sure that the ensemble model is not just retraining on the training set, and thus end up overfitting, we follow the process from \citet{polleySuperLearnerPrediction2010}. Using cross-validation for time-series\sidenote{See \autoref{subsec:cross_validation}.} data with $10$ folds, the training data is split up into folds sorted by time. Each fold is fitted with all five models, and the prediction of the next fold is made for all five models. This is repeated for the remaining folds until one ends up with a matrix of predictions $\vec{Z} \in \mathbb{R}^{(N \times 5)}$ for $N$ training samples. Since all the folds in $\vec{Z}$ consists of predictions on unseen data\sidenote{The predictions for the very first fold is based on training data.}, this prevents overfitting. The meta learner fits $\vec{Z}$ to the actual predictions of the training data $y$ in the usual way. The combination of a meta learner fitted to the predictions of individual models is called an ensemble model.

At first an XGBoost model was used as the meta learner yielding decent results, however, still performing worse than the single XGB model ($\mathrm{MAD} = \SI{9.57}{\percent}$). To better understand the issue, the meta learner was changed to a linear model which would basically just compute a weighted average of the different models:
\begin{equation}
  \label{eq:h:meta_learner}
  \Psi_\mathrm{meta}(\vec{x}) = \sum_{i=1}^5 \alpha_i \Psi_i(\vec{x}),
\end{equation}
where $\bm{\alpha}$ is a vector of the weights for the meta learner and $\Psi_i$ is one of the individual ML models. The linear model performed even worse than using XGB as the meta learner ($\mathrm{MAD} = \SI{10.48}{\percent}$), yet it was more transparent. During the debugging process it was realized that none of these models actually optimize the evaluation function, MAD, directly. The XGBoost model was trained using the Cauchy loss found earlier and the linear regression model using a simple squared error loss. Since a simple weighted average should work as the meta learner \citep{polleySuperLearnerPrediction2010}, a custom algorithm for finding $\bm{\alpha}$ according to MAD was implemented.

Given the training data, the evaluation function as a function of $\bm{\alpha}$ was minimized using the MINUIT algorithm \cite{1975CoPhC..10..343J} via the iminuit \cite{iminuit} Python interface. It yielded decent results, yet they were all very dependent on the initial parameter of the fit indicating many local minima. A scan over the 5-dimensional hyperspace in steps of $0.01$ was thus conducted and the result of this scan was used as the new initial parameter in the minimization routine. This yielded the following result:
\begin{equation}
  \bm{\alpha} = \begin{bmatrix*}[r] \alpha_\mathrm{LIN} \\  \alpha_\mathrm{KNN} \\ \alpha_\mathrm{SVR} \\ \alpha_\mathrm{XGB} \\ \alpha_\mathrm{LGB} \end{bmatrix*} = \begin{bmatrix*}[r] 0.202 \\  0.002 \\ 0.001 \\ 81.302 \\ 20.002 \end{bmatrix*} \times 10^{-2}.
\end{equation}
The fact that it sums to more than \num{1} just corresponds to an overall scaling. When using the found $\bm{\alpha}$ in equation \eqref{eq:h:meta_learner}, one gets the ensemble model (ENS) shown in Figure~\ref{fig:h:multiple_models} with a $\mathrm{MAD} = \SI{9.20}{\percent}$. This value is the evaluation loss on the test set based on only the training data and it outperforms all of the individual models. Even though the ensemble model has the best evaluation score, the performance gain is very little. Since it also requires a lot of training time (five different models have to be fitted), the ensemble model should mostly be seen as a proof of concept.

The paragraphs above refer to apartments, however, the intermediate results for houses showed the same pattern. The combined model along with its performance is shown in Figure~\ref{fig:h:multiple_models_villa}

\section{Discussion}
\label{sec:h:discussion}

The subproject of estimating housing prices has focussed a lot on experimenting with different machine learning models and how to optimize them. As shown in the previous sections, the choice of ML model is by far the most important. Actually, the gain from hyperparameter optimization is quite small, especially considering the amount of time spent on it\sidenote{Not only user-time while programming it, but also the computational ressources spent.}. With the dataset at hand, decent results were achieved, however, they were nowhere near the performance of the realtors' predictions. 

There are two main reasons for this. The first being that realtors are educated within this field and thus have developed the skills required for estimating the price of a house during many years of hard work. The second reason is the fact that the realtors have access to a lot more information than the ML models have. The ML models seen so far have not been trained on any \emph{indoor variables}. The area of the house, the number of rooms, the name of the street, and the distance to a highway are all variables that are in the data set but none of them describe the overall quality of the house, the maintenance level, the age of the kitchen or bathroom. These indoor features are invisible to the ML model. 

During the project it was investigated how to get access to these variables. At first the online images from each sale were suggested. Unfortunately, it turned out that Boligsiden only have the right to use them while a residence is for sale; when it is sold all rights return to the photographer. Fortunately, the images are not the only piece of data that provide more information about the condition of the residence: also the descriptions do that. 

The descriptions turned out to be available for most of the sales and were investigated for a short period. At that time of the small investigation, the MAD for (a subset of) the apartments was around $\pm \SI{14}{\percent}$ and $\pm \SI{20}{\percent}$ for houses. By using methods from the big natural language processing (NLP) community within the field of machine learning, it was possible to reduce the MAD to around $\pm \SI{12}{\percent}$ and $\pm \SI{15}{\percent}$ for apartments and houses respectively. From the improvement in performance it is noticeable how apartments in general are much more uniform compared to houses where the \q{inside} is more decisive regarding the price. 

\begin{figure}[h!]
  \centering
  \includegraphics[width=0.8\textwidth]{figures/housing_text/villa_tfidf.pdf}
  \caption[Feature Importance of Villas With Descriptions]
          {Feature importance of villas when the descriptions are also used. The variables are sorted top to bottom according to their overall feature importance, i.e. the previous public property valuation \code{EjendomdsVaerdi0} is the most important single feature. Along the $x$-axis is the impact on model output, in this example the price in \si{\Mkr} This axis is colored by the value of the feature, from \textcolor{blue}{low} (blue) to \textcolor{red}{high} (red). In this particular example we see that high values of the previous public property valuation has high, positive impact on the model prediction -- exactly as expected. This is in contrast to the word \q{requires} (\colorbox{light-gray}{\texttt{trænger}}) where a high value has a negative impact.} 
  \label{fig:h:shap_text}
\end{figure}

The methods for translating the text to numerical variables decipherable by classical ML models were for instance simple \emph{bag of words} (BOW) models and \emph{Term Frequency, Inverse Document Frequency} (TF-IDF) but also slightly more advanced statistical tools such as \emph{Latent Dirichlet Allocation} (LDA). 

An old example of this was a house based model that was trained with the five numerical variables\sidenote{With the Danish names: \code{Ejendomsvaerdi0}, \code{GeoPostNr}, \code{ByggeAAr}, \code{Afstand_Kyst}, and \code{BeregnetAreal}.}: the PPPV, postal code, year of construction, distance to shore, and the weighted area. In addition ti the numerical variables, also the text descriptions (encoded with TF-IDF) were trained on. Figure~\ref{fig:h:shap_text} shows the SHAP summary plot of the trained model. The three most important features turned out to be the numerical ones, however, the word \code{flot} (\q{pretty}) was ranked \nth{4}. The model also learnt that \code{flot} has a positive impact on the price compared to \colorbox{light-gray}{\texttt{trænger}} (\q{requires}) which has a negative impact.
% \code{Ejendomsvaerdi0} (PPPV), \code{GeoPostNr} (postal code), \code{ByggeAAr} (year of construction), \code{Afstand_Kyst} (distance to shore), and \code{BeregnetAreal} (weighted area) 

The descriptions turned out to be more time-consuming to extract for Boligsiden and along with the fact that the overall deadline was quickly approaching, the remaining time was focussed on the main part of the project, the quark gluon discrimination. Given more time, the text analysis would definitely be the first step for further improving the accuracy and precision of the price predictions. 

Another step would be to apply more modern deep learning\sidenote{Basically advanced neural networks with many layers.} methods. These methods were briefly experimented with in the initial stages of this subproject but showed inferior performance compared to BDTs. It is generally accepted in the ML community (with some modifications) that neural networks underperform, or at least not outperform, classic ML methods on structured data\sidenote{Structured data are data that can be described by a spread sheet, i.e. has a well-defined number of variables and observations. This is why it is also known as \emph{tabular data}.} \autocite{klambauerSelfNormalizingNeuralNetworks2017}. Most often they have the inherent complexity needed to perform as well as other ML methods, however, this requires extensive architecture optimization, or, in short; the hypothesis space for neural networks is much larger than for classical ML methods and thus requires more care to avoid overfitting.

\section{Conclusion}
\label{sec:h:conclusion}

XXX \TODO




% \chapter{Danish Housing Prices}
% \label{ch:housing_price_analysis}
% \epigraph{\textit{``Buy land, they’re not making it anymore.''}}{--- Mark Twain}
% \epigraph{\textit{``It’s tangible, it’s solid, it’s beautiful. It’s artistic, from my standpoint, and I just love real estate.''}}{--- Donald Trump}
% \newthought{Housing markets} have always been a playing field for economists, property speculator, real estate developers, and realtors. The author of this thesis is by no metric any of these, not even close to, yet, when the issue of estimating Danish housing prices came up, it was too much of a challenge just lying there to let it go. Estimating housing prices is a classical economical discipline with papers from the Danish National Bank developing a regional model of the Danish housing market \autocite{hviidWorkingPaperRegional2017} to analysis of the financial crisis in 2008-2009 and its effects on the Copenhagen metropolitan area \autocite{mulalicFinancialCrisisDiverging2017}. 

% \begin{figure}
%   \includegraphics[width=0.98\textwidth, trim=10 20 10 10, clip]{figures/housing_price_index_dst/housingindex_wide.pdf}
%   \caption[Danish Housing Price Index]
%           {Price Index of the Danish housing market. Prince index of Danish one-family houses and owner-occupied apartments where \textcolor{red}{houses} are shown in red and \textcolor{blue}{apartments} in blue, where full lines are for the entirety of Denmark and dashed lines are only for Copenhagen. Errorbars (scaled up with a factor of 2) are shown as colored bands. The price index and its uncertainty is based on numbers from \citet{dstPriceIndexEJ14}, however, rescaled to 100 in 2000 (instead of 2006 as it was in the data).}
%   \label{fig:h:price_index}
% \end{figure}

% If one takes a look at the time development of the Danish housing\sidenote[][1cm]{\q{Housing} means both actual houses and privately owned apartments in this project.} market, the Danish governmental organization for statistics, Statistics Denmark, releases a price index \citep{dstPriceIndexEJ14} for both one-family houses (OFH) and owner-occupied apartments (OOA) quarterly, this is shown in Figure~\ref{fig:h:price_index}. Here it is easy to see the effect of the financial crisis around 2008, but also the steady increase in the housing market since then in both Copenhagen and the entire country. 

% The goal of this subproject\sidenote{The housing price estimation subproject.} is not to predict any future collapse as the financial markets as we saw upwards of 10 years ago. Instead, it is to learn patterns in the price of houses in steady times. The goal is training a computer\sidenote{In contrary to \citet{hviidWorkingPaperRegional2017, mulalicFinancialCrisisDiverging2017} and others who base their models on macro-economic principles.} to automatically to find these patterns and see if we can improve this model.

% In chapter XXX I will introduce the data, do EDA and so on. Then I will introduce the models in XXX, start the HPO in chapter XXX, and talk about the results in XXX, the SHAP values in XXX and the time predictions in XXX. 

% This subproject was done in collaboration with Boligsiden without whom it would not have been possible. During this project, common python data science tools from the SciPy ecosystem\autocite{virtanenSciPyFundamentalAlgorithms2019} such as NumPy, Matplotlib, Pandas, Scikit-Learn, Scipy has been used extensively and should thus also be mentioned. 

% \section{Data Preparation and Exploratory Data Analysis}
% \label{sec:h:data_cleaning}
% \epigraph{\textit{``\SI{80}{\percent} of data science is cleaning the data and \SI{20}{\percent} is complaining about cleaning the data.''}}{--- Anthony Goldbloom, Kaggle}


% The first part of any data science project is actually getting the data and being able to read it. This has been an iterative process that has improved over time. The last data transfer we got was September \nth{3} 2019 and consisted of a \SI{522.4}{\mega\byte} CSV file with dimensions $(\num{711212}, \num{171})$. This section will go through the data cleaning process.

% Before any further data analysis is performed, all of the data is loaded, except three columns which only contain internal information for Boligsiden\sidenote{The variables are \code{Sag_Kvhx}, \code{Enhed_GOP_BoligtypeKode}, and \code{Bygning_GOP_Matrikelnr}.}. To get a better understanding of all of the variables, histograms showing the one-dimensional distributions of all of the variables were made.

% Four particularly interesting ones are seen in Figure~\ref{fig:h:variable_overview}: 
% the distribution of the date of the sale \code{SalgsPris} in subfigure~\subref{fig:h:variable_overview_date},  
% the distribution of the type of residence \code{SagtypeNr} in subfigure~\subref{fig:h:variable_overview_type},
% the distribution of the longitude of the residence \code{GisX_WGS84} in subfigure~\subref{fig:h:variable_overview_longitude}, and
% the distribution of the area of the residence \code{ArealBolig} in subfigure~\subref{fig:h:variable_overview_area}.

% \begin{figure*}
%   \centering
%   % \vspace*{-\abovecaptionskip}
%   \subfloat[\label{fig:h:variable_overview_date}]{\qquad}
%   \includegraphics[width=0.45\textwidth, page=2, trim=15 15 15 15, clip]{figures/housing/overview_fig.pdf}\hfil
%   \subfloat[\label{fig:h:variable_overview_type}]{\qquad}
%   \includegraphics[width=0.45\textwidth, page=6, trim=15 15 15 15, clip]{figures/housing/overview_fig.pdf}
%   \subfloat[\label{fig:h:variable_overview_longitude}]{\qquad}
%   \includegraphics[width=0.45\textwidth, page=20, trim=15 15 15 15, clip]{figures/housing/overview_fig.pdf}\hfil
%   \subfloat[\label{fig:h:variable_overview_area}]{\qquad}
%   \includegraphics[width=0.45\textwidth, page=23, trim=15 15 15 15, clip]{figures/housing/overview_fig.pdf}
%   \caption[Distributions for the housing price dataset]{Distributions of four out of the 168 input variables. 
%            Subplot ~\protect\subref{fig:h:variable_overview_date} shows the date of the sale, 
%            Subplot ~\protect\subref{fig:h:variable_overview_type} shows the type of residence,
%            Subplot ~\protect\subref{fig:h:variable_overview_longitude} shows the longitude,
%            Subplot ~\protect\subref{fig:h:variable_overview_area} shows the area og the house.}
%   \label{fig:h:variable_overview}
%   \vspace{\abovecaptionskip}
% \end{figure*}

% The distribution of the date of sale, Figure~\ref{fig:h:variable_overview} \subref{fig:h:variable_overview_date}, is an interesting variable because it shows how Boligsiden has been collecting more and more data over time. Here 2007 and 2019 are clear outliers since their current database only contains sales from the end of 2007, and 2019 only contains data from the first eight months of the year. The type of residence is a discrete code that Boligsiden uses to differentiate between different types of residences. The mapping between code and description can be seen in Table~\ref{tab:h:salgstype}. 

% \begin{margintable}
%   \centering
%   \begin{tabular}{@{}rl@{}}
%   % \toprule
%   Type & Name           \\ 
%   \midrule
%   100  & Villa          \\ 
%   200  & Rækkehus       \\
%   300  & Ejerlejlighed  \\
%   400  & Fritidsbolig   \\
%   401  & Kolonihave     \\
%   500  & Andelsbolig    \\
%   600  & Landejendom    \\
%   700  & Helårsgrund    \\
%   800  & Fritidsgrund   \\
%   900  & Villalejlighed \\
%   1000 & Kvæggård       \\
%   1100 & Svinegård      \\
%   1200 & Planteavlsgård \\
%   1300 & Skovejendom    \\
%   1400 & Lystejendom    \\
%   1500 & Specialejendom \\ 
%   % \bottomrule
%   \end{tabular}
%   \vspace{\abovecaptionskip}
%   \caption{XXX \TODO.}
%   \label{tab:h:salgstype}
% \end{margintable}

% In this project only one-family houses -- \q{Villas} in Danish -- with code \num{100} and owner-occupied apartments -- \q{Ejerlejlighed} in Danish -- with code 300 are considered. As can be seen from Figure~\ref{fig:h:variable_overview} \subref{fig:h:variable_overview_type} these two types of residences are also the most frequent sales with close to \num{400000} and \num{150000} sales in total. The longitude distribution, Figure~\ref{fig:h:variable_overview} \subref{fig:h:variable_overview_longitude}, is mostly interesting due to fact that it clearly shows how the Great Belt and especially the Baltic Sea separates Denmark into three parts; the Western part, the Eastern part, and then Bornholm. Note that more than \SI{5}{\percent} of the residences' locations are unknown values: Not A Number \q{NANs}. The distribution of the area, Figure~\ref{fig:h:variable_overview} \subref{fig:h:variable_overview_area}, shows that most residences are between \SI{50}{\meter^2} and \SI{200}{\meter^2}, as expected in Denmark. However, a relatively large part of the residences, \SI{2.5}{\percent}, are listed as having an area of \SI{0}{\meter^2}. All of the 1D-distributions can be seen in Figure~\ref{fig:h:variable_overview_all_1}--\ref{fig:h:variable_overview_all_14}. 

% The geographic distribution of sales can be seen in Figure~\ref{fig:h:geo_overview}. The residences are coloured according the square meter price in Figure~\ref{fig:h:geo_overview} \subref{fig:h:geo_overview_sqm_price} and according to the sales price in Figure~\ref{fig:h:geo_overview} \subref{fig:h:geo_overview_sales_price}. Notice the strong correlation between the distance to water and the square meter price, a correlation that is less visible when looking at the sales price. Since these plots each contain \num{674647} points\sidenote{Only sales with a valid GPS-coordinate and area of residence are shown}, over-plotting quickly becomes an issue. To circumvent this, the great software package called DataShader was used \autocite{bednarDatashaderRevealingStructure2019} which in a simple, consistent, and not at least fast manner allows one to plot big data.
% % \begin{figure}
% %   \includegraphics[width=0.9\textwidth, trim=0 0 0 0, clip]{figures/housing/Denmark_Overview_SqmPrice.png}
% %   \caption[Geographic overview of square meter price in Denmark]
% %           {Geographic overview of square meter prices for houses and apartments in Denmark (excluding Bornholm for visual purposes). Notice the strong correlation with the major cities and the shore line. Also notice the three outliers west of Jutland. }
% %   \label{fig:h:geo_overview}
% % \end{figure}

% \begin{figure}
%   \centering
%   % \vspace*{-\abovecaptionskip}
%   \subfloat[\label{fig:h:geo_overview_sqm_price}]{\qquad}
%   \includegraphics[draft, width=0.4\textwidth]{figures/housing/Denmark_Overview_SqmPrice.png}\hfil
%   \subfloat[\label{fig:h:geo_overview_sales_price}]{\qquad}
%   \includegraphics[draft, width=0.4\textwidth]{figures/housing/Denmark_Overview_SalesPrice.png}
%   \caption[Distributions for the housing price dataset]{Geographic distribution of the sold residences. 
%            In subplot ~\protect\subref{fig:h:geo_overview_sqm_price} the sales are colored according to their square meter price and in subplot ~\protect\subref{fig:h:geo_overview_sales_price} according to the sales price. 
%            }
%   \label{fig:h:geo_overview}
%   \vspace{\abovecaptionskip}
% \end{figure}


% The most important of the features is the sales price, called \code{SalgsPris} in the dataset, and its distribution shown in Figure~\ref{fig:h:price_overview_price}. This is a positively skewed distribution that shares visual similarities with a log-normal distribution. The mode of the distribution is \SI{1.1}{\Mkr} and the median\sidenote{The mean is \SI{2.0}{\Mkr} but this value is heavily influenced by a few very high values.} is \SI{1.6}{\Mkr}, but remember that this variables is shown for all types of residences.  

% \begin{figure}
%   \includegraphics[width=0.98\textwidth, page=1, trim=15 15 15 15, clip]{figures/housing/overview_fig.pdf}
%   \caption[Histogram of prices of houses and apartments sold in Denmark]
%           {Histogram of prices of houses and apartments sold in Denmark.}
%   \label{fig:h:price_overview_price}
% \end{figure}

% \subsection{Correlations}

% Having shown the \num{1}D-distributions of all the different variables in the previous section, the next step would be to look at the correlations between the variables. Since there are \num{168} input variables, it is almost impossible to understand every inter-variable correlation, however, it is tried in Figure~\ref{fig:h:correlations_all_lin} in the appendix. Here the correlation between all numerical variables that are not obviously related to other variables (like the GPS-coordinates that are in both lattitude-longitude and ETRS89\sidenote{European Terrestrial Reference System \num{1989}.} format), and with the condition that it has to have one inter-variable correlation higher than $|\rho| > \SI{30}{\percent}$, are plotted as a $(86 \times 86)$-dimensional heatmap.

% Even though inter-variable correlations are important in the exploratory data analysis (EDA) phase, what is more important is to get a better understanding of how the input variables correlate with the output variables; the sales price. This is shown in Figure~\ref{fig:h:corr_lin} for the variables where $\abs{\rho} > \SI{10}{\percent}$. It is the previous property evaluations, \code{EjdVurdering_EjendomsVaerdi} that are positively correlated the most with the sales price, which does not come as any huge surprise. Other positively correlated variables are the cost of ownership\sidenote{This a great example of the fact that correlation does not imply causation}, area, number of bathrooms, its longitude, and distance to nearest wind mill. In the other end, the local income tax, \code{Kommune_SkatteProcent}, is the variable that is the most negatively correlated to the sales price, followed by the geographical variables related to province, municipality, and postcode.

% \begin{figure*}
%   \centerfloat
%   \includegraphics[width=0.98\textwidth, trim=0 0 0 40, clip]{figures/housing/lin_correlation.pdf}
%   \caption[Linear correlation between variables and price]
%           {Linear correlation between variables and price for variables where the correlation coefficient $\rho$ is $\abs{\rho} > \SI{10}{\percent}$.}
%   \label{fig:h:corr_lin}
% \end{figure*}

% The correlation $\rho$ used above is the linear correlation which thus only captures linear relationships between variables. All modern machine learning algorithms, however, are also able to capture higher-order correlations and thus a higher-order correlation measure is needed. The maximal information coefficient (MIC), which is a value between \num{0} and \num{1}, is such a non-linear measure. It finds correlation based on the intuition that if two variables are correlated, it should be possible to split the data up into smaller grids where, if they are correlated, the grid that contain points should contain many points and the rest of the grids should be (relatively) empty \autocite{reshefDetectingNovelAssociations2011}. This is in comparison to two uncorrelated variables which would simply display noisy behavior and only have few grids with many points in. \citet{albanesePracticalToolMaximal2018a} extended on this idea and developed the computationally efficient algorithm called \code{MICtools} which computes the estimator for MIC: $\mathrm{MIC}_e$. An example of this tool non-linear correlation is seen in Figure~\ref{fig:h:MIC_example_small}. Here the relationship between the normal linear correlation $\rho$ and $\mathrm{MIC}_e$ can be seen for four synthetic datasets. Notice that $\mathrm{MIC}_e$ does it particularly well for the sine wave, and decent for the line and parabola, but only slightly captures the relationship for the exponential growth. The influence of noise on $\rho$ and $\mathrm{MIC}_e$ can be see in Figure~\ref{fig:h:MIC_example} in the appendix. 

% \begin{figure}
%   \centering
%   \includegraphics[width=0.99\textwidth, trim=10 10 10 10, clip]{figures/housing/MIC_test_small.pdf}
%   \caption[MIC non-linear correlation.]
%           {MIC non-linear correlation. XXX \TODO}
%   \label{fig:h:MIC_example_small}
% \end{figure}

% Using $\mathrm{MIC}_e$ as the correlation measure between the numerical variables and the sales prices, the variables with a $\mathrm{MIC}_e$-score higher than \SI{10}{\percent} can be seen in Figure~\ref{fig:h:corr_MIC}. Again, the previous property evaluations are the most correlated features to the sales price followed by the parish code, \code{SogneKode}. In general the geographical variables score high here, with also the post code, municipality number, and longitude. 

% \begin{figure*}
%   \includegraphics[width=0.9\textwidth, trim=0 0 0 40, clip]{figures/housing/MIC_plot.pdf}
%   \caption[Non-linear correlation between variables and price]
%           {Non-linear correlation between variables and price using Maximal Information Coefficient (MIC) for variables where $\text{MIC}>10\%$.}
%   \label{fig:h:corr_MIC}
% \end{figure*}

% \subsection{Validity of input variables}

% The fact that some of the variables contains considerable amount of invalid values, NANs, requires this to be taken into account before any further analysis. The validity, defined as the percentage of valid observations, of every variable is shown in Figure~\ref{fig:h:nans}. Here the 168 variables variables are grouped together into 25 variables where each group share the same validity. An example of this are all of the 16 different distance-variables\sidenote{Distance to: Fængsel, Hede, Højspaendingsledning, Industri, JernbaneSynlig, Kirke, Kirkegård, Kyst, Landingsbane, Motorvej, MotorvejTilFraKørsel, RekreativtOmråde, Rigsgrænse, Sportsanlæg, Strand, Vindmølle}. We see that most of the variables have validities around more than 
% \SI{85}{\percent}, however, a few of the variables, especially information about the building, \code{BygningsInfo}, have validities less than \SI{20}{\percent}. 

% \begin{figure}
%   \includegraphics[width=0.98\textwidth, trim=0 0 0 0, clip]{figures/housing/missing_bar.pdf}
%   \caption[Validity of input features]
%           {Percentage of valid counts for each variable grouped together in categories.}
%   \label{fig:h:nans}
% \end{figure}

% To see how closely related the different validity groups are, one can look at the dendrogram in Figure~\ref{fig:h:nans_dendrogram}. The dendrogram is based on a hierarchical clustering algorithm \citep{virtanenSciPyFundamentalAlgorithms2019} where the different groups are clustered according to their linear correlation. This diagram is supposed to be read in a top-down approach, where it can be see that the name of the street, \code{Vejnavn}, and the number of the residence, \code{HusNr}, correlate a lot and are thus clustered very early. The year of the last time the residence was rebuilt or greatly modified, \code{OmbygningsAAr}, does not correlate strongly with any of the other variables and is thus the last variable to be clustered together. 

% \begin{figure}
%   \includegraphics[width=0.98\textwidth, trim=35 10 0 10, clip]{figures/housing/missing_dendrogram.pdf}
%   \caption[Validity Dendrogram]
%           {Validity Dendrogram \TODO.}
%   \label{fig:h:nans_dendrogram}
% \end{figure}

% To see a heatmap of the inter-variable correlations, see Figure~\ref{fig:h:nans_heatmap} in the appendix. 

% \subsection{Cuts}
% Given the 1D input variable distributions and their validity, we apply some very basic cuts before any further analysis. These cuts are seen in Table \ref{tab:h:initial_cuts}. Sales type, \code{OISSalgsType}, is a OIS\sidenote{OIS is short for \q{Den Offentlige Informationsserver}, the Danish public information server, and it collects information about Danish residences \autocite{oisWwwOISDk}.} code which describes what type of sale it is: when it is \num{1} it is considered a normal sale, compared e.g. forced sales. These cuts are seen as the minimum requirements for what constitutes a curated dataset with no obvious outliers. The reason why the time requirement is applied, is to reduce the effect of the financial crisis to creep into the model. 

% \begin{table}[h]
%   \begin{tabular}{@{}llcr@{}}
%   % \toprule
%                & Description                                          & Remaining & Removed \\ 
%   \midrule
%   Area         & \SI{20}{\meter\squared} $\leq$ Area $\leq$ \SI{500}{\meter\squared} & \num{689140}              & \num{23666}     \\
%   Price        & \SI{0.1}{\Mkr} $\leq$ Price $\leq$ \SI{100}{\Mkr}                   & \num{687546}              & \num[group-minimum-digits=3]{1594}      \\
%   Type         & Has sales type \num{1}                                              & \num{605415}              & \num{82131}     \\
%   GPS          & Has valid GPS coordinates                                           & \num{578860}              & \num{26555}     \\
%   Private      & Only non-business sales                                             & \num{549140}              & \num{29720}     \\
%   Time         & Sold in \num{2009} or later                                         & \num{520548}              & \num{28592}     \\
%   \bottomrule
%   \end{tabular}
%   \vspace{\abovecaptionskip}
%   \caption{XXX \TODO.}
%   \label{tab:h:initial_cuts}
% \end{table}

% \section{Feature Augmentation}
% \label{sec:h:feature_augmentation}

% \begin{margintable}
%   % \begin{tabular}{llr}
%   \begin{tabular*}{\textwidth}{l @{\extracolsep{\fill}} lr}
%   % \toprule
%   String & Explanation  & Code \\ \midrule
%   NAN    & No side door & 0    \\
%   TH, TV & Right, left  & 11   \\
%   MF     & Center       & 12   \\
%   --     & The rest         & 15 
%   \end{tabular*}
%   \vspace{3mm}
%   \caption{XXX \TODO.}
%   \label{tab:h:sidedoor_code}
%   \vspace{3mm}
% \end{margintable}

% \begin{margintable}
%   % \vspace{\abovecaptionskip}
%   % \begin{tabular}{llr}
%   \begin{tabular*}{\textwidth}{l @{\extracolsep{\fill}} lr}
%   Contains   & Explanation  & Code \\ \midrule
%   Vej        & Road       & 0    \\
%   Gade       & Street     & 1    \\
%   Alle, Allé & Avenue     & 2    \\
%   Boulevard  & Boulevard  & 3    \\
%   --         & The rest   & -1  
%   \end{tabular*}
%   \vspace{3mm}
%   \caption{XXX \TODO.}
%   \label{tab:h:road_code}
%   \vspace{3mm}
% \end{margintable}

% \begin{margintable}
%   \begin{tabular*}{\textwidth}{l @{\extracolsep{\fill}} r}
%   % \begin{tabular}{lr}
%   Energy rating label    & Code \\ \midrule
%   A2020                  & 2 \\
%   A2015 & 4  \\
%   A2010 & 6 \\
%   A2 & 8 \\
%   A1 & 10 \\
%   A &  12 \\
%   B  & 20 \\
%   C  & 30 \\
%   D  & 40 \\
%   E  & 50 \\
%   F2  & 60 \\
%   F1  & 62 \\
%   F  & 64 \\
%   G2  & 70 \\
%   G1  & 72 \\
%   G  & 74 \\
%   H, I, J, K, M  & 90 \\
%   NAN  & 100
%   \end{tabular*}
%   % \end{tabular}
%   \vspace{3mm}
%   \caption{XXX \TODO.}
%   \label{tab:h:energy_code}
%   \vspace{3mm}
% \end{margintable}

% Until now all of the analysis have dealt with different types of residences all together. From now on, the rest of the analysis will be applied on single-family houses and owner-occupied apartments independently. 

% First invalid counts are dropped, such that variables which contain more than \SI{10}{\percent} NANs are dropped, and duplicate rows are also removed. Then some manual features are added based on the day of the month, the month, and the year are extracted from the sales date and the sales date is also represented as both a float and as the numbers of days since January \nth{1}, \num{2009}. From the number of the house, \code{HusNr}, the number is extracted along with a boolean flag indicating whether or not it includes a letter (eg. \q{27B}). The number of the side door, \code{SidedoerNr}, is formatted according to Table~\ref{tab:h:sidedoor_code} and the road name is according to  Table~\ref{tab:h:road_code}. The age of the house is added\sidenote{In addition to only having the year the house was was built} and the amount of time (in year) since last major modification. The energy rating label, \code{EnergiMaerke}, is also converted from strings to values according to  Table~\ref{tab:h:energy_code} 

% Finally, some of the variables in the dataset are not suitable for machine learning or simply transformations of other varibles. These are variables such as the ID, when the house was deleted at Boligsiden, or the cash price, since we want the model to learn the price of the house and not simply the efficiency of the realtor.

% \subsection{Time-Dependent Price Index}

% In addition to the manual data augmentation in \autoref{sec:h:feature_augmentation}, a time-dependent price index is also added. We make use of the open source package called Prophet made by \citet{taylorForecastingScale} at Facebook. It is based on a decomposable time series model \autocite{harveyEstimationProceduresStructural1990} with two\sidenote{In their paper, \citet{taylorForecastingScale} include a holiday component in their analysis as well which is not included in this project.} components; trend $g(t)$ and seasonality $s(t)$:
% \begin{equation}
%   y_\mathrm{Prophet}(t) = g(t) + s(t) + \epsilon_t,
% \end{equation}
% where $\epsilon_t$ is a normally distributed error term. \citet{taylorForecastingScale} fit this equation with a generalized additive model (GAM) \autocite{hastieGeneralizedAdditiveModels1987} which they argue has several practical advantages compared to ARIMA models\sidenote{AutoRegressive Integrated Moving Average}. 

% We fit the Prophet model on the weekly median price pr. square meter (PPSM) up until (and including) \num{2017}. The results of the prophet model fitted on OOAs are seen in Figure~\ref{fig:h:prophet_forecast} and Figure~\ref{fig:h:prophet_trends}. In Figure~\ref{fig:h:prophet_forecast} the weakly median price pr. square meter for OOAs are shown as black dots with the fitted Prophet model shown in blue. The $1\sigma$ uncertainty intervals are shown as the transparent blue band. The Prophet model not only allows predicting previous and future PPSMs, it also return the uncertainty of this prediction. The future predictions for the PPSM are the values after 2018. The trend and seasonality of the model are shown in Figure~\ref{fig:h:prophet_trends} where the top plot is the overall trend $g(t)$ and the bottom plot is the seasonality $s(t)$. Whereas the trend just continues to rise, the seasonality shows that residences are generally sold for a higher price in the Summer months compared to the Winter months. 

% \begin{figure}
%   \includegraphics[draft, width=0.9\textwidth, trim=15 15 15 15, clip]{figures/housing/Ejerlejlighed_v18_cut_all_Ncols_all_prophet_forecast.png}
%   \caption[Prophet Forecast for apartments]
%           {The predictions of the Facebook Prophet model trained on square meter prices for owner-occupied apartments sold before January 1st, 2018. The data is down-sampled to weekly bins where the median of each week is used as in input to the Prophet model. This can be seen as black dots in the figure. The \textcolor{blue}{model's forecasts} for 2018 and 2019 are shown in blue with a light blue \textcolor{light-blue}{error band} showing the \num{1}-$\sigma$ confidence interval.
%           }
%   \label{fig:h:prophet_forecast}
% \end{figure}

% \begin{figure}
%   \includegraphics[draft, width=0.9\textwidth, trim=15 15 15 15, clip]{figures/housing/Ejerlejlighed_v18_cut_all_Ncols_all_prophet_trends.pdf}
%   \caption[Prophet Trends]
%           {The trends of the Facebook Prophet model trained on square meter prices for owner-occupied apartments sold before January 1st, 2018. In the top plot is the overall trend as a function of year and in the bottom plot is the yearly variation as a function of day of year. It can be seen that the square meter price is higher during the Summer months compared to the Winter months, however, compared to the overall trend this effect is minor ($<10\%$). 
%           }
%   \label{fig:h:prophet_trends}
% \end{figure}

% The Prophet model plots for OFHs can be seen in Figure~\ref{fig:h:prophet_forecast_villa} and Figure~\ref{fig:h:prophet_trends_villa} in the appendix. 

% Using the Prophet model, we define the price index (PI) to be the Prophet-predicted PPSM, $y_\mathrm{Prophet}$, for each residence normalized by the mean to give values around \num{1}:
% \begin{equation}
%   % \mathrm{PI}(t) = \frac{y_\mathrm{Prophet}(t)}{\mathrm{mean}(y_\mathrm{Prophet}(t))}.
%   \mathrm{PI}(t) = \frac{y_\mathrm{Prophet}(t)}{\langle y_\mathrm{Prophet}(t) \rangle},
% \end{equation}
% where $\langle \boldsymbol{\cdot} \rangle$ refers to the average. The price index thus works as a measure of the national price for houses or apartments at a given time and is added as a variable to the dataset. 

% % \begin{figure}
% %   \includegraphics[width=0.9\textwidth]{figures/housing/tinglysning_fig.pdf}
% %   \caption[Registration of property]
% %           {Overview of registration of property as a function of amount of owners (\code{AntEjere}), lowest share (\code{MindsteAndel}) and biggest share (\code{StoersteAndel}) written as \code{[AntEjere, MindsteAndel, StoersteAndel]}.
% %           }
% %   \label{fig:h:tinglysning}
% % \end{figure}

% \section{Evaluation Function}

% The choice of evaluation function $f_\mathrm{eval}$ is an important decision. The evaluation function will be based on the relative prediction $z$: 
% \begin{equation}
%   z = \frac{\hat{y}-y}{y},
% \end{equation}
% where $y$ is true price and $\hat{y}$ the predicted one. The relative prediction is defined such that it is positive when $\hat{y}>y$, due to the outlier cuts made earlier the denominator is made sure to always be positive (and never \num{0}), and $z$ is expected to approximately follow a normal distribution. Initially the mean of $z$ was considered as the choice of $f_\mathrm{eval}$ though this only ensures a minimum of bias, not necessarily a low spread. This lead the discussion on to look at the standard deviation of $z$ as $f_\mathrm{eval}$. The mean and the standard deviation, however, are not a very robust estimators since they are heavily influenced by outliers. The mean (and thus also the standard deviation) has a \emph{asymptotic breakdown point} at \SI{0}{\percent}, where the breakdown point is defined as the smallest fraction\sidenote{Where the \q{asymptotic} in \q{asymptotic breakdown point} refers to when the number of samples goes to infinity.} of bad observations that can cause an estimator to become arbitrarily small or large: a single outlier with an arbitrarily large value may cause the mean to diverge to that large value \autocite{huber2011robust}. In comparison, the median has an asymptotic breakdown point of \SI{50}{\percent} and is thus a more robust estimator of centrality. A robust measure of the variability or dispersion of a sample $\vec{x}$ -- compared to e.g. the standard deviation $\sigma$ -- is the median absolute deviation (MAD) written as:
% \begin{equation}
%   \mathrm{MAD}(\vec{x}) = c \cdot \mathrm{median} \left( \abs{\vec{x} - \mathrm{median}(\vec{x})} \right), \quad c = \frac{1}{\Phi^{-1}(\frac{3}{4})},
%   \label{eq:h:MAD}
% \end{equation}
% where $c$ is a normalization constant to make MAD a consistent estimator of the standard deviation $\sigma$ assuming normally distributed data and $\Phi^{-1}$ is the percent point function\sidenote{Inverse of the cumulative distribution function.} \autocite{leysDetectingOutliersNot2013}.
% The MAD is thus the median of the absolute differences between the data and the median of the data. We are, however, not just interested in having the distribution of the (relative) predictions as narrow as possible, we also want it centered around \num{0}. We thus continue with the following evaluation function:
% \begin{equation}
%   \begin{split}
%     \mathrm{MAD}_0(\vec{x})  &\equiv c \cdot  \mathrm{median} \left( \abs{\vec{x} - 0} \right) = c \cdot \mathrm{median} \left( \abs{\vec{x}} \right) \\
%     f_\mathrm{eval}(\vec{z}) &\equiv \mathrm{MAD}_0(\vec{z}) = c \cdot  \mathrm{median} \left( \abs{\vec{z}} \right).
%   \end{split}
% \end{equation}
% Here $\vec{z} \in \mathbb{R}^N$ is the vector of all relative price predictions. To get an intuition about the size of a \q{good} value of $\mathrm{MAD}_0$, one could calculate it comparing the asking price with the actual sales price. Doing so, one finds: $f_\mathrm{eval}(\vec{z}_\mathrm{OFH}) = \SI{11.35}{\percent}$ for houses and $f_\mathrm{eval}(\vec{z}_\mathrm{OOA}) = \SI{5.72}{\percent}$ for apartments. In some cases $f_\mathrm{eval}$ will still be referred to as MAD, however, it will be mentioned explicatly if the form in equation \eqref{eq:h:MAD} is meant. 

% \marginnote[-2cm]{The MAD is assuming symmetric distributions and thus non-symmetric robust measures of the variability of a sample have been developed. See \citet{rousseeuwAlternativesMedianAbsolute1993} for more details.}


% \section{Initial Hyperparameter Optimization}

% With the initial cleaning and feature adding done, the shapes of the ML-ready datasets are: (\num{291317}, \num{144}) for houses (OFHs) and (\num{114166}, \num{144}) for apartments (OOA), both sharing the same variables. All of the variables which are used from this point on can be seen in Table~\ref{tab:h:all_variables} in the appendix. 
% The data are split intro training and test sets such that training is defined as every sale from before \num{2018}, every sale from \num{2018} is the test set, and since more data came since the project started, \num{2019} is a small extra test set. 

% \begin{margintable}
%   \begin{tabular}{lrr}
%               & Houses       & Apartments   \\ \midrule
%    Train      & \num{240070} & \num{93115}  \\   
%    Test       & \num{34628}  & \num{14183}  \\   
%    \num{2019} & \num{16619}  & \num{6868} 
%   \end{tabular}
%   \vspace{3mm}
%   \caption{\label{tab:h:train_test_split}train test split XXX \TODO.}
%   \vspace{3mm}
% \end{margintable}

% \begin{margintable}
%   \begin{tabular}{lrr}
% % \begin{table}
%   % \begin{tabular}{lll}
%    Tight      & Houses        & Apartments  \\ \midrule
%    Train      & \num{143179}  & \num{57795} \\   
%    Test       & \num{20338}   & \num{8376}  \\   
%    \num{2019} & \num{9683}    & \num{4030} 
%   \end{tabular}
%   \vspace{3mm}
%   \caption{\label{tab:h:train_test_split_tight}train test split tight XXX \TODO.}
%   \vspace{3mm}
% \end{margintable}

% The number of observations for the different sets can be seen in Table~\ref{tab:h:train_test_split}. Since the dataset has been shown to be quite noisy with a lot invalid counts, a \q{tight} selection of the data is also applied. The tight selection is defined as residences which are within the \SI{1}{\percent} to \SI{99}{\percent} quantiles of all\sidenote{Except the variables that contains the words: \q{aar} (year), \q{dato} (date), and  \q{prophet}.} numeric variables with more than 3 unique values. The number of observations for the different tight sets can be seen in Table~\ref{tab:h:train_test_split_tight}. 

% Before the data is properly fitted and the model is trained, a small study into the effect of some various hyperparameters was performed. This study investigated the effect of the old sales by assigning them a lower weight depending on time. It was investigated whether or not the model would perform better if samples got the time-dependent weight $w(t)$ given by:
% \begin{equation}
%   \begin{split}
%     w'(t) &= e^{ k \cdot t}, \quad k = \frac{\log 2}{T_{\frac{1}{2}}} \\
%     w(t) &= \frac{w'(t)}{\langle w'(t) \rangle}
%   \end{split}
% \end{equation}
% % TODO: add parts about what weights are.
% where $T_{\frac{1}{2}}$ is the half-life. 

% \begin{marginfigure}
%   \includegraphics[width=0.99\textwidth]{figures/housing/Villa_v18_cut_all_Ncols_all_half_life_weights.pdf}
%   \caption[XXX ]
%     {Time here is years after January \nth{1}, 2009. XXX \TODO. 
%     }
%   \label{fig:h:half-life}
% \end{marginfigure}

% This is illustrated in Figure~\ref{fig:h:half-life} for the different values of $T_{\frac{1}{2}}$ used in the study.  In addition to the weight, it was also investigated whether or not a $\log_{10}$ transformation of the price would increase performance. The reasoning behind this would be that some machine learning methods assume that the dependent variable, $y$, is normally distributed. Finally the choice of loss function was also added to the study for the five different loss functions defined in \autoref{sec:ml:loss_function}. A grid search\sidenote{Grid search was acceptable since the parameter space is small and two of the three dimensions is non-numerical.} was performed for:
% \begin{equation}
%   \begin{split}
%     T_{\frac{1}{2}} &\in \{2.5,\, 5,\, 10,\, 20,\, \infty \}~\mathrm{ years} \\
%     \log_{10} &\in \{\mathrm{True},\, \mathrm{False} \} \\
%     \ell &\in \{ \ell_\mathrm{Cauchy},\, \ell_\mathrm{Fair},\, \ell_\mathrm{LogCosh},\, \ell_\mathrm{SE},\, \ell_\mathrm{Welsch}\}
%   \end{split}
% \end{equation}

% % \begin{margintable}
% %   \begin{tabular}{@{}ccrc@{}}
% %     Half-life & $\log_{10}$ & $N_\mathrm{trees}$ & $f_\mathrm{eval}$ \\
% %     \midrule
% %     \num{2.5} & True & \num{393} & \num{0.1609} \\
% %     \num{2.5} & False & \num{887} & \num{0.1449} \\
% %     \num{5} & True & \num{296} & \num{0.1599} \\
% %     \num{5} & False & \num{776} & \num{0.1476} \\
% %     \num{10} & True & \num{266} & \num{0.1611} \\
% %     \num{10} & False & \num{822} & \num{0.1457} \\
% %     \num{20} & True & \num{257} & \num{0.1614} \\
% %     $\mathbf{20}$ & \textbf{False} & $\mathbf{932}$ & $\mathbf{0.1439}$ \\
% %     $\infty$ & True & \num{418} & \num{0.1567} \\
% %     $\infty$ & False & \num{1015} & \num{0.1468} 
% %   \end{tabular}
% %   \vspace{2mm}
% %   \caption{\label{tab:h:HPO_initial_Cauchy}Cauchy. Ejerlejlighed XX}
% % \end{margintable}


% \begin{margintable}
%   \begin{tabular}{@{}ccrc@{}}
%     %\toprule
%     Half-life & $\log_{10}$ & $N_\mathrm{trees}$ & $f_\mathrm{eval}$ \\
%     \midrule
%     \num{2.5} & True & \num{293} & \num{0.1598} \\
%     \num{2.5} & False & \num{814} & \num{0.1466} \\
%     \num{5} & True & \num{304} & \num{0.1610} \\
%     \num{5} & False & \num{923} & \num{0.1468} \\
%     \num{10} & True & \num{266} & \num{0.1610} \\
%     $\mathbf{10}$ & \textbf{False} & $\mathbf{770}$ & $\mathbf{0.1450}$ \\
%     \num{20} & True & \num{288} & \num{0.1613} \\
%     \num{20} & False & \num{967} & \num{0.1467} \\
%     $\infty$ & True & \num{340} & \num{0.1601} \\
%     $\infty$ & False & \num{807} & \num{0.1480} \\
%     %\bottomrule
%   \end{tabular}
%   \caption{\label{tab:h:HPO_initial_Cauchy-ejerlejlighed}Cauchy-ejerlejlighed.}
% \end{margintable}



% \begin{margintable}
%   \begin{tabular}{@{}ccrc@{}}
%     %\toprule
%     Half-life & $\log_{10}$ & $N_\mathrm{trees}$ & $f_\mathrm{eval}$ \\
%     \midrule
%     \num{2.5} & True & \num{434} & \num{0.1991} \\
%     \num{2.5} & False & \num{1007} & \num{0.1872} \\
%     \num{5} & True & \num{350} & \num{0.1999} \\
%     \num{5} & False & \num{1130} & \num{0.1858} \\
%     \num{10} & True & \num{436} & \num{0.1992} \\
%     \num{10} & False & \num{1183} & \num{0.1850} \\
%     \num{20} & True & \num{397} & \num{0.2003} \\
%     $\mathbf{20}$ & \textbf{False} & $\mathbf{1514}$ & $\mathbf{0.1833}$ \\
%     $\infty$ & True & \num{449} & \num{0.1992} \\
%     $\infty$ & False & \num{1351} & \num{0.1844} \\
%     %\bottomrule
%   \end{tabular}
%   \caption{\label{tab:h:HPO_initial_Cauchy-villa}Cauchy-villa.}
% \end{margintable}




% The GS is run on the training set with \num{5}-fold CV, early stopping is applied with a patience of \num{100}, and XGBoost \autocite{chenXGBoostScalableTree2016} (XGB) is used as the ML model. For apartments the loss function with the lowest $f_\mathrm{eval}$ was the Cauchy loss with $T_{\frac{1}{2}}=20$ years and no $\log_{10}$ transformation. This BDT terminated by early stopping after \num{932} trees, see Table~\ref{tab:h:HPO_initial_Cauchy-ejerlejlighed}. 
% For houses the loss function with the lowest $f_\mathrm{eval}$ was the Fair loss (also) with $T_{\frac{1}{2}}=20$ years and (also) no $\log_{10}$ transformation. This BDT terminated by early stopping after \num{1996} trees, see Table~\ref{tab:h:HPO_initial_Cauchy-villa}. 
% It is interesting to note that both HPO for houses and apartments choose the same half-life and not to do any transformation\sidenote{The reason why the $\log_{10}$ transformation is included even to begin with, was that initially it should better results than no transformation. However, this turned out to be a numerical consequence which was alleviated by dividing all prices with a million, such that $y$ had units of \si{\Mkr} instead of \si{\kr}.}. All of the results for the apartments can be seen in Table~\ref{tab:h:HPO_initial_Rmse-ejerlejlighed-appendix}--\ref{tab:h:HPO_initial_Fair-ejerlejlighed-appendix} in the appendix along with all of results for the houses in Table~\ref{tab:h:HPO_initial_Rmse-villa-appendix}--\ref{tab:h:HPO_initial_Fair-villa-appendix}. 

% A visualization of the HPO results can be seen as the parallel coordinate plot in Figure~\ref{fig:h:initial_CV_res_parallel_coords_ejer}. Here the hyperparameters (along the time taken and the number of trees) of the HPO are plotted along the abscissa (x-axis) and the value of the hyperparameter on the ordinate (y-axis). Every iteration of the HPO is thus a line on the plot. The lines are colored according to their evaluation score; the best hyperparameter is shown in red. For the hyperparamater \code{log10} \code{0} means False and \code{1} means True, for \code{Halftime} $\infty$ is mapped to \code{30}, and for \code{objektive} the functions Cauchy (0), Fair (1), LogCosh (2) SquaredError (3), and Welsch (4) are mapped to the integers in the parentheses. The same plot for houses can be seen in Figure~\ref{fig:h:initial_CV_res_parallel_coords_villa} in the appendix.

% \begin{figure}
%   \includegraphics[width=0.95\textwidth, trim=0 0 0 0, clip]{figures/housing/Ejerlejlighed_v19_cut_all_Ncols_all_CV_viz_initial_HPO.pdf}
%   \caption[Overview of initial hyperparamater optimization of the housing model for apartments]
%           {Hyperparameter optimization results of the housing model for apartments. The results are shown as parallel coordinates with each hyperparameter along the x-axis and the value of that parameter on the y-axis. Each line is an event in the 4-dimensional space colored according to the performance of that hyperparameter as measured by $\mathrm{MAD}_0$ from \textcolor{viridis-dark}{highest} MAD in dark blue to \textcolor{viridis-light}{lowest} AUC in yellow. The \textcolor{red}{single best hyperparameter} is shown in red. For the hyperparamater \code{log10} \code{0} means False and \code{1} means True, for \code{Halftime} $\infty$ is mapped to \code{30}, and for \code{objective} the functions Cauchy (0), Fair (1), LogCosh (2) SquaredError (3), and Welsch (4) are mapped to the integers in the parentheses.
%           } 
%   \label{fig:h:initial_CV_res_parallel_coords_ejer}
% \end{figure}

% What can be concluded from Figure~\ref{fig:h:initial_CV_res_parallel_coords_ejer} is that there is a clear preference from not $\log$-transforming the data, that the BDTs with many trees\sidenote{Remember that the number of trees where selected by early stopping.} generally performed better than the ones with fewer trees, that it is difficult to see a clear pattern for the half-life, and that there seems to be a tendency for the Cauchy loss to be the best, however, it is still ambiguous. What is not seen in the figure, however, are how the uncertainties\sidenote{The uncertainties here are the standard deviation (not of the mean) of the \num{5} folds in the cross validation.} of the different iterations also matter. 
% \begin{marginfigure}[1cm]
%   \centerfloat
%   \includegraphics[width=0.95\textwidth, trim=0 0 0 0, clip]{figures/housing/Ejerlejlighed_v19_cut_all_Ncols_all_MAD_gridsearch_half.pdf}
%   \caption[XXX]
%           {XXX Halflife $T_{\frac{1}{2}}$.
%           }
%   \label{fig:h:hpo_gridsearch_objective}
% \end{marginfigure}

% \begin{marginfigure}[1cm]
%   \centerfloat
%   \includegraphics[width=0.95\textwidth, trim=0 0 0 0, clip]{figures/housing/Ejerlejlighed_v19_cut_all_Ncols_all_MAD_gridsearch_obj.pdf}
%   \caption[XXX]
%           {Objective function XXX.
%           } 
%   \label{fig:h:hpo_gridsearch_halflife}
% \end{marginfigure}
% % \begin{figure}
% %   \centerfloat
% %   % \vspace*{-\abovecaptionskip}
% %   \subfloat[Objective \label{fig:h:hpo_gridsearch_objective}]{\qquad}
% %   \includegraphics[width=0.4\textwidth]{figures/housing/Ejerlejlighed_v18_cut_all_Ncols_all_MAD_gridsearch_obj.pdf}\hfil
% %   \subfloat[\label{fig:h:hpo_gridsearch_halflife} $T_{\frac{1}{2}}$]{\qquad}
% %   \includegraphics[width=0.4\textwidth]{figures/housing/Ejerlejlighed_v18_cut_all_Ncols_all_MAD_gridsearch_half.pdf}
% %   \caption[X]{XXX. 
% %            In subplot ~\protect\subref{fig:h:hpo_gridsearch_objective} XXX and in subplot ~\protect\subref{fig:h:hpo_gridsearch_halflife} XXX. 
% %            }
% %   \label{fig:h:hpo_gridsearch}
% %   \vspace{\abovecaptionskip}
% % % \end{figure}
% For the half-life and the choice of objective function these can be seen in Figure~\ref{fig:h:hpo_gridsearch_objective} and \ref{fig:h:hpo_gridsearch_halflife}. In the first of the two figures it is easily seen that even though $T_{\frac{1}{2}}=\SI{10}{\yr}$ is the minimum, the uncertainties are so large that nothing can be concluded with regards to the half-life parameter related to the weights $w(t)$. In contrary, in the second figure there is a clear performance difference between the different loss functions where the Cauchy loss archives the best (lowest) value of $f_\mathrm{eval}$ and especially the Squared Error is disregarded. 

% The rest of the machine learning models thus continue with the following hyperparamaters:

% \begin{table}[h]
%   \centerfloat
%   \begin{tabular}{@{}llcl@{}}
%                & $\log_{10}$  & Half-life & Loss function \\ \midrule
%   Apartments   & False & \SI{20}{\yr} & Cauchy \\
%   Houses       & False & \SI{20}{\yr} & Cauchy
%   \end{tabular}
%   % \vspace{\abovecaptionskip}
%   % \caption{XXX}
%   \label{tab:h:initial_hpo}
% \end{table}



% \FloatBarrier
% \section{Hyperparameter Optimization}
% With the initial HPO set, the actual training of the models was started. An XGBoost model was fitted to the each of the two training sets, one for apartments and one for houses, where the hyperparameters were optimized using both random search and Bayesian optimization each run for \num{100} iteratations with \num{5}-fold cross validation and early stopping\sidenote{This takes around 6 hours for both RS and GS for the apartments when run with 15 cores on HEP. XXX}. The hyperparameters to be optimzed where the following:
% \begin{itemize}
%   \item[] The \code{subsample} varible controls the row-subsampling and is a number between \num{0} and \num{1}. 
%   \item[] The hyperparameter \code{colsample_bytree} controls the column-downsampling for each tree, so how many columns (or variables) are each tree allowed to fit to. Is a number between \num{0} and \num{1}.
%   \item[] The \code{max_depth} controls the maximum depth of every tree. Is a positive integer.  
%   \item[] The \code{min_child_weight} variable controls when the decision tree algorithm should stop splitting a node into further nodes (and will thus turn it into a leaf). Is a positive integer
%   \item[] \code{reg_lambda} controls the L2 regularization term. Is a positive number. 
%   \item[] \code{reg_alpha} controls the L1 regularization term. Is a positive number.
% \end{itemize}

% The ranges of the hyperparameters were chosen by a manual, iterative process of fitting a subset of the data (\SI{1}{\percent}--\SI{10}{\percent}) and making sure that the best hyperparameter is not sufficiently close to the range; if it is, then the range is extended. The final HPO ranges chosen can be seen in Table~\ref{tab:h:hpo_ranges}. Here $\mathcal{U}(a, b)$ refers to a uniform distribution from $a$ to $b$, and $\mathcal{U}_\mathrm{int}(a, b)$ is the same only an integer distribution. 
% \begin{margintable}
%   \centerfloat
%   \begin{tabular}{@{}ll@{}}
%   Hyperparameter          &  Range                      \\ \midrule
%   \code{subsample}        & $\mathcal{U}(0.5, 0.9)$           \\
%   \code{colsample_bytree} & $\mathcal{U}(0.1, 0.99)$           \\
%   \code{max_depth}        & $\mathcal{U}_\mathrm{int}(1, 20)$ \\
%   \code{min_child_weight} & $\mathcal{U}_\mathrm{int}(1, 30)$ \\
%   \code{reg_lambda}       & $\mathcal{U}(0.1, 4)$  \\
%   \code{reg_alpha}        & $\mathcal{U}(0.1, 4)$
%   \end{tabular}
%   % \vspace{\abovecaptionskip}
%   \vspace{3mm}
%   \caption{\label{tab:h:hpo_ranges} XXX}
% \end{margintable}

% The fitting pipeline for this subproject is to first run both RS and BO as HPOs to compare their results. The best of the two models are chosen and used afterwards where the learning rate $\eta$ is reduced from $\eta=0.1$ to $\eta=0.01$  and is finally optimised by early stopping. In the end one ends up with a model that has been HPO optimized for prepocessing optimizations ($\log$-transformations), loss function, sample weights, normal XGBoost parameters and finally the learning rate. This pipeline has been manually implemented in Python since no other packagages provide the same flexibility as a custom implementation that works fully automatically. 

% The results of the RS and BO can be seen in Figure~\ref{fig:h:CV_res_RS_parallel_coords_ejer} and \ref{fig:h:CV_res_BO_parallel_coords_ejer} (in the appendix). The correpsonding plots for houses can be seen in Figure~\ref{fig:h:CV_res_RS_parallel_coords_villa} and \ref{fig:h:CV_res_BO_parallel_coords_villa} in the appendix. 

% \begin{figure}
%   \includegraphics[width=0.95\textwidth, trim=0 0 0 0, clip]{figures/housing/Ejerlejlighed_v19_cut_all_Ncols_all_CV_viz_HPO_RS.pdf}
%   \caption[XXX]
%           {Hyperparameter optimization results of XGBoost parameters of the housing model for apartments shown as parallel coordinates. Here shown for random search as hyperparameter optimization.
%           } 
%   \label{fig:h:CV_res_RS_parallel_coords_ejer}
% \end{figure}

% As with the initial HPO, it is important to compare the results in Figure~\ref{fig:h:CV_res_RS_parallel_coords_ejer} with their uncertainties. This can be seen in Figure~\ref{fig:h:CV_res_RS_uncertainties_ejer}. Here the value of the evaluation function along with its \num{1}$\sigma$  and \num{2}$\sigma$ uncertainties are plotted as a function of the iteration along the abscissa. The minimum value of $f_\mathrm{eval}$ is shown in red. Even though this is the minumum value, notice how flat of a minimum this is: most of the other iterations are within \num{1}$\sigma$. The plot for BO in Figure~\ref{fig:h:CV_res_BO_uncertainties_ejer} shows the similar characteristic and does not show any improvement over time as was otherwise expected for BO compared to RS. 
% % XXX add villas

% \begin{figure}
%   \includegraphics[width=0.95\textwidth, trim=0 0 0 0, clip]{figures/housing/Ejerlejlighed_v19_cut_all_Ncols_all_xgb_score_over_time_random.pdf}
%   \caption[Hyperparameter optimization: random search results]
%           {The results of running random search (RS) as hyperparameter optimization (HPO) on apartments using the XGB-model. The \textcolor{red}{minimum (mean) loss} along with its uncertainty is shown in red, the \textcolor{blue}{means} for the different iterations of RS in blue, and as light blue bands are the \textcolor{blue}{one (and two) standard deviation(s) of the means (SDOM)}, all as a a function of iteration number.} 
%   \label{fig:h:CV_res_RS_uncertainties_ejer}
% \end{figure}

% The best of the RS and GS models are chosen for subsequent analysis by first reducing the learning rate to the $\eta=0.01$ and then find the best number of estimators by early stopping. The evaluation function as a function of number of estimators, also known as the \emph{learning curve}, is seen in Figure~\ref{fig:h:CV_res_ES_learning_curve_ejer}. This curve is the realisation of Figure~\ref{fig:ml:empirical_risk} in real data, where it first improves a lot and then finds a stable plateau. The minimum is shown in red with its uncertainty. To reduce the risk of overfitting and model complexity -- with the further advantage of resulting in faster model at predition time -- we keep the model with the lowest number of estimators that are still within $1\sigma$ of the minimimum: see the orange point in the figure. This results in a model that contains less than a fifth of the number of trees and is thus also significantly simpler and faster at inference time. This is the final hyperparameter optimized model that will be used for the further analysis. 

% \begin{figure}
%   \includegraphics[draft, width=0.95\textwidth, trim=0 0 0 0, clip]{figures/housing/Ejerlejlighed_v19_cut_all_Ncols_all_xgb_early_stopping_fig.png}
%   \caption[Early Stopping results]
%           {The results of early stopping on apartments using the XGB-model. The \textcolor{red}{minimum (mean) loss} along with its uncertainty is shown in red, the \textcolor{blue}{means} for the different iterations of RS in blue, and as light blue bands are the \textcolor{blue}{one (and two) standard deviation(s)}, all as a a function of number of estimators (trees). In orange the \textcolor{orange}{\q{best} number of estimators} is shown, defined as the lowest number of estimators which are still within $1\sigma$ of the minimum value. This leads to a model that has a performance that is within $1\sigma$ of the best model, but a lot simpler and faster.} 
%   \label{fig:h:CV_res_ES_learning_curve_ejer}
% \end{figure}










% \FloatBarrier
% \section{Results}

% The performances of the different models, the RS-optimized, the BO-optimised and the best of the two $\eta$-optimized with ES, are shown in Figure~\ref{fig:h:CV_res_performance_ejer}. Here the distribution of the relative price prediction $\vec{z}$ of the model evaluated on the test set, apartments sold in \num{2018}, is shown for the three models. In addition to the distributions, also the performance metrics are shown: the value of the evaluation function\sidenote{Written as \code{MAD}} along with the fraction of $z$ that are within the specified percentage. In this particular instance it is seen that \SI{41.3}{\percent} of the predictions by the final ES model are less than $\pm\SI{5}{\percent}$ wrong, \SI{69.8}{\percent} within $\pm\SI{10}{\percent}$, and \SI{91.9}{\percent} within $\pm\SI{20}{\percent}$. Note that the difference between the three models are minor and that they distributions almost cannot be distinguished between each other. 

% \begin{figure}[h!]
%   \includegraphics[width=0.95\textwidth, trim=0 0 0 0, clip]{figures/housing/Ejerlejlighed_v19_cut_all_Ncols_all_xgb_z_hist_metrics.pdf}
%   \caption[Performance of XGB-model on apartment prices]
%           {Histogram of $z$-values of the XGB-model trained on apartments. The performance after hyperparameter optimization (HPO) using \textcolor{blue}{Random Search} (RS) is shown in blue, for \textcolor{red}{Bayesian Optimization} (BO) in red. After finding the best model, BO in this case, the model is retrained using \textcolor{green}{early stopping}, the performance of which is shown in green.
%           } 
%   \label{fig:h:CV_res_performance_ejer}
% \end{figure}

% An interesting observation from the performance metrics of the test set, is the low value of $f_\mathrm{eval}=0.9289$ (\code{MAD}). By looking at Figure~\ref{fig:h:CV_res_ES_learning_curve_ejer} one would expect a test loss of around \num{0.12} assuming iid. samples. However, this assumption does not seem to be fulfilled for the test set. The performances of the realtors are also better for the test set than the training set, and by comparing the test set with the extra 2019 set it seems to be 2019 that was an extra easy year without too many outliers, see Table~\ref{tab:h:realtor_mad_train_test_2019}. The \q{Tight} in the the table corresponds to the realtors performance on the tight version of the different datasets. The performance of the XGB models on the tight test set can be seen in Figure~\ref{fig:h:CV_res_performance_ejer_tight} in the appendix, where it can be seen that the final model has $f_\mathrm{eval}=0.8383$. 
% \begin{margintable}
%   \centerfloat
%   \begin{tabular}{@{}rlll@{}}
%           & Train               & Test                & \num{2019}          \\ \midrule
%   Normal  & \SI{5.80}{\percent} & \SI{4.97}{\percent} & \SI{6.19}{\percent} \\
%   Tight   & \SI{5.69}{\percent} & \SI{4.94}{\percent} & \SI{6.19}{\percent} 
%   \end{tabular}
%   \vspace{\abovecaptionskip}
%   \caption{XXX}
%   \label{tab:h:realtor_mad_train_test_2019}
% \end{margintable}

% To gauge the predictive power of the model over time, we applied the model to the next months data, evaluated the results for that month and continued like that for all the months in the test set (\num{2018}) and \num{2019}. We apply two different methods of forecasting: \q{static} forecasting where the model is only trained once, and \q{dynamic} forecasting where the model is retrained after each month on all of the previous sales. These predictions allows the relative predictions $\vec{z}$ to be calculated and the MAD and the standard deviation (SD) of $\vec{z}$ are shown in Figure~\ref{fig:h:forecast_MAD_SD}. 
% \begin{figure}
%   \includegraphics[width=0.9\textwidth, trim=0 0 0 0, clip]{figures/housing/Ejerlejlighed_v19_cut_all_Ncols_all_xgb_forecast_prediction_MAD.pdf}
%   \caption[2018 XGB Forecast]
%           {Performance of 1-month forecasts for apartments sold in \num{2018} and \num{2010}. For both plots the XGB model is trained on data up to (but excluding) 2018. Top) The performance of the static model's prediction for both the \textcolor{blue}{standard deviation (SD)} and \textcolor{red}{MAD} of the $z$-scores. Bottom) Same as above, however, based a dynamic model, i.e. a model which is retrained after every month to include the previous month's sales.} 
%   \label{fig:h:forecast_MAD_SD}
% \end{figure}

% In the top subplot the results for the static forecast are shown, whereas the dynamic results are shown in the bottom subplot. The errorbars are calculated using the usual variance of the standard deviation:
% \begin{equation}
%   \sigma_\mu  = \frac{\sigma}{\sqrt{N}}, \quad \sigma_\sigma = \frac{\sigma}{\sqrt{2N}},
% \end{equation}
% where $\sigma_\mu$ is the standard deviation of the mean and $\sigma_\sigma$ is the standard deviation of the standard deviation\sidenote{That this estimator is biased does not matter since we are in the large $N$ limit, $N \sim 1000$} \autocite{Barlow:0471922951}.
% Notice the large fluctuations in SD over time compared to MAD which an effect of MAD being a robust estimator. What is also interesting to note is that the MAD seems to increase over time for the static model, albeit slowly. In comparison, for the dynamic model this seem less pronounced. This figure not only shows the time dependence of the performance of the model, but also that the model is quite stable over time, at least for the dynamic model. 

% Using the relative price predictions $\vec{z}$, we construct the Market Index, $\mathrm{MI}$. This is an index which measures the overall level of the Danish housing market based on the assumption that if the houses sold in a month are generally sold at a higher price than was predicted by the model, there can be two reasons for it: either the model was wrong or the market simply changed in the time span. With the latter assumption, the ratio between the prediction and the actual price of a residence is thus a measure of the market index:
% \begin{equation}
%   \begin{split}
%     z_\mathrm{mi} &\equiv \frac{\hat{y}}{y} = 1 - \frac{\hat{y}-y}{y}  = 1-z \\
%     \mathrm{MI}_\mathrm{mean} &= \mathrm{mean}(\vec{z}_\mathrm{mi}) \\
%     \mathrm{MI}_\mathrm{median} &= \mathrm{median}(\vec{z}_\mathrm{mi}).
%   \end{split}
% \end{equation}
% Here $\vec{z}_\mathrm{mi}$ is the vector of ratios between prediction and actual price, and the market index can then be estimated using either the mean $\mathrm{MI}_\mathrm{mean}$ or the median $\mathrm{MI}_\mathrm{median}$. The market indices for every month of the forecast described in the previous figure can be seen in Figure~\ref{fig:h:forecast_MarketIndex}. In the top panel the market index for the static model is shown where it is visible for $\mathrm{MI}_\mathrm{median}$ how it cosistently overshoots. Compare this to the dynamic $\mathrm{MI}_\mathrm{median}$ whic fluctuates around \num{0}. BLABLA mere analyse her XXX. That the $\mathrm{MI}_\mathrm{mean}$ is consistently lower than $\mathrm{MI}_\mathrm{median}$ is an indication of a low tail in $z_\mathrm{mi}$ (\emph{negative skewness}) which means that some of the the predictions are much lower than the actual salesprice. BLABLA, XXX.

% \begin{figure}
%   \includegraphics[width=0.9\textwidth, trim=0 0 0 0, clip]{figures/housing/Ejerlejlighed_v19_cut_all_Ncols_all_xgb_forecast_prediction_MarketIndex.pdf}
%   \caption[2018 XGB Forecast]
%           {XXX \TODO.} 
%   \label{fig:h:forecast_MarketIndex}
% \end{figure}

% The final results for the model is seen in Table~\ref{tab:h:results_ejer} for owner-occupied apartments and in in Table~\ref{tab:h:results_villa} for one family houses. The MAD is around \SI{9}{\percent} for apartments and \SI{16}{\percent} for houses, which is still worse than the realtors' prediction, yet still acceptable for a model that does not have any variables describing the indoor conditions. For apartments around \SI{40}{\percent} of all the predicitions are within $\pm \SI{5}{\percent}$ and more than \SI{90}{\percent} are within $\pm \SI{20}{\percent}$ which is similar to the performance of the professional automated property evaluations from e.g. \citet{bolighedBolighedUsikkerhedDatavurderingen}. Also shown in the tables are the mean of the relative predictions $\mu_{\vec{z}}$ which shows that the  The performance on the tight cuts can be seen in Table~\ref{tab:h:results_ejer_tight} and \ref{tab:h:results_villa_tight} in the appendix. 


% \begin{table}
%   \centerfloat
%   \begin{tabular}{@{}lccccc@{}}
%     {} &      MAD (\%) & $\leq 5\% (\%)$ &  $\leq 10\% (\%)$ &   $\leq 20\% (\%)$ & $\mu_\vec{z}$              \\
%     \midrule
%     Train & \num{7.83} & \num{47.90} & \num{75.74} & \num{93.97} &   $+0.0128 \pm 0.0008$ \\
%     Test  & \num{9.29} & \num{41.33} & \num{69.77} & \num{91.91} &  $-0.0067 \pm 0.0012$ \\
%     2019  & \num{9.89} & \num{38.85} & \num{66.76} & \num{90.04} &  $-0.000 \pm 0.002$ 
%     \end{tabular}
%   \vspace{\abovecaptionskip}
%   \caption{XXX ejer}
%   \label{tab:h:results_ejer}
% \end{table}


% \begin{table}
%   \centerfloat
%   \begin{tabular}{@{}lccccc@{}}
%     {} &      MAD (\%) & $\leq 5\% (\%)$ &  $\leq 10\% (\%)$ &   $\leq 20\% (\%)$ & $\mu_\vec{z}$              \\
%     \midrule
%     Train & \num{14.12} & \num{28.95} & \num{51.89} & \num{78.04} &  $0.0449 \pm 0.0007$ \\
%     Test  & \num{15.79} & \num{25.61} & \num{47.52} & \num{75.14} &  $0.0310 \pm 0.0016$ \\
%     2019  & \num{16.50} & \num{24.01} & \num{45.89} & \num{74.22} &  $0.031 \pm 0.002$ 
%     \end{tabular}
%   \vspace{\abovecaptionskip}
%   \caption{XXX villa}
%   \label{tab:h:results_villa}
% \end{table}


% \FloatBarrier
% \section{Model Inspection}

% One of the most important aspects of applying advanced machine learning methods, in addition to getting accurate predictions, is understanding the model. As mentioned in \autoref{sec:ml:feature_importance}, it is possible to use SHAP values to inspect the trained model for both local predictions and for global feature importances $\phi_i^\mathrm{tot}$. An example of how to use SHAP to better understand a local prediction is seen in Figure~\ref{fig:h:shap_single_apartment}. Here the SHAP values for a particular sale are visualized as a bar chart where the green colors are positive values, red values negative values and the blue is the final prediction $\hat{y}$. Remember that for SHAP values the prediction is the sum of all of the SHAP values, see equation \eqref{eq:ml:additive_feature_attribution_method}. Here $\phi_0$ is the expectation value denoted as \code{Bias} in the plot. To show all \num{143} variables would make the figure excessively large, so two extra bins have been added: the \code{Overflow} bar which is the sum of the remaining positive SHAP values and likewise with \code{Underflow} for negative values. The sum of all the green and red bars adds up to $\hat{y}=\SI{6.86}{\Mkr}$ in this particular instance and the actual sold value was $y=\SI{6.35}{\Mkr}$. Thus, in cases where there is a large discrepancy between the predicted and actual sales prices, one use can use this tool to better understand why the prediction was estimated as it was. 

% \begin{figure}[ht!]
%   \includegraphics[width=0.95\textwidth, trim=0 0 0 40, clip]{figures/housing/Ejerlejlighed_v19_cut_all_Ncols_all_SHAP_fig_loc=489266.pdf}
%   \caption[SHAP Prediction Explanation for apartment]
%           {Model explanation for XGB model for a specific apartment. The bars are the variables in the dataset that the model found most important sorted after their importance for this particular apartment. The bias bar refers to the expected value of the model, which is simply the mean of the training set which acts as the naive prediction baseline. The \q{cutoff positive (negative)} bars are the sum of the remaining positive (negative) values that are not shown. On the right hand side of the plot is the model prediction shown. The model prediction is the sum of all of the bars in the left par (\SI{6.86}{\Mkr} in this example). The \textcolor{red}{negative} values are shown in red, \textcolor{green}{positive} ones in green, and the \textcolor{blue}{prediction value} in blue. 
%           } 
%   \label{fig:h:shap_single_apartment}
% \end{figure}

% To get an overview of which variables are most important on a global\sidenote{\q{Global} here meaning for the entire dataset.} scale, $\phi_i^\mathrm{tot}$, see Figure~\ref{fig:h:shap_overview}. Here the variables are sorted according to the normalised (i.e. summing to one) $\phi_i^\mathrm{tot}$ which is shown in parentheses after each variable name. In the center of the plot is shown a dot-plot of the dataset plotted with the SHAP value on the abscissa and colored according to the feature value. 
% The way to interpret this plot is as follows. Take a variable of interest, e.g. the area of the apartment \code{ArealBolig} with $\phi_\mathrm{ArealBolig}^\mathrm{tot}=\SI{5.35}{\percent}$. Each dot is a sample in the dataset plotted as a function of their SHAP value with a spread such that the height corresponds to the SHAP-distribution of that specific feature. For the area it can be seen that there is a long tail towards high SHAP-values, however, most of the samples have a slightly negative SHAP value. The dots are colored according to their feature value and it can thus be seen that large apartments (red) are given a higher SHAP value than small apartments; precisely as expected from the model. In contrary, when the the total days on market (DOM) (\code{LiggetidSamlet}) is large it pushes the prediction in the negative direction. 

% \begin{figure}[ht!]
%   \includegraphics[width=0.8\textwidth, trim=0 0 0 1, clip]{figures/housing/Ejerlejlighed_v19_cut_all_Ncols_all_xgb_tight_SHAP_vals_summary.pdf}
%   \caption[Feature importance of apartments prices using XGB]
%           {Feature importance of apartment prices using the XGB-model. The feature importance is measured using SHAP values. The variables are sorted top to bottom according to their overall feature importance, i.e. the previous public property valuation \code{EjendomdsVaerdi0} is the most important single feature. Along the x-axis is the impact on model output, in this example the price in XXX. This axes is colored by the value of the feature, from \textcolor{blue}{low} (blue) to \textcolor{red}{high} (red). In this particular example we see that high values of the previous public property valuation has high, positive impact on the model prediction -- exactly as expected. This is exactly opposite the total days on market (DOM) described by the variable \code{LiggetidSamlet} where a high value has a negative impact.  
%           } 
%   \label{fig:h:shap_overview}
% \end{figure}

% The SHAP-package\citep{lundbergConsistentIndividualizedFeature2018} not only allow for 1D dependences to be gauged, it also allows for so-called \emph{interaction plots}. These plots shows the 2D-dependece between the variable and the SHAP value colored according to a second variable. Since the previous public property valuation (PPPV) (\code{EjendomdsVaerdi0}) is the most important of the features, the interaction plot of this variable is seen in Figure~\ref{fig:h:shap_overview_interaction}. Here the SHAP value of the \code{EjendomdsVaerdi0} is plotted as a function of \code{EjendomdsVaerdi0}. Ignoring the colors for a second, this plot shows that the higher the PPPV, the higher the model output. The colors on the other hand shows how this trend depends on time by the variable \code{SalgsDato_siden0} which is the number of days since January \nth{1} 2009 that the apartment was. The plot shows that if the apartment has a high PPPV then the newer sales has an even higher SHAP value then older sales, agreeing with the fact that the market has gone up since 2009. On the other hand, for low PPPV apartments the relationship with time is inverse, however, the effect is much smaller here. The SHAP-package chose to color by \code{SalgsDato_siden0} since this is the variable which explains most of the variation for a given PPPV. 

% \begin{figure}[ht!]
%   \centerfloat
%   \includegraphics[draft, width=0.98\textwidth, trim=10 10 100 40, clip, page=1]{figures/housing/Ejerlejlighed_v19_cut_all_Ncols_all_xgb_tight_SHAP_vals_interaction.pdf}
%   \caption[Feature importance of apartments prices using XGB XXX]
%           {Feature importance of apartment prices using the XGB-model. XXX
%           } 
%   \label{fig:h:shap_overview_interaction}
% \end{figure}

% \FloatBarrier
% \section{Multiple Models}

% In addition to the XGBoost \autocite{chenXGBoostScalableTree2016} (XGB) BDT model, several other models were also tested. During the sub-project the LightGBM \autocite{keLightGBMHighlyEfficient2017} (LGB), also a BDT model, was released and started gaining traction in the ML community, especially for large-scale data analysis (Big Data). In comparison to XGBoost, LightGBM implements some extra binning and categorical assumptions that greatly speeds up the fitting process. It was trained in the same was as the XGBoost model with the same HPO-process and range for the hyperparameters. 

% To compliment the BDTs models, a simple linear model (LIN) with $L_2$ loss, so-called \emph{ridge} regression, was fitted where all of the NaNs were median-imputed\sidenote{Which means that all invalid values were replaced with the median along each column-} and the input features were scaled\sidenote{Scaling of input features is generally an important preprocessing step for non-tree based ML models.} with a robust scaler from Scipy \autocite{virtanenSciPyFundamentalAlgorithms2019} in the $(25, 75)$ quantile range and the regularization parameter was hyperparameter optimized. This linear model is quick and easy to both implement and fit, and can be seen as the simplest baseline model.

% The $K$-nearest neighbors (KNN) algorithm was fitted to the data with a data preprocessing pipeline similar to the linear model with median imputation and robust scaling of the input features. For the HPO the number of neighbors, $K$, was optimized for together with the $p$-norm of the metric, $p \in \left\{ 1, 2\right\}$ (Manhattan, Eucludian, see also \autoref{subsec:regularization}).

% Finally, support vector machines\sidenote{For regression, also known as support vector regression\autocite{awadSupportVectorRegression2015}.} (SVR) were used with the same preprocessing pipeline as the previous two methods and hyperparameter optmized for the $L_2$ regularization parameter $C$ and the kernel coefficient $\gamma$ for the radial basis function (RBF) kernel. 

% The results for the five different models are shown in Figure~\ref{fig:h:multiple_models}, where the two subplots are the same up to a logarithmic scaling of the ordinate axis in the right plot. It is easily seen how XGBoost and LightGBM are the best-performing models together with the ensemble (ENS) of the different models, see paragraph below for further explanation. In the figure is also shown the MAD on the test set, which confirms the visual clue: that XGBoost performs best, followed by the ensemble model.

% \begin{figure}[ht!]
%   \centerfloat
%   \includegraphics[draft, width=0.98\textwidth, trim=10 130 40 10, clip]{figures/housing/Ejerlejlighed_v19_cut_all_Ncols_all_all_models.pdf}
%   \caption[Multiple Models XXX]
%           {Multiple models. XXX
%           } 
%   \label{fig:h:multiple_models}
% \end{figure}

% The five different models -- XGB, LGB, LIN, KNN, SVR -- should be able to each capture different parts of the hyperdimensional phase space and an ensemble of these models should would thus be expected to be as good or better as the best of the individual models. This kind of ensemble model is called sometimes called \emph{super learner} in the statistics community \autocite{polleySuperLearnerPrediction2010,vanSuperLearner2007}. To make sure that the ensemble model is not just retraining on the training set, and thus end up overfitting, we follow the process from \citet{polleySuperLearnerPrediction2010}. Using cross-validation for time-series data with $k=10$, the training data is split up into folds sorted by time. Each fold is fitted with all five models, and the prediction of the next fold is made for all five models. This is repeated for the remaining folds untill one ends up with a matrix of predictions $Z \in \mathbb{R}^{(N \times 5)}$ for $N$ training samples. Since all the folds in $Z$ consists of predictions on unseen data\sidenote{The predictions for the very first fold is based on training data.} this prevents overfitting. The meta learner then fits $Z$ to the actual predictions of the training data $y$ in the ususal way. The combination of a meta learner fitted to the predictions of individual models is called an ensemble model.

% At first an XGBoost model was used as the meta learner yielding decent results, however, still performing worse than the single XGB model ($\mathrm{MAD} = \SI{9.57}{\percent}$). To better understand the issue, the meta learner was changed to a linear model which would basically just compute a weighted average of the different models:
% \begin{equation}
%   \label{eq:h:meta_learner}
%   \Psi_\mathrm{meta}(\vec{x}) = \sum_{i=1}^5 \alpha_i \Psi_i(\vec{x}),
% \end{equation}
% where $\bm{\alpha}$ is a vector of the weights for the meta learner and $\Psi$ is an ML-model. The linear model performed even worse ($\mathrm{MAD} = \SI{10.48}{\percent}$) than the XGB model, yet it was more transparent. During the debugging process it was realised that none of these models actually optimize the evaluation function, MAD, directly. The XGBoost model was using the Cauchy loss found earlier and the linear regresson model a simple squared error loss. Since a simple weighted average should work as the meta learner \citep{polleySuperLearnerPrediction2010}, a custom algorithm for finding $\bm{\alpha}$ according to MAD was implemented.

% Given the training data, the evaluation function as a function of $\bm{\alpha}$ was minimized using of the MINUIT algorithm\cite{1975CoPhC..10..343J} via the iminuit\cite{iminuit} Python interface. It yielded decent result, yet they were all very dependent on the initial parameter of the fit indicating many local minima. A scan over the 5-dimensional hyperspace in steps of $0.01$ was thus conducted and the result of this scan was used a the new initial parameter in the minimization routine. This yielded the following result:
% \begin{equation}
%   \bm{\alpha} = \begin{bmatrix*}[r] \alpha_\mathrm{LIN} \\  \alpha_\mathrm{KNN} \\ \alpha_\mathrm{SVR} \\ \alpha_\mathrm{XGB} \\ \alpha_\mathrm{LGB} \end{bmatrix*} = \begin{bmatrix*}[r] \SI{0.202}{\percent} \\  \SI{0.002}{\percent} \\\SI{0.001}{\percent} \\\SI{81.302}{\percent} \\\SI{20.002}{\percent} \end{bmatrix*}.
% \end{equation}
% The fact that it sums to more than \num{1} just corresponds to an overall scaling. When using the found $\bm{\alpha}$ in equation \eqref{eq:h:meta_learner}, one gets the ensemble model (ENS) shown in Figure~\ref{fig:h:multiple_models} with a $\mathrm{MAD} = \SI{9.20}{\percent}$. This value is the evaluation loss on the test set based on only the training data and thus outperforms all of the individual models. 

% The paragraphs above refer to apartments, however, the intermediate results for houses showed the same pattern. The combined model along with their performance can be seen in Fig XXX in the appendix. 

% \section{Discussion}

% The subproject of estimating housing prices has focussed a lot on experimenting with different machine learning models and how to optimize them. As it can be seen in the previous sections, the choice of ML model is by far the most important. Actually, the gain from HPO is quite small, especially considering the amount of time spent on it\sidenote{Not only user-time programming it, but also computational ressources used.}. With the dataset at hand, decent results were made, however, nowhere near the performance of the realtors'. There are two main resons for this, the first being that realtors are educated within this field and thus has developed the skills required for estimating the price of a house over many years of hard work. The second reason is the fact that the realtor has acces to a lot more data than ML models have. We are not in possesion of any \emph{indoor} variables as we call it. The area of the house, the number of rooms, the name of the street, and the distance to a highway are all varibles that are in the data set but none of them describe the overall quality of the house, the maintenance level, the the age of the kitchen or bathroom. These features are invisble to the ML model. 

% During the project it was investigated how to get acces to these variables. At first the online images from each sale was suggested, however, it turned out that Boligsiden only have the right to use them while a residence is for sale; when it is sold all rights return to the photographer. The images are not the only thing that provide more information about the condition of the residence, also the descriptions does that. They turned out to be available for most of the sales and was investigated for a short period. At that time of the project, the MAD for (a tight subset of the) apartments was around $\pm \SI{10}{\percent}$ and $\pm \SI{20}{\percent}$ for houses. By using methods from the big natural language processing (NLP) community with in the field of machine learning, it was possible to reduce the MAD to around $\pm \SI{8}{\percent}$ and $\pm \SI{15}{\percent}$ for apartments and houses respectively. From the improvement in performance it is visible how apartments in general are much more uniform compared to houses where the \q{inside} is more decisive regarding the price. 

% The methods for translating the text to numerical variables decipherable by classical ML models where for instance simple \emph{bag of words} (BOW) models and term \emph{Term Frequency, Inverse Document Frequency} (TF-IDF) but also slightly more advanced statistical tools such as \emph{Latent Dirichlet Allocation} (LDA). An old example of the learnt text model is seen in Figure~\ref{fig:h:shap_text} where a housing-based model was trained with the five numerical variables \code{Ejendomsvaerdi0} (PPPV), \code{GeoPostNr} (postal code), \code{ByggeAAr} (year of construction), \code{Afstand_Kyst} (distance to shore), and \code{BeregnetAreal} (weighted area) and the text descriptions (encoded with TF-IDF). The summary of the trained model is as a SHAP plot in Figure~\ref{fig:h:shap_text}. As expected, the most important features are the numerical ones, however, the word \code{flot} (\q{pretty}) was in top five. The model also learnt that \code{flot} has a positive impact on the price compared to \colorbox{light-gray}{\texttt{trænger}} (\q{requires}) which has a negative impact.

% \begin{marginfigure}
%   \includegraphics[width=0.99\textwidth]{figures/housing_text/villa_tfidf.pdf}
%   \caption[SHAP plot villa TFIDF XXX]{SHAP plot villa TFIDF XXX.}
%   \label{fig:h:shap_text}
% \end{marginfigure}

% The descriptions turned out to be more time-consuming to extract for Boligsiden and along with the fact that overall deadline was quickly approching, the remaining time was focussed on the main part of the project, the quark gluon discrimination. Given more time, the text analysis would definitely be first step for further improving the accuracy and precision of the price predictions. 

% Another step would be to apply more modern deep learning\sidenote{Basically advanced neural networks with many layers.} methods. These methods were briefly experimented with in the intial stages of this subproject but showed inferior performance compared to BDTs. It is a generally accepted truth (with modifications) in the ML community that neural networks underperform, or at least not outperform, classical ML methods on structured data\sidenote{In general data that can be described by a spread sheet, i.e. has a well-defined number of variables and observations}. Most often they have the inherent complexity to perform as well as ML methods, however, this requires extensive architecture optimization, or, in short; the hypothesis space for neural networks is much larger than for classical ML methods and thus requires more care to avoid overfitting.


\FloatBarrier
\chapter{Particle Physics and LEP}
\label{ch:particle_physcis_LEP}
% \immediate\write{./quote.sh \the\value{chapter}}
% \epigraph{\input{/tmp/fortune-\the\value{chapter}}}{--- \textup{\texttt{/usr/bin/fortune}}}


\begin{figure}
  \includegraphics[width=0.9\textwidth]{figures/feynman_diagrams/eeZqqg.pdf}
  \caption[Feynman diagram for the jet production at LEP]{Feynman diagram showing the $e^+ e^- \rightarrow Z^0$ production at LEP. The $Z^0$ has several decay modes where the $Z \rightarrow q\bar{q}g$ is shown here.}
  \label{fig:hep:feynman_3j_qqg}
  %\zsavepos{pos:textfig}
\end{figure}






The \TL document classes define a style similar to the
style Edward Tufte uses in his books and handouts.  Tufte's style is known
for its extensive use of sidenotes, tight integration of graphics with
text, and well-set typography.  This document aims to be at once a
demonstration of the features of the \TL document classes
and a style guide to their use.

\section{Page Layout}\label{sec:page-layout}
\subsection{Headings}\label{sec:headings}\index{headings}
This style provides \textsc{a}- and \textsc{b}-heads (that is,
\Verb|\section| and \Verb|\subsection|), demonstrated above.

If you need more than two levels of section headings, you'll have to define
them yourself at the moment; there are no pre-defined styles for anything below
a \Verb|\subsection|.  As Bringhurst points out in \textit{The Elements of
Typographic Style}, 
%  \citep{Bringhurst2005} 
you should ``use as many levels of
headings as you need: no more, and no fewer.''

The \TL classes will emit an error if you try to use
\linebreak\Verb|\subsubsection| and smaller headings.

% let's start a new thought -- a new section
\newthought{In his later books}, 
%  \citep{Tufte2006}
 Tufte
starts each section with a bit of vertical space, a non-indented paragraph,
and sets the first few words of the sentence in \textsc{small caps}.  To
accomplish this using this style, use the \doccmddef{newthought} command:
\begin{docspec}
  \doccmd{newthought}\{In his later books\}, Tufte starts\ldots
\end{docspec}



\chapter{Quark Gluon Analysis}
\label{ch:quark_gluon_analysis}

\section{Sidenotes}\label{sec:sidenotes}
One of the most prominent and distinctive features of this style is the
extensive use of sidenotes.  There is a wide margin to provide ample room
for sidenotes and small figures.  Any \doccmd{footnote}s will automatically
be converted to sidenotes.\footnote{This is a sidenote that was entered
using the \texttt{\textbackslash footnote} command.}  If you'd like to place ancillary
information in the margin without the sidenote mark (the superscript
number), you can use the \doccmd{marginnote} command.\marginnote{This is a
margin note.  Notice that there isn't a number preceding the note, and
there is no number in the main text where this note was written.}





\begin{figure}
  \includegraphics[width=0.95\textwidth, trim=0 0 0 0, clip]{figures/quarks/btagging_variables_hist-down_sample=1.00-ML_vars=vertex-selection=b-ejet_min=4-n_iter_RS_lgb=99-n_iter_RS_xgb=9-cdot_cut=0.90-version=19.pdf}
  \caption[Histograms of the vertex variables]
          {Histograms of the three vertex variables, \code{projet}, \code{bqvjet}, and \code{ptljet}, used as input variables in the b-tagging models. In blue colors the variables are shown for \textcolor{blue}{true b-jets}, in red for \textcolor{red}{true c-jets}, in green for \textcolor{green}{true g-jets}, and in orange for \textcolor{orange}{all of the jets} (including non q-matched). In fully opaque color are shown the distributions for 2-jet events, in dashed (and lighter color) 3-jet events, and in semi-transparent 4-jet events. Notice the logarithmic y-axis, that there are no g-jets for 2-jet events (as expected), and that all of the distributions are very similar not matter how many jets.
          } 
  \label{fig:q:vertex_variables}
\end{figure}



\begin{figure}
  \includegraphics[width=0.95\textwidth, trim=20 20 50 50, clip]{figures/quarks/df_UMAP-X=1120952-n_neighbors=250-min_dist=0.2-metric=euclidean-input2b_njet=4_algorithm=UMAP_single.pdf}
  \caption[UMAP vizualisation of vertex variables]
          {Vizualisation of the vertex variables for the different categories: \textcolor{blue}{true b-jets} in blue, \textcolor{red}{true c-jets} in red, \textcolor{green}{true uds-jets} in green, \textcolor{orange}{true g-jets} in orange, and \textcolor{purple}{non q-matched}. The clustering is performed with the UMAP algorithm which outputs a 2D-projection. This projection is then visualized using the Datashader which takes takes care of point size, avoids over- and under-plotting, and color intensity. 
          } 
  \label{fig:q:UMAP_vertex}
\end{figure}



\begin{figure}
  \includegraphics[width=0.95\textwidth, trim=0 0 0 30, clip]{figures/quarks/y_pred_3_jet_hist-down_sample=1.00-ML_vars=vertex-selection=b-ejet_min=4-n_iter_RS_lgb=99-n_iter_RS_xgb=9-cdot_cut=0.90-version=19.pdf}
  \caption[b-tag scores in 3-jet events]
          {Histogram of b-tag scores (model prediction) in 3-jet events for \textcolor{blue}{NNB} (the neural network trained by ATLAS, also called \code{nnbjet}) in blue, \textcolor{red}{XGB} in red, and \textcolor{green}{XGB} in green. We see that the XGB predictions closely match those of NNB which is a good confirmation of a successful fit.  
          } 
  \label{fig:q:btag_scores_3j}
\end{figure}




\begin{figure}
  \includegraphics[width=0.95\textwidth, trim=10 10 10 40, clip]{figures/quarks/ROC_4_jet-down_sample=1.00-ML_vars=vertex-selection=b-ejet_min=4-n_iter_RS_lgb=99-n_iter_RS_xgb=9-cdot_cut=0.90-version=19.pdf}
  \caption[ROC curve for b-tag in 4-jet events]
          {ROC curve of the three b-tag models in 3-jet events for \textcolor{blue}{NNB} (the neural network trained by ATLAS, also called \code{nnbjet}) in blue, \textcolor{red}{XGB} in red, and \textcolor{green}{XGB} in green. In the legend the Area Under Curve (AUC) is also shown. Notice that the XGB and XGB models share performance and it is thus due to overplotting that only the green line for XGB can be seen. In the particle physics community False Positive Rate (FPR) is sometimes better known as background efficiency and True Positive Rate (TPR) as signal efficiency.  
          } 
  \label{fig:q:roc_btag_4j}
\end{figure}




\begin{figure}
  \includegraphics[width=0.95\textwidth, trim=10 10 10 40, clip]{figures/quarks/gtag_y_pred_4_jet_hist-down_sample=1.00-ML_vars=vertex-selection=b-ejet_min=4-n_iter_RS_lgb=99-n_iter_RS_xgb=9-cdot_cut=0.90-version=19.pdf}
  \caption[g-tag scores in 4-jet events]
          {
            Histogram of g-tag scores (model prediction) in 4-jet events for \textcolor{blue}{XGB: Energy Ordered} in blue, \textcolor{red}{XGB: Shuffled} in red, \textcolor{green}{XGB: All Permutations} in green, \textcolor{orange}{XGB: Sum 1D} in orange, \textcolor{purple}{PermNet: Energy Ordered} in purple, \textcolor{light-blue}{PermNet: Shuffled} in light-blue, \textcolor{light-green}{PermNet: All Permutations} in light-green, \textcolor{light-purple}{PermNet: Sum 1D} in light-purple.  Here XGB and PermNet are the two different type of models and \q{Energy Ordered}, \q{Shuffled}, \q{All Permutations}, and \q{Sum 1D} are the different methods used for making the input data permutation invariant.  
          }   
  \label{fig:q:gtag_scores_4j}
\end{figure}


\begin{figure}
  \includegraphics[width=0.95\textwidth, trim=10 10 10 40, clip]{figures/quarks/gtag-histogram-sigbkg-down_sample=1.00-ML_vars=vertex-selection=b-ejet_min=4-n_iter_RS_lgb=99-n_iter_RS_xgb=9-cdot_cut=0.90-version=19-njet=4.pdf}
  \caption[g-tag scores in 4-jet events for signal and background]
          {Histogram of g-tag scores (model prediction) from the XGB-model in 4-jet events for \textcolor{blue}{b signal} in blue, \textcolor{red}{c signal} in red, \textcolor{green}{l signal} in green, \textcolor{orange}{b background} in orange, \textcolor{purple}{c background} in purple, \textcolor{light-blue}{l background} in light-blue.
          } 
  \label{fig:q:gtag_scores_4j_sig_bkg}
\end{figure}





\begin{figure}
  \includegraphics[width=0.95\textwidth, trim=10 10 10 40, clip]{figures/quarks/gtag_ROC_4_jet-down_sample=1.00-ML_vars=vertex-selection=b-ejet_min=4-n_iter_RS_lgb=99-n_iter_RS_xgb=9-cdot_cut=0.90-version=19.pdf}
  \caption[ROC curve for g-tag in 4-jet events]
          {ROC curve of the eight g-tag models in 4-jet events. First one in dashed black is the ROC curve that you get by random chance. The colors are the same as in \figref{fig:q:gtag_scores_4j} and in the legend also the Area Under the ROC curve (AUC) is shown. 
          Notice that the XGB model which uses the energy ordered data produced the best model, however, this model is not permutation invariant. Of the permutation invariant models (the rest), the XGB model trained on all permutations of the b-tags performs highest. The lowest performing models are the two models trained only on the 1-dimensional sum of b-tags, as expected, however, still with a better performance than expected by the author.  
          } 
  \label{fig:q:roc_gtag_4j}
\end{figure}





\begin{figure}
  \includegraphics[width=0.95\textwidth, trim=10 10 10 20, clip]{figures/quarks/gtag_sum_method_njet=4-down_sample=1.00-ML_vars=vertex-selection=b-ejet_min=4-n_iter_RS_lgb=99-n_iter_RS_xgb=9-cdot_cut=0.90-version=19.pdf}
  \caption[1D Sum Model Cuts for 4-jets]
          {Histogram of the distribution of \textcolor{blue}{signal} in blue and \textcolor{red}{background} in red for 1-dimensional sum of b-tags training data. A histogram of the \textcolor{green}{cut values} from the XGB model trained on this data is shown in green together with a rug plot of the cut values in black. Notice how most of the cuts match up with the signal peak at around a $\sum \beta_i \sim 2.1$, however, there are also quite a lot of cuts around $\sum \beta_i \sim 0.5$.
          } 
  \label{fig:q:1d_sum_model_cuts_4j}
\end{figure}


\begin{figure}
  \includegraphics[width=0.95\textwidth, trim=10 10 10 20, clip]{figures/quarks/gtag_sum_models_njet=4-down_sample=1.00-ML_vars=vertex-selection=b-ejet_min=4-n_iter_RS_lgb=99-n_iter_RS_xgb=9-cdot_cut=0.90-version=19.pdf}
  \caption[1D Sum Models Predictions and Signal Fraction for 4-jets]
          {Plot of the (1D) g-tag scores as a function of $\sum \beta_i$ for the \textcolor{blue}{XGB} model in blue and the \textcolor{red}{PermNet} model in red. Here the g-tag scores are just the models' output values when input a uniformly spaced grid of $\sum \beta_i$ values between 0 and 4. The signal fraction (based on the signal and background histograms in \figref{fig:q:1d_sum_model_cuts_4j}) is plotted as black error bars where the size of the error bars is based on the propagated uncertainties of the signal and background histogram assuming Poissonian statistics. Notice how both models capture the overall trend of the signal fraction with the PermNet being particularly close. 
          } 
  \label{fig:q:1d_sum_models_signal_fraction_4j}
\end{figure}




\begin{figure}
  \includegraphics[width=0.95\textwidth, trim=10 10 10 60, clip]{figures/quarks/cv_res_lgb-down_sample=1.00-ML_vars=vertex-selection=b-ejet_min=4-n_iter_RS_lgb=99-n_iter_RS_xgb=9-cdot_cut=0.90-version=19.pdf}
  \caption[Hyperparameter Optimization of b- and g-tagging]
          {Hyperparameter Optimization (HPO) results after running 100 iterations of Random Search (only 10 for XGB). In the top row are the results of the 3-jet models and in the bottom row the results of the 4-jet models. From left to right, we have first) the b-tagging results of XGB, second) the b-tagging results of XGB using only 10 iterations of RS, third) the g-tagging results of XGB fit on the Energy Ordered b-tags, and forth) the g-tagging results of XGB fit on the shuffled b-tags. Notice the different ranges on the y-axes.
          } 
  \label{fig:q:CV_res_iterations}
\end{figure}


\begin{figure}
  \includegraphics[width=0.95\textwidth, trim=0 0 0 0, clip]{figures/quarks/CV_viz-njet=3-name=lf_gtag_shuffled_lgb_down_sample=1.00-ML_vars=vertex-selection=b-ejet_min=4-n_iter_RS_lgb=99-n_iter_RS_xgb=9-cdot_cut=0.90-version=19.pdf}
  \caption[Overview of Hyperparamaters of g-tagging for 3-jet shuffled events]
          {Hyperparameter optimization results of g-tagging for 3-jet shuffled events. The results are shown as parallel coordinates with each hyperparameter along the x-axis and the value of that parameter on the y-axis. Each line is an event in the 4-dimensional space colored according to the performance of that hyperparameter as measured by AUC from \textcolor{viridis-dark}{highest} AUC in dark blue to \textcolor{viridis-light}{lowest} AUC in yellow. The \textcolor{red}{single best hyperparameter} is shown in red. 
          } 
  \label{fig:q:initial_CV_res_parallel_coords}
\end{figure}



\begin{figure}
  \includegraphics[width=0.95\textwidth, trim=0 0 0 40, clip]{figures/quarks/shap_values-down_sample=1.00-ML_vars=vertex-selection=b-ejet_min=4-n_iter_RS_lgb=99-n_iter_RS_xgb=9-cdot_cut=0.90-version=19-njet=3loc=24325621.pdf}
  \caption[SHAP Prediction Explanation for b-like jet]
          {Model explanation for the 3-jet b-tagging model for a b-like jet. The first column is the bias of the training set which acts as the naive prediction baseline, the rest are the input data variables. On the right hand side of the plot is the model prediction shown. The left part of the plot is shown in log-odds space, the right part in probability space. The model prediction is the sum of the log-odds (5.09 in this example) transformed into probability space. The \textcolor{red}{negative} log-odd values are shown in red, \textcolor{green}{positive} ones in green, and the \textcolor{blue}{prediction value} in blue. 
          } 
  \label{fig:q:shap_signal}
\end{figure}


\begin{figure}
  \includegraphics[width=0.95\textwidth, trim=0 0 0 40, clip]{figures/quarks/2d-histograms-ejet-btag-comparison-down_sample=1.00-ML_vars=vertex-selection=b-ejet_min=4-n_iter_RS_lgb=99-n_iter_RS_xgb=9-cdot_cut=0.90-version=19-njet=3.pdf}
  \caption[Monte Carlo -- Data bias for b-tags and jet energy]
          {Comparison of the b-tag and jet energy (\code{Ejet}) distributions for Monte Carlo (MC) versus data. In the top row the 2D-distributions are shown for MC on the left (without the extra MCb samples) and data on the right. In the bottom row the 1D marginal distrubtions are shown for the b-tag and the jet energy with \textcolor{red}{data} in red and \textcolor{blue}{Monte Carlo} ones in blue. Notice the the almost identical distributions in b-tag. 
          } 
  \label{fig:q:btag_Ejet_comparison}
\end{figure}

\begin{figure}
  \includegraphics[width=0.95\textwidth, trim=0 0 0 40, clip]{figures/quarks/eff_b_bsig-down_sample=1.00-ML_vars=vertex-selection=b-ejet_min=4-n_iter_RS_lgb=99-n_iter_RS_xgb=9-cdot_cut=0.90-version=19.pdf}
  \caption[b-Tagging Efficiency $\varepsilon_b^{b\mathrm{-sig}}$ as a function of jet energy]
          {Efficiency of the b-tags for b-jets in the b-signal region for 3-jet events, $\varepsilon_b^{b\mathrm{-sig}}$, as a function of jet energy \code{Ejet}. The b-signal region is defined as $\beta > 0.9$. In the plot the efficiencies are shown for \textcolor{blue}{MC Truth} in blue, \textcolor{red}{MC TTP} in red, \textcolor{green}{MC Truth TTP} in green, and \textcolor{orange}{Data TTP} in orange. The efficiencies (the errorbars) can be read off on the left y-axis and the counts (histograms) on the right y-axis. The abbreviation TTP is short for \q{Tag, Tag, Probe} where two jets in a event are used as tags and the probe is then used for further analysis. Notice how both MC TTP and Data TTP follow each other closely.  
          } 
  \label{fig:q:effiency_btag_bjet_bsig}
\end{figure}


\begin{figure}
  \includegraphics[width=0.95\textwidth, trim=0 0 0 40, clip]{figures/quarks/eff_b_gsig-down_sample=1.00-ML_vars=vertex-selection=b-ejet_min=4-n_iter_RS_lgb=99-n_iter_RS_xgb=9-cdot_cut=0.90-version=19.pdf}
  \caption[b-Tagging Efficiency $\varepsilon_b^{g\mathrm{-sig}}$ as a function of jet energy]
          {Efficiency of the b-tags for b-jets in the g-signal region for 3-jet events, $\varepsilon_b^{g\mathrm{-sig}}$, as a function of jet energy \code{Ejet}. The g-signal region is defined as $\beta < 0.4$. In the plot the efficiencies are shown for \textcolor{blue}{MC Truth} in blue, \textcolor{red}{MC TTP} in red, \textcolor{green}{MC Truth TTP} in green, and \textcolor{orange}{Data TTP} in orange. The efficiencies (the errorbars) can be read off on the left y-axis and the counts (histograms) on the right y-axis. The abbreviation TTP is short for \q{Tag, Tag, Probe} where two jets in a event are used as tags and the probe is then used for further analysis. Notice how both MC TTP and Data TTP follow each other closely.  
          } 
  \label{fig:q:effiency_btag_bjet_gsig}
\end{figure}


\begin{figure}
  \includegraphics[width=0.95\textwidth, trim=0 0 0 40, clip]{figures/quarks/eff_g_gsig-down_sample=1.00-ML_vars=vertex-selection=b-ejet_min=4-n_iter_RS_lgb=99-n_iter_RS_xgb=9-cdot_cut=0.90-version=19.pdf}
  \caption[b-Tagging Efficiency $\varepsilon_g^{g\mathrm{-sig}}$ as a function of jet energy]
          {Efficiency of the b-tags for g-jets in the g-signal region for 3-jet events, $\varepsilon_g^{g\mathrm{-sig}}$, as a function of jet energy \code{Ejet}. The g-signal region is defined as $\beta < 0.4$. In the plot the efficiencies are shown for \textcolor{blue}{MC Truth} in blue, \textcolor{red}{MC TTP} in red, \textcolor{green}{MC Truth TTP} in green, and \textcolor{orange}{Data TTP} in orange. The efficiencies (the errorbars) can be read off on the left y-axis and the counts (histograms) on the right y-axis. The abbreviation TTP is short for \q{Tag, Tag, Probe} where two jets in a event are used as tags and the probe is then used for further analysis. Notice how both MC TTP and Data TTP follow each other closely.  
          } 
  \label{fig:q:effiency_btag_gjet_gsig}
\end{figure}

\begin{figure}
  \includegraphics[width=0.95\textwidth, trim=0 0 0 30, clip]{figures/quarks/eff_g_bsig-down_sample=1.00-ML_vars=vertex-selection=b-ejet_min=4-n_iter_RS_lgb=99-n_iter_RS_xgb=9-cdot_cut=0.90-version=19.pdf}
  \caption[b-Tagging Efficiency $\varepsilon_g^{b\mathrm{-sig}}$ as a function of jet energy]
          {Efficiency of the b-tags for g-jets in the b-signal region for 3-jet events, $\varepsilon_g^{b\mathrm{-sig}}$, as a function of jet energy \code{Ejet}. The b-signal region is defined as $\beta > 0.9$. In the plot the efficiencies are shown for \textcolor{blue}{MC Truth} in blue, \textcolor{red}{MC TTP} in red, \textcolor{green}{MC Truth TTP} in green, and \textcolor{orange}{Data TTP} in orange. The efficiencies (the errorbars) can be read off on the left y-axis and the counts (histograms) on the right y-axis. The abbreviation TTP is short for \q{Tag, Tag, Probe} where two jets in a event are used as tags and the probe is then used for further analysis.  
          } 
  \label{fig:q:effiency_btag_gjet_bsig}
\end{figure}


\begin{figure}
  \includegraphics[width=0.95\textwidth, trim=0 0 0 40, clip]{figures/quarks/eff_bbg_m_mean-down_sample=1.00-ML_vars=vertex-selection=b-ejet_min=4-n_iter_RS_lgb=99-n_iter_RS_xgb=9-cdot_cut=0.90-version=19.pdf}
  \caption[g-Tagging proxy efficiency for $b\bar{b}g$-events as function of the mean invariant mass]
          {Proxy efficiency of the g-tags for $b\bar{b}g$ 3-jet events as a function of the mean of the two invariant masses $m_{bg}$ and $m_{\bar{b}g}$. The proxy efficiency $\varepsilon_{b\bar{b}g}$ is measured by finding $b\bar{b}g$-events where $\beta_b > 0.9$, $\beta_{\bar{b}}>0.9$, and $\beta_g < 0.4$. and then calculating  $\varepsilon_{b\bar{b}g} = \varepsilon_b^{b\mathrm{-sig}} \cdot \varepsilon_{\bar{b}}^{b\mathrm{-sig}} \cdot  \varepsilon_g^{g\mathrm{-sig}} $. In the top plot $\varepsilon_{b\bar{b}g}$ is shown for \textcolor{blue}{MC} in blue and \textcolor{red}{Data} in red where the counts in each bin can be read on right y-axis. In the bottom plot the ratio between Data and MC is shown.
          } 
  \label{fig:q:effiency_btag_bbg_m_mean}
\end{figure}


\begin{figure}
  \includegraphics[width=0.95\textwidth, trim=0 0 0 40, clip]{figures/quarks/eff_bbg_gtag-down_sample=1.00-ML_vars=vertex-selection=b-ejet_min=4-n_iter_RS_lgb=99-n_iter_RS_xgb=9-cdot_cut=0.90-version=19.pdf}
  \caption[g-Tagging proxy efficiency for $b\bar{b}g$-events as function of g-tag]
          {Proxy efficiency of the g-tags for $b\bar{b}g$ 3-jet events as a function of the event's g-tag. The proxy efficiency $\varepsilon_{b\bar{b}g}$ is measured by finding $b\bar{b}g$-events where $\beta_b > 0.9$, $\beta_{\bar{b}}>0.9$, and $\beta_g < 0.4$. and then calculating  $\varepsilon_{b\bar{b}g} = \varepsilon_b^{b\mathrm{-sig}} \cdot \varepsilon_{\bar{b}}^{b\mathrm{-sig}} \cdot  \varepsilon_g^{g\mathrm{-sig}} $. In the top plot $\varepsilon_{b\bar{b}g}$ is shown for \textcolor{blue}{MC} in blue and \textcolor{red}{Data} in red where the counts in each bin can be read on right y-axis. In the bottom plot the ratio between Data and MC is shown.
          } 
  \label{fig:q:effiency_btag_bbg_gtag}
\end{figure}



\begin{figure}
  \includegraphics[width=0.95\textwidth, trim=0 0 0 40, clip, page=1]{figures/quarks/efficiency_events-down_sample=1.00-ML_vars=vertex-selection=b-ejet_min=4-n_iter_RS_lgb=99-n_iter_RS_xgb=9-cdot_cut=0.90-version=19-njet=4.pdf}
  \caption[g-Tagging efficiency for 4-jet events in MC as a function of normalized gluon gluon jet energy difference]
          {Efficiency of the g-tags for 4-jet events as a function of normalized gluon gluon jet energy difference in Monte Carlo. The efficiency is measured as the number of events with a g-tag higher than 0.8 ($\gamma > 0.8$) out of the total number and the normalized gluon gluon jet energy difference $A$ is $A=\frac{E_{g_\mathrm{max}}-E_{g_\mathrm{min}}}{E_{g_\mathrm{max}}+E_{g_\mathrm{min}}}$ where $E_{g_\mathrm{max}}$ ($E_{g_\mathrm{min}}$) refers to the energy of the gluon with the highest (lowest) energy. The efficiency is plotted for \textcolor{blue}{signal events} according to MC Truth in blue and \textcolor{red}{background events} according to MC Truth in red.
          } 
  \label{fig:q:effiency_gtag_E_diff}
\end{figure}


\begin{figure}
  \includegraphics[width=0.95\textwidth, trim=0 0 0 65, clip, page=1]{figures/quarks/gtag-closure_test-down_sample=1.00-ML_vars=vertex-selection=b-ejet_min=4-n_iter_RS_lgb=99-n_iter_RS_xgb=9-cdot_cut=0.90-version=19-njet=3.pdf}
  \caption[Closure plot between MC Truth and the corrected g-tagging model in 4-jet events for the normalized gluon gluon jet energy difference]
          {Closure plot between MC Truth and the corrected g-tagging model in 4-jet events for the normalized gluon gluon jet energy difference. The corrected g-taggingg model is described in further detail in section XXX \TODO. In the top part of the plot the \textcolor{blue}{MC Truth} is shown in blue, the \textcolor{red}{corrected g-tagging model} \code{"PDF 2gg"} in red, the \textcolor{green}{g-signal distribution} in semi-transparent green and the \textcolor{orange}{g-sideband distribution} in semi-transparent orange. In the bottom part of the plot the ratio between MC Truth and the output of the corrected g-tagging model is shown. The normalized gluon gluon jet energy difference $A$ is $A=\frac{E_{g_\mathrm{max}}-E_{g_\mathrm{min}}}{E_{g_\mathrm{max}}+E_{g_\mathrm{min}}}$ where $E_{g_\mathrm{max}}$ ($E_{g_\mathrm{min}}$) refers to the energy of the gluon with the highest (lowest) energy.
          } 
  \label{fig:q:closure_E_diff}
\end{figure}


\begin{figure}
  \includegraphics[width=0.95\textwidth, trim=0 0 0 118, clip, page=1]{figures/quarks/gtag-R_kt_CA_overview-down_sample=1.00-ML_vars=vertex-selection=b-ejet_min=4-n_iter_RS_lgb=99-n_iter_RS_xgb=9-cdot_cut=0.90-version=19-njet=4}
  \caption[R kt CA overview  XXX \TODO]
          {R kt CA overview XXX \TODO
          } 
  \label{fig:q:R_kt_CA_overview}
\end{figure}

\begin{figure}
  \includegraphics[width=0.95\textwidth, trim=0 0 0 30, clip, page=2]{figures/quarks/gtag-R_kt_CA_overview-down_sample=1.00-ML_vars=vertex-selection=b-ejet_min=4-n_iter_RS_lgb=99-n_iter_RS_xgb=9-cdot_cut=0.90-version=19-njet=4}
  \caption[R kt CA cut region A  XXX \TODO]
          {R kt CA cut region A XXX \TODO
          } 
  \label{fig:q:R_kt_CA_cut_A}
\end{figure}


The specification of the \doccmddef{sidenote} command is:
\begin{docspec}
  \doccmd{sidenote}[\docopt{number}][\docopt{offset}]\{\docarg{Sidenote text.}\}
\end{docspec}

Both the \docopt{number} and \docopt{offset} arguments are optional.  If you
provide a \docopt{number} argument, then that number will be used as the
sidenote number.  It will change of the number of the current sidenote only and
will not affect the numbering sequence of subsequent sidenotes.

Sometimes a sidenote may run over the top of other text or graphics in the
margin space.  If this happens, you can adjust the vertical position of the
sidenote by providing a dimension in the \docopt{offset} argument.  Some
examples of valid dimensions are:
\begin{docspec}
  \ttfamily 1.0in \qquad 2.54cm \qquad 254mm \qquad 6\Verb|\baselineskip|
\end{docspec}
If the dimension is positive it will push the sidenote down the page; if the
dimension is negative, it will move the sidenote up the page.

While both the \docopt{number} and \docopt{offset} arguments are optional, they
must be provided in order.  To adjust the vertical position of the sidenote
while leaving the sidenote number alone, use the following syntax:
\begin{docspec}
  \doccmd{sidenote}[][\docopt{offset}]\{\docarg{Sidenote text.}\}
\end{docspec}
The empty brackets tell the \Verb|\sidenote| command to use the default
sidenote number.

If you \emph{only} want to change the sidenote number, however, you may
completely omit the \docopt{offset} argument:
\begin{docspec}
  \doccmd{sidenote}[\docopt{number}]\{\docarg{Sidenote text.}\}
\end{docspec}

The \doccmddef{marginnote} command has a similar \docarg{offset} argument:
\begin{docspec}
  \doccmd{marginnote}[\docopt{offset}]\{\docarg{Margin note text.}\}
\end{docspec}

\section{References}
References are placed alongside their citations as sidenotes,
as well.  This can be accomplished using the normal \doccmddef{citep}
command.\sidenote{The first paragraph of this document includes a citation.}

The complete list of references may also be printed automatically by using
the \doccmddef{bibliography} command.  (See the end of this document for an
example.)  If you do not want to print a bibliography at the end of your
document, use the \doccmddef{nobibliography} command in its place.  


\section{Figures and Tables}\label{sec:figures-and-tables}
Images and graphics play an integral role in Tufte's work.
In addition to the standard \docenvdef{figure} and \docenvdef{tabular} environments,
this style provides special figure and table environments for full-width
floats.

Full page--width figures and tables may be placed in \docenvdef{figure*} or
\docenvdef{table*} environments.  To place figures or tables in the margin,
use the \docenvdef{marginfigure} or \docenvdef{margintable} environments as follows
(see figure~\ref{fig:marginfig}):

\begin{marginfigure}%
  \includegraphics[width=\linewidth]{helix}
  \caption{This is a margin figure.  The helix is defined by 
    $x = \cos(2\pi z)$, $y = \sin(2\pi z)$, and $z = [0, 2.7]$.  The figure was
    drawn using Asymptote (\url{http://asymptote.sf.net/}).}
  \label{fig:marginfig}
\end{marginfigure}

\begin{docspec}
\textbackslash begin\{marginfigure\}\\
  \qquad\textbackslash includegraphics\{helix\}\\
  \qquad\textbackslash caption\{This is a margin figure.\}\\
  \qquad\textbackslash label\{fig:marginfig\}\\
\textbackslash end\{marginfigure\}\\
\end{docspec}

The \docenv{marginfigure} and \docenv{margintable} environments accept an optional parameter \docopt{offset} that adjusts the vertical position of the figure or table.  See the ``\nameref{sec:sidenotes}'' section above for examples.  The specifications are:
\begin{docspec}
  \textbackslash{begin\{marginfigure\}[\docopt{offset}]}\\
  \qquad\ldots\\
  \textbackslash{end\{marginfigure\}}\\
  \mbox{}\\
  \textbackslash{begin\{margintable\}[\docopt{offset}]}\\
  \qquad\ldots\\
  \textbackslash{end\{margintable\}}\\
\end{docspec}

Figure~\ref{fig:fullfig} is an example of the \docenv{figure*}
environment and figure~\ref{fig:textfig} is an example of the normal
\docenv{figure} environment.

\begin{figure*}[h]
  \includegraphics[width=\linewidth]{sine.pdf}%
  \caption{This graph shows $y = \sin x$ from about $x = [-10, 10]$.
  \emph{Notice that this figure takes up the full page width.}}%
  \label{fig:fullfig}%
\end{figure*}

\begin{figure}
  \includegraphics{hilbertcurves.pdf}
%  \checkparity This is an \pageparity\ page.%
  \caption[Hilbert curves of various degrees $n$.]{Hilbert curves of various degrees $n$. \emph{Notice that this figure only takes up the main textblock width.}}
  \label{fig:textfig}
  %\zsavepos{pos:textfig}
\end{figure}

As with sidenotes and marginnotes, a caption may sometimes require vertical
adjustment. The \doccmddef{caption} command now takes a second optional
argument that enables you to do this by providing a dimension \docopt{offset}.
You may specify the caption in any one of the following forms:
\begin{docspec}
  \doccmd{caption}\{\docarg{long caption}\}\\
  \doccmd{caption}[\docarg{short caption}]\{\docarg{long caption}\}\\
  \doccmd{caption}[][\docopt{offset}]\{\docarg{long caption}\}\\
  \doccmd{caption}[\docarg{short caption}][\docopt{offset}]%
                  \{\docarg{long caption}\}
\end{docspec}
A positive \docopt{offset} will push the caption down the page. The short
caption, if provided, is what appears in the list of figures/tables, otherwise
the ``long'' caption appears there. Note that although the arguments
\docopt{short caption} and \docopt{offset} are both optional, they must be
provided in order. Thus, to specify an \docopt{offset} without specifying a
\docopt{short caption}, you must include the first set of empty brackets
\Verb|[]|, which tell \doccmd{caption} to use the default ``long'' caption. As
an example, the caption to figure~\ref{fig:textfig} above was given in the form
\begin{docspec}
  \doccmd{caption}[Hilbert curves...]\{Hilbert curves...\}
\end{docspec}

Table~\ref{tab:normaltab} shows table created with the \docpkg{booktabs}
package.  Notice the lack of vertical rules---they serve only to clutter
the table's data.

\begin{table}[ht]
  \centering
  \fontfamily{ppl}\selectfont
  \begin{tabular}{ll}
    \toprule
    Margin & Length \\
    \midrule
    Paper width & \unit[8\nicefrac{1}{2}]{inches} \\
    Paper height & \unit[11]{inches} \\
    Textblock width & \unit[6\nicefrac{1}{2}]{inches} \\
    Textblock/sidenote gutter & \unit[\nicefrac{3}{8}]{inches} \\
    Sidenote width & \unit[2]{inches} \\
    \bottomrule
  \end{tabular}
  \caption{Here are the dimensions of the various margins used in the Tufte-handout class.}
  \label{tab:normaltab}
  %\zsavepos{pos:normaltab}
\end{table}

\newthought{Occasionally} \LaTeX{} will generate an error message:\label{err:too-many-floats}
\begin{docspec}
  Error: Too many unprocessed floats
\end{docspec}
\LaTeX{} tries to place floats in the best position on the page.  Until it's
finished composing the page, however, it won't know where those positions are.
If you have a lot of floats on a page (including sidenotes, margin notes,
figures, tables, etc.), \LaTeX{} may run out of ``slots'' to keep track of them
and will generate the above error.

\LaTeX{} initially allocates 18 slots for storing floats.  To work around this
limitation, the \TL document classes provide a \doccmddef{morefloats} command
that will reserve more slots.

The first time \doccmd{morefloats} is called, it allocates an additional 34
slots.  The second time \doccmd{morefloats} is called, it allocates another 26
slots.

The \doccmd{morefloats} command may only be used two times.  Calling it a
third time will generate an error message.  (This is because we can't safely
allocate many more floats or \LaTeX{} will run out of memory.)

If, after using the \doccmd{morefloats} command twice, you continue to get the
\texttt{Too many unprocessed floats} error, there are a couple things you can
do.

The \doccmddef{FloatBarrier} command will immediately process all the floats
before typesetting more material.  Since \doccmd{FloatBarrier} will start a new
paragraph, you should place this command at the beginning or end of a
paragraph.

The \doccmddef{clearpage} command will also process the floats before
continuing, but instead of starting a new paragraph, it will start a new page.

You can also try moving your floats around a bit: move a figure or table to the
next page or reduce the number of sidenotes.  (Each sidenote actually uses
\emph{two} slots.)

After the floats have placed, \LaTeX{} will mark those slots as unused so they
are available for the next page to be composed.

\section{Captions}
You may notice that the captions are sometimes misaligned.
Due to the way \LaTeX's float mechanism works, we can't know for sure where it
decided to put a float. Therefore, the \TL document classes provide commands to
override the caption position.

\paragraph{Vertical alignment} To override the vertical alignment, use the
\doccmd{setfloatalignment} command inside the float environment.  For
example:

\begin{fullwidth}
\begin{docspec}
  \textbackslash begin\{figure\}[btp]\\
  \qquad \textbackslash includegraphics\{sinewave\}\\
  \qquad \textbackslash caption\{This is an example of a sine wave.\}\\
  \qquad \textbackslash label\{fig:sinewave\}\\
  \qquad \hlred{\textbackslash setfloatalignment\{b\}\% forces caption to be bottom-aligned}\\
  \textbackslash end\{figure\}
\end{docspec}
\end{fullwidth}

\noindent The syntax of the \doccmddef{setfloatalignment} command is:

\begin{docspec}
  \doccmd{setfloatalignment}\{\docopt{pos}\}
\end{docspec}

\noindent where \docopt{pos} can be either \texttt{b} for bottom-aligned
captions, or \texttt{t} for top-aligned captions.

\paragraph{Horizontal alignment}\label{par:overriding-horizontal}
To override the horizontal alignment, use either the \doccmd{forceversofloat}
or the \doccmd{forcerectofloat} command inside of the float environment.  For
example:

\begin{fullwidth}
\begin{docspec}
  \textbackslash begin\{figure\}[btp]\\
  \qquad \textbackslash includegraphics\{sinewave\}\\
  \qquad \textbackslash caption\{This is an example of a sine wave.\}\\
  \qquad \textbackslash label\{fig:sinewave\}\\
  \qquad \hlred{\textbackslash forceversofloat\% forces caption to be set to the left of the float}\\
  \textbackslash end\{figure\}
\end{docspec}
\end{fullwidth}

The \doccmddef{forceversofloat} command causes the algorithm to assume the
float has been placed on a verso page---that is, a page on the left side of a
two-page spread.  Conversely, the \doccmddef{forcerectofloat} command causes
the algorithm to assume the float has been placed on a recto page---that is, a
page on the right side of a two-page spread.


\section{Full-width text blocks}

In addition to the new float types, there is a \docenvdef{fullwidth}
environment that stretches across the main text block and the sidenotes
area.

\begin{Verbatim}
\begin{fullwidth}
Lorem ipsum dolor sit amet...
\end{fullwidth}
\end{Verbatim}

\begin{fullwidth}
\small\itshape\lipsum[1]
\end{fullwidth}

\section{Typography}\label{sec:typography}

\subsection{Typefaces}\label{sec:typefaces}\index{typefaces}
If the Palatino, \textsf{Helvetica}, and \texttt{Bera Mono} typefaces are installed, this style
will use them automatically.  Otherwise, we'll fall back on the Computer Modern
typefaces.

\subsection{Letterspacing}\label{sec:letterspacing}
This document class includes two new commands and some improvements on
existing commands for letterspacing.

When setting strings of \allcaps{ALL CAPS} or \smallcaps{small caps}, the
letter\-spacing---that is, the spacing between the letters---should be
% increased slightly. \citep{Bringhurst2005}  The \doccmddef{allcaps} command has proper letterspacing for
strings of \allcaps{FULL CAPITAL LETTERS}, and the \doccmddef{smallcaps} command
has letterspacing for \smallcaps{small capital letters}.  These commands
will also automatically convert the case of the text to upper- or
lowercase, respectively.

The \doccmddef{textsc} command has also been redefined to include
letterspacing.  The case of the \doccmd{textsc} argument is left as is,
however.  This allows one to use both uppercase and lowercase letters:
\textsc{The Initial Letters Of The Words In This Sentence Are Capitalized.}



\section{Document Class Options}\label{sec:options}

\index{class options|(}
The \doccls{tufte-book} class is based on the \LaTeX\ \doccls{book}
document class.  Therefore, you can pass any of the typical book
options.  There are a few options that are specific to the
\doccls{tufte-book} document class, however.

The \docclsoptdef{a4paper} option will set the paper size to \smallcaps{A4} instead of
the default \smallcaps{US} letter size.

The \docclsoptdef{sfsidenotes} option will set the sidenotes and title block in a 
\textsf{sans serif} typeface instead of the default roman.

The \docclsoptdef{twoside} option will modify the running heads so that the page
number is printed on the outside edge (as opposed to always printing the page
number on the right-side edge in \docclsoptdef{oneside} mode).  

The \docclsoptdef{symmetric} option typesets the sidenotes on the outside edge of
the page.  This is how books are traditionally printed, but is contrary to
Tufte's book design which sets the sidenotes on the right side of the page.
This option implicitly sets the \docclsopt{twoside} option.

The \docclsoptdef{justified} option sets all the text fully justified (flush left
and right).  The default is to set the text ragged right.  
The body text of Tufte's books are set ragged right.  This prevents
needless hyphenation and makes it easier to read the text in the slightly
narrower column.

The \docclsoptdef{bidi} option loads the \docpkg{bidi} package which is used with
\tXeLaTeX\ to typeset bi-directional text.  Since the \docpkg{bidi}
package needs to be loaded before the sidenotes and citep commands are defined,
it can't be loaded in the document preamble.

The \docclsoptdef{debug} option causes the \TL classes to output debug
information to the log file which is useful in troubleshooting bugs.  It will
also cause the graphics to be replaced by outlines.

The \docclsoptdef{nofonts} option prevents the \TL classes from
automatically loading the Palatino and Helvetica typefaces.  You should use
this option if you wish to load your own fonts.  If you're using \tXeLaTeX, this
option is implied (\ie, the Palatino and Helvetica fonts aren't loaded if you
use \tXeLaTeX).  

The \docclsoptdef{nols} option inhibits the letterspacing code.  The \TL\
classes try to load the appropriate letterspacing package (either pdf\TeX's
\docpkg{letterspace} package or the \docpkg{soul} package).  If you're using
\tXeLaTeX\ with \docpkg{fontenc}, however, you should configure your own
letterspacing.  

The \docclsoptdef{notitlepage} option causes \doccmd{maketitle} to generate a title
block instead of a title page.  The \doccls{book} class defaults to a title
page and the \doccls{handout} class defaults to the title block.  There is an
analogous \docclsoptdef{titlepage} option that forces \doccmd{maketitle} to
generate a full title page instead of the title block.

The \docclsoptdef{notoc} option suppresses \TL's custom table of contents
(\textsc{toc}) design.  The current \textsc{toc} design only shows unnumbered
chapter titles; it doesn't show sections or subsections.  The \docclsopt{notoc}
option will revert to \LaTeX's \textsc{toc} design.

The \docclsoptdef{nohyper} option prevents the \docpkg{hyperref} package from
being loaded.  The default is to load the \docpkg{hyperref} package and use the
\doccmd{title} and \doccmd{author} contents as metadata for the generated
\textsc{pdf}.

\index{class options|)}



\chapter{Discussion and Outlook}
\label{ch:discussion_outlook}

The \TL document classes are designed to closely emulate Tufte's book
design by default.  However, each document is different and you may encounter
situations where the default settings are insufficient.  This chapter explores
many of the ways you can adjust the \TL document classes to better fit
your needs.

\section{File Hooks}
\label{sec:filehooks}

\index{file hooks|(}
If you create many documents using the \TL classes, it's easier to
store your customizations in a separate file instead of copying them into the
preamble of each document.  The \TL classes provide three file hooks:
\docfilehook{tufte-common-local.tex}{common}, \docfilehook{tufte-book-local.tex}{book}, and
\docfilehook{tufte-handout-local.tex}{handout}.\sloppy

\begin{description}
  \item[\docfilehook{tufte-common-local.tex}{common}]
    If this file exists, it will be loaded by all of the \TL document
    classes just prior to any document-class-specific code.  If your
    customizations or code should be included in both the book and handout
    classes, use this file hook.
  \item[\docfilehook{tufte-book-local.tex}{book}] 
    If this file exists, it will be loaded after all of the common and
    book-specific code has been read.  If your customizations apply only to the
    book class, use this file hook.
  \item[\docfilehook{tufte-common-handout.tex}{handout}] 
    If this file exists, it will be loaded after all of the common and
    handout-specific code has been read.  If your customizations apply only to
    the handout class, use this file hook.
\end{description}

\index{file hooks|)}



\chapter{Conclusion}
\label{ch:conclusion}
% \epigraph{\textit{``TL;DR.''}}{--- The Internet}


\section{\TL Website}\label{sec:website}
The website for the \TL packages is located at
\url{http://code.google.com/p/tufte-latex/}.  On our website, you'll find
links to our \smallcaps{svn} repository, mailing lists, bug tracker, and documentation.

\section{\TL Mailing Lists}\label{sec:mailing-lists}
There are two mailing lists for the \TL project:

\paragraph{Discussion list}
The \texttt{tufte-latex} discussion list is for asking questions, getting
assistance with problems, and help with troubleshooting.  Release announcements
are also posted to this list.  You can subscribe to the \texttt{tufte-latex}
discussion list at \url{http://groups.google.com/group/tufte-latex}.

\paragraph{Commits list}
The \texttt{tufte-latex-commits} list is a read-only mailing list.  A message
is sent to the list any time the \TL code has been updated.  If you'd like to
keep up with the latest code developments, you may subscribe to this list.  You
can subscribe to the \texttt{tufte-latex-commits} mailing list at
\url{http://groups.google.com/group/tufte-latex-commits}.

\section{Getting Help}\label{sec:getting-help}
If you've encountered a problem with one of the \TL document classes, have a
question, or would like to report a bug, please send an email to our
mailing list or visit our website.

To help us troubleshoot the problem more quickly, please try to compile your
document using the \docclsopt{debug} class option and send the generated
\texttt{.log} file to the mailing list with a brief description of the problem.



\section{Errors, Warnings, and Informational Messages}\label{sec:tl-messages}
The following is a list of all of the errors, warnings, and other messages generated by the \TL classes and a brief description of their meanings.
\index{error messages}\index{warning messages}\index{debug messages}

% Errors
\docmsg{Error: \doccmd{subparagraph} is undefined by this class.}{%
The \doccmd{subparagraph} command is not defined in the \TL document classes.
If you'd like to use the \doccmd{subparagraph} command, you'll need to redefine
it yourself.  See the ``Headings'' section on page~\pageref{sec:headings} for a
description of the heading styles available in the \TL document classes.}

\docmsg{Error: \doccmd{subsubsection} is undefined by this class.}{%
The \doccmd{subsubsection} command is not defined in the \TL document classes.
If you'd like to use the \doccmd{subsubsection} command, you'll need to
redefine it yourself.  See the ``Headings'' section on
page~\pageref{sec:headings} for a description of the heading styles available
in the \TL document classes.}

\docmsg{Error: You may only call \doccmd{morefloats} twice. See the\par\noindent\ \ \ \ \ \ \ \ Tufte-LaTeX documentation for other workarounds.}{%
\LaTeX{} allocates 18 slots for storing floats.  The first time
\doccmd{morefloats} is called, it allocates an additional 34 slots.  The second
time \doccmd{morefloats} is called, it allocates another 26 slots.

The \doccmd{morefloats} command may only be called two times.  Calling it a
third time will generate this error message.  See
page~\pageref{err:too-many-floats} for more information.}

% Warnings
\docmsg{Warning: Option `\docopt{class option}' is not supported -{}- ignoring option.}{%
This warning appears when you've tried to use \docopt{class option} with a \TL
document class, but \docopt{class option} isn't supported by the \TL document
class.  In this situation, \docopt{class option} is ignored.}

% Info / Debug messages
\docmsg{Info: The `\docclsopt{symmetric}' option implies `\docclsopt{twoside}'}{%
You specified the \docclsopt{symmetric} document class option.  This option automatically forces the \docclsopt{twoside} option as well.  See page~\pageref{clsopt:symmetric} for more information on the \docclsopt{symmetric} class option.}


\section{Package Dependencies}\label{sec:dependencies}
The following is a list of packages that the \TL document
classes rely upon.  Packages marked with an asterisk are optional.
\begin{multicols}{2}
\begin{itemize}
  \item xifthen
  \item ifpdf*
  \item ifxetex*
  \item hyperref
  \item geometry
  \item ragged2e
  \item chngpage \emph{or} changepage
  \item paralist
  \item textcase
  \item soul*
  \item letterspace*
  \item setspace
  \item natbib \emph{and} bibentry
  \item optparams
  \item placeins
  \item mathpazo*
  \item helvet*
  \item fontenc
  \item beramono*
  \item fancyhdr
  \item xcolor
  \item textcomp
  \item titlesec
  \item titletoc
\end{itemize}
\end{multicols}

%%
% The back matter contains appendices, bibliographies, indices, glossaries, etc.

\appendix
\chapter{Housing Prices Appendix}

\begin{figure*}[h!]
  \includegraphics[draft, width=0.9\textwidth, trim=10 10 10 10, clip]{figures/housing/missing_heatmap.pdf}
  \caption[Validity Heatmap]
          {XXXX \TODO.}
  \label{fig:h:nans_heatmap}
\end{figure*}


% Input all 1D-variables for housing
% \begin{figure*}
  \centering
  \subfloat{\qquad}
  \includegraphics[width=0.45\textwidth, page=1, trim=15 0 15 0, clip]{figures/housing/overview_fig.pdf}\hfil
  \subfloat{\qquad}
  \includegraphics[width=0.45\textwidth, page=2, trim=15 0 15 0, clip]{figures/housing/overview_fig.pdf}
  \subfloat{\qquad}
  \includegraphics[width=0.45\textwidth, page=3, trim=15 0 15 0, clip]{figures/housing/overview_fig.pdf}\hfil
  \subfloat{\qquad}
  \includegraphics[width=0.45\textwidth, page=4, trim=15 0 15 0, clip]{figures/housing/overview_fig.pdf}
  \subfloat{\qquad}
  \includegraphics[width=0.45\textwidth, page=5, trim=15 0 15 0, clip]{figures/housing/overview_fig.pdf}\hfil
  \subfloat{\qquad}
  \includegraphics[width=0.45\textwidth, page=6, trim=15 0 15 0, clip]{figures/housing/overview_fig.pdf}
  \subfloat{\qquad}
  \includegraphics[width=0.45\textwidth, page=7, trim=15 0 15 0, clip]{figures/housing/overview_fig.pdf}\hfil
  \subfloat{\qquad}
  \includegraphics[width=0.45\textwidth, page=8, trim=15 0 15 0, clip]{figures/housing/overview_fig.pdf}
  \subfloat{\qquad}
  \includegraphics[width=0.45\textwidth, page=9, trim=15 0 15 0, clip]{figures/housing/overview_fig.pdf}\hfil
  \subfloat{\qquad}
  \includegraphics[width=0.45\textwidth, page=10, trim=15 0 15 0, clip]{figures/housing/overview_fig.pdf}
  \subfloat{\qquad}
  \includegraphics[width=0.45\textwidth, page=11, trim=15 0 15 0, clip]{figures/housing/overview_fig.pdf}\hfil
  \subfloat{\qquad}
  \includegraphics[width=0.45\textwidth, page=12, trim=15 0 15 0, clip]{figures/housing/overview_fig.pdf}
  \caption[Distributions for the housing price dataset]{Distributions the 168 input variables (excluding \code{ID} and \code{Vejnavn}).}
  \label{fig:h:variable_overview_all_1}
  \vspace{\abovecaptionskip}
\end{figure*}

\begin{figure*}
  \centering
  \subfloat{\qquad}
  \includegraphics[width=0.45\textwidth, page=13, trim=15 0 15 0, clip]{figures/housing/overview_fig.pdf}\hfil
  \subfloat{\qquad}
  \includegraphics[width=0.45\textwidth, page=14, trim=15 0 15 0, clip]{figures/housing/overview_fig.pdf}
  \subfloat{\qquad}
  \includegraphics[width=0.45\textwidth, page=15, trim=15 0 15 0, clip]{figures/housing/overview_fig.pdf}\hfil
  \subfloat{\qquad}
  \includegraphics[width=0.45\textwidth, page=16, trim=15 0 15 0, clip]{figures/housing/overview_fig.pdf}
  \subfloat{\qquad}
  \includegraphics[width=0.45\textwidth, page=17, trim=15 0 15 0, clip]{figures/housing/overview_fig.pdf}\hfil
  \subfloat{\qquad}
  \includegraphics[width=0.45\textwidth, page=18, trim=15 0 15 0, clip]{figures/housing/overview_fig.pdf}
  \subfloat{\qquad}
  \includegraphics[width=0.45\textwidth, page=19, trim=15 0 15 0, clip]{figures/housing/overview_fig.pdf}\hfil
  \subfloat{\qquad}
  \includegraphics[width=0.45\textwidth, page=20, trim=15 0 15 0, clip]{figures/housing/overview_fig.pdf}
  \subfloat{\qquad}
  \includegraphics[width=0.45\textwidth, page=21, trim=15 0 15 0, clip]{figures/housing/overview_fig.pdf}\hfil
  \subfloat{\qquad}
  \includegraphics[width=0.45\textwidth, page=22, trim=15 0 15 0, clip]{figures/housing/overview_fig.pdf}
  \subfloat{\qquad}
  \includegraphics[width=0.45\textwidth, page=23, trim=15 0 15 0, clip]{figures/housing/overview_fig.pdf}\hfil
  \subfloat{\qquad}
  \includegraphics[width=0.45\textwidth, page=24, trim=15 0 15 0, clip]{figures/housing/overview_fig.pdf}
  \caption[Distributions for the housing price dataset]{Distributions the 168 input variables (excluding \code{ID} and \code{Vejnavn}).}
  \label{fig:h:variable_overview_all_2}
  \vspace{\abovecaptionskip}
\end{figure*}

\begin{figure*}
  \centering
  \subfloat{\qquad}
  \includegraphics[width=0.45\textwidth, page=25, trim=15 0 15 0, clip]{figures/housing/overview_fig.pdf}\hfil
  \subfloat{\qquad}
  \includegraphics[width=0.45\textwidth, page=26, trim=15 0 15 0, clip]{figures/housing/overview_fig.pdf}
  \subfloat{\qquad}
  \includegraphics[width=0.45\textwidth, page=27, trim=15 0 15 0, clip]{figures/housing/overview_fig.pdf}\hfil
  \subfloat{\qquad}
  \includegraphics[width=0.45\textwidth, page=28, trim=15 0 15 0, clip]{figures/housing/overview_fig.pdf}
  \subfloat{\qquad}
  \includegraphics[width=0.45\textwidth, page=29, trim=15 0 15 0, clip]{figures/housing/overview_fig.pdf}\hfil
  \subfloat{\qquad}
  \includegraphics[width=0.45\textwidth, page=30, trim=15 0 15 0, clip]{figures/housing/overview_fig.pdf}
  \subfloat{\qquad}
  \includegraphics[width=0.45\textwidth, page=31, trim=15 0 15 0, clip]{figures/housing/overview_fig.pdf}\hfil
  \subfloat{\qquad}
  \includegraphics[width=0.45\textwidth, page=32, trim=15 0 15 0, clip]{figures/housing/overview_fig.pdf}
  \subfloat{\qquad}
  \includegraphics[width=0.45\textwidth, page=33, trim=15 0 15 0, clip]{figures/housing/overview_fig.pdf}\hfil
  \subfloat{\qquad}
  \includegraphics[width=0.45\textwidth, page=34, trim=15 0 15 0, clip]{figures/housing/overview_fig.pdf}
  \subfloat{\qquad}
  \includegraphics[width=0.45\textwidth, page=35, trim=15 0 15 0, clip]{figures/housing/overview_fig.pdf}\hfil
  \subfloat{\qquad}
  \includegraphics[width=0.45\textwidth, page=36, trim=15 0 15 0, clip]{figures/housing/overview_fig.pdf}
  \caption[Distributions for the housing price dataset]{Distributions the 168 input variables (excluding \code{ID} and \code{Vejnavn}).}
  \label{fig:h:variable_overview_all_3}
  \vspace{\abovecaptionskip}
\end{figure*}

\begin{figure*}
  \centering
  \subfloat{\qquad}
  \includegraphics[width=0.45\textwidth, page=37, trim=15 0 15 0, clip]{figures/housing/overview_fig.pdf}\hfil
  \subfloat{\qquad}
  \includegraphics[width=0.45\textwidth, page=38, trim=15 0 15 0, clip]{figures/housing/overview_fig.pdf}
  \subfloat{\qquad}
  \includegraphics[width=0.45\textwidth, page=39, trim=15 0 15 0, clip]{figures/housing/overview_fig.pdf}\hfil
  \subfloat{\qquad}
  \includegraphics[width=0.45\textwidth, page=40, trim=15 0 15 0, clip]{figures/housing/overview_fig.pdf}
  \subfloat{\qquad}
  \includegraphics[width=0.45\textwidth, page=41, trim=15 0 15 0, clip]{figures/housing/overview_fig.pdf}\hfil
  \subfloat{\qquad}
  \includegraphics[width=0.45\textwidth, page=42, trim=15 0 15 0, clip]{figures/housing/overview_fig.pdf}
  \subfloat{\qquad}
  \includegraphics[width=0.45\textwidth, page=43, trim=15 0 15 0, clip]{figures/housing/overview_fig.pdf}\hfil
  \subfloat{\qquad}
  \includegraphics[width=0.45\textwidth, page=44, trim=15 0 15 0, clip]{figures/housing/overview_fig.pdf}
  \subfloat{\qquad}
  \includegraphics[width=0.45\textwidth, page=45, trim=15 0 15 0, clip]{figures/housing/overview_fig.pdf}\hfil
  \subfloat{\qquad}
  \includegraphics[width=0.45\textwidth, page=46, trim=15 0 15 0, clip]{figures/housing/overview_fig.pdf}
  \subfloat{\qquad}
  \includegraphics[width=0.45\textwidth, page=47, trim=15 0 15 0, clip]{figures/housing/overview_fig.pdf}\hfil
  \subfloat{\qquad}
  \includegraphics[width=0.45\textwidth, page=48, trim=15 0 15 0, clip]{figures/housing/overview_fig.pdf}
  \caption[Distributions for the housing price dataset]{Distributions the 168 input variables (excluding \code{ID} and \code{Vejnavn}).}
  \label{fig:h:variable_overview_all_4}
  \vspace{\abovecaptionskip}
\end{figure*}

\begin{figure*}
  \centering
  \subfloat{\qquad}
  \includegraphics[width=0.45\textwidth, page=49, trim=15 0 15 0, clip]{figures/housing/overview_fig.pdf}\hfil
  \subfloat{\qquad}
  \includegraphics[width=0.45\textwidth, page=50, trim=15 0 15 0, clip]{figures/housing/overview_fig.pdf}
  \subfloat{\qquad}
  \includegraphics[width=0.45\textwidth, page=51, trim=15 0 15 0, clip]{figures/housing/overview_fig.pdf}\hfil
  \subfloat{\qquad}
  \includegraphics[width=0.45\textwidth, page=52, trim=15 0 15 0, clip]{figures/housing/overview_fig.pdf}
  \subfloat{\qquad}
  \includegraphics[width=0.45\textwidth, page=53, trim=15 0 15 0, clip]{figures/housing/overview_fig.pdf}\hfil
  \subfloat{\qquad}
  \includegraphics[width=0.45\textwidth, page=54, trim=15 0 15 0, clip]{figures/housing/overview_fig.pdf}
  \subfloat{\qquad}
  \includegraphics[width=0.45\textwidth, page=55, trim=15 0 15 0, clip]{figures/housing/overview_fig.pdf}\hfil
  \subfloat{\qquad}
  \includegraphics[width=0.45\textwidth, page=56, trim=15 0 15 0, clip]{figures/housing/overview_fig.pdf}
  \subfloat{\qquad}
  \includegraphics[width=0.45\textwidth, page=57, trim=15 0 15 0, clip]{figures/housing/overview_fig.pdf}\hfil
  \subfloat{\qquad}
  \includegraphics[width=0.45\textwidth, page=58, trim=15 0 15 0, clip]{figures/housing/overview_fig.pdf}
  \subfloat{\qquad}
  \includegraphics[width=0.45\textwidth, page=59, trim=15 0 15 0, clip]{figures/housing/overview_fig.pdf}\hfil
  \subfloat{\qquad}
  \includegraphics[width=0.45\textwidth, page=60, trim=15 0 15 0, clip]{figures/housing/overview_fig.pdf}
  \caption[Distributions for the housing price dataset]{Distributions the 168 input variables (excluding \code{ID} and \code{Vejnavn}).}
  \label{fig:h:variable_overview_all_5}
  \vspace{\abovecaptionskip}
\end{figure*}

\begin{figure*}
  \centering
  \subfloat{\qquad}
  \includegraphics[width=0.45\textwidth, page=61, trim=15 0 15 0, clip]{figures/housing/overview_fig.pdf}\hfil
  \subfloat{\qquad}
  \includegraphics[width=0.45\textwidth, page=62, trim=15 0 15 0, clip]{figures/housing/overview_fig.pdf}
  \subfloat{\qquad}
  \includegraphics[width=0.45\textwidth, page=63, trim=15 0 15 0, clip]{figures/housing/overview_fig.pdf}\hfil
  \subfloat{\qquad}
  \includegraphics[width=0.45\textwidth, page=64, trim=15 0 15 0, clip]{figures/housing/overview_fig.pdf}
  \subfloat{\qquad}
  \includegraphics[width=0.45\textwidth, page=65, trim=15 0 15 0, clip]{figures/housing/overview_fig.pdf}\hfil
  \subfloat{\qquad}
  \includegraphics[width=0.45\textwidth, page=66, trim=15 0 15 0, clip]{figures/housing/overview_fig.pdf}
  \subfloat{\qquad}
  \includegraphics[width=0.45\textwidth, page=67, trim=15 0 15 0, clip]{figures/housing/overview_fig.pdf}\hfil
  \subfloat{\qquad}
  \includegraphics[width=0.45\textwidth, page=68, trim=15 0 15 0, clip]{figures/housing/overview_fig.pdf}
  \subfloat{\qquad}
  \includegraphics[width=0.45\textwidth, page=69, trim=15 0 15 0, clip]{figures/housing/overview_fig.pdf}\hfil
  \subfloat{\qquad}
  \includegraphics[width=0.45\textwidth, page=70, trim=15 0 15 0, clip]{figures/housing/overview_fig.pdf}
  \subfloat{\qquad}
  \includegraphics[width=0.45\textwidth, page=71, trim=15 0 15 0, clip]{figures/housing/overview_fig.pdf}\hfil
  \subfloat{\qquad}
  \includegraphics[width=0.45\textwidth, page=72, trim=15 0 15 0, clip]{figures/housing/overview_fig.pdf}
  \caption[Distributions for the housing price dataset]{Distributions the 168 input variables (excluding \code{ID} and \code{Vejnavn}).}
  \label{fig:h:variable_overview_all_6}
  \vspace{\abovecaptionskip}
\end{figure*}

\begin{figure*}
  \centering
  \subfloat{\qquad}
  \includegraphics[width=0.45\textwidth, page=73, trim=15 0 15 0, clip]{figures/housing/overview_fig.pdf}\hfil
  \subfloat{\qquad}
  \includegraphics[width=0.45\textwidth, page=74, trim=15 0 15 0, clip]{figures/housing/overview_fig.pdf}
  \subfloat{\qquad}
  \includegraphics[width=0.45\textwidth, page=75, trim=15 0 15 0, clip]{figures/housing/overview_fig.pdf}\hfil
  \subfloat{\qquad}
  \includegraphics[width=0.45\textwidth, page=76, trim=15 0 15 0, clip]{figures/housing/overview_fig.pdf}
  \subfloat{\qquad}
  \includegraphics[width=0.45\textwidth, page=77, trim=15 0 15 0, clip]{figures/housing/overview_fig.pdf}\hfil
  \subfloat{\qquad}
  \includegraphics[width=0.45\textwidth, page=78, trim=15 0 15 0, clip]{figures/housing/overview_fig.pdf}
  \subfloat{\qquad}
  \includegraphics[width=0.45\textwidth, page=79, trim=15 0 15 0, clip]{figures/housing/overview_fig.pdf}\hfil
  \subfloat{\qquad}
  \includegraphics[width=0.45\textwidth, page=80, trim=15 0 15 0, clip]{figures/housing/overview_fig.pdf}
  \subfloat{\qquad}
  \includegraphics[width=0.45\textwidth, page=81, trim=15 0 15 0, clip]{figures/housing/overview_fig.pdf}\hfil
  \subfloat{\qquad}
  \includegraphics[width=0.45\textwidth, page=82, trim=15 0 15 0, clip]{figures/housing/overview_fig.pdf}
  \subfloat{\qquad}
  \includegraphics[width=0.45\textwidth, page=83, trim=15 0 15 0, clip]{figures/housing/overview_fig.pdf}\hfil
  \subfloat{\qquad}
  \includegraphics[width=0.45\textwidth, page=84, trim=15 0 15 0, clip]{figures/housing/overview_fig.pdf}
  \caption[Distributions for the housing price dataset]{Distributions the 168 input variables (excluding \code{ID} and \code{Vejnavn}).}
  \label{fig:h:variable_overview_all_7}
  \vspace{\abovecaptionskip}
\end{figure*}

\begin{figure*}
  \centering
  \subfloat{\qquad}
  \includegraphics[width=0.45\textwidth, page=85, trim=15 0 15 0, clip]{figures/housing/overview_fig.pdf}\hfil
  \subfloat{\qquad}
  \includegraphics[width=0.45\textwidth, page=86, trim=15 0 15 0, clip]{figures/housing/overview_fig.pdf}
  \subfloat{\qquad}
  \includegraphics[width=0.45\textwidth, page=87, trim=15 0 15 0, clip]{figures/housing/overview_fig.pdf}\hfil
  \subfloat{\qquad}
  \includegraphics[width=0.45\textwidth, page=88, trim=15 0 15 0, clip]{figures/housing/overview_fig.pdf}
  \subfloat{\qquad}
  \includegraphics[width=0.45\textwidth, page=89, trim=15 0 15 0, clip]{figures/housing/overview_fig.pdf}\hfil
  \subfloat{\qquad}
  \includegraphics[width=0.45\textwidth, page=90, trim=15 0 15 0, clip]{figures/housing/overview_fig.pdf}
  \subfloat{\qquad}
  \includegraphics[width=0.45\textwidth, page=91, trim=15 0 15 0, clip]{figures/housing/overview_fig.pdf}\hfil
  \subfloat{\qquad}
  \includegraphics[width=0.45\textwidth, page=92, trim=15 0 15 0, clip]{figures/housing/overview_fig.pdf}
  \subfloat{\qquad}
  \includegraphics[width=0.45\textwidth, page=93, trim=15 0 15 0, clip]{figures/housing/overview_fig.pdf}\hfil
  \subfloat{\qquad}
  \includegraphics[width=0.45\textwidth, page=94, trim=15 0 15 0, clip]{figures/housing/overview_fig.pdf}
  \subfloat{\qquad}
  \includegraphics[width=0.45\textwidth, page=95, trim=15 0 15 0, clip]{figures/housing/overview_fig.pdf}\hfil
  \subfloat{\qquad}
  \includegraphics[width=0.45\textwidth, page=96, trim=15 0 15 0, clip]{figures/housing/overview_fig.pdf}
  \caption[Distributions for the housing price dataset]{Distributions the 168 input variables (excluding \code{ID} and \code{Vejnavn}).}
  \label{fig:h:variable_overview_all_8}
  \vspace{\abovecaptionskip}
\end{figure*}

\begin{figure*}
  \centering
  \subfloat{\qquad}
  \includegraphics[width=0.45\textwidth, page=97, trim=15 0 15 0, clip]{figures/housing/overview_fig.pdf}\hfil
  \subfloat{\qquad}
  \includegraphics[width=0.45\textwidth, page=98, trim=15 0 15 0, clip]{figures/housing/overview_fig.pdf}
  \subfloat{\qquad}
  \includegraphics[width=0.45\textwidth, page=99, trim=15 0 15 0, clip]{figures/housing/overview_fig.pdf}\hfil
  \subfloat{\qquad}
  \includegraphics[width=0.45\textwidth, page=100, trim=15 0 15 0, clip]{figures/housing/overview_fig.pdf}
  \subfloat{\qquad}
  \includegraphics[width=0.45\textwidth, page=101, trim=15 0 15 0, clip]{figures/housing/overview_fig.pdf}\hfil
  \subfloat{\qquad}
  \includegraphics[width=0.45\textwidth, page=102, trim=15 0 15 0, clip]{figures/housing/overview_fig.pdf}
  \subfloat{\qquad}
  \includegraphics[width=0.45\textwidth, page=103, trim=15 0 15 0, clip]{figures/housing/overview_fig.pdf}\hfil
  \subfloat{\qquad}
  \includegraphics[width=0.45\textwidth, page=104, trim=15 0 15 0, clip]{figures/housing/overview_fig.pdf}
  \subfloat{\qquad}
  \includegraphics[width=0.45\textwidth, page=105, trim=15 0 15 0, clip]{figures/housing/overview_fig.pdf}\hfil
  \subfloat{\qquad}
  \includegraphics[width=0.45\textwidth, page=106, trim=15 0 15 0, clip]{figures/housing/overview_fig.pdf}
  \subfloat{\qquad}
  \includegraphics[width=0.45\textwidth, page=107, trim=15 0 15 0, clip]{figures/housing/overview_fig.pdf}\hfil
  \subfloat{\qquad}
  \includegraphics[width=0.45\textwidth, page=108, trim=15 0 15 0, clip]{figures/housing/overview_fig.pdf}
  \caption[Distributions for the housing price dataset]{Distributions the 168 input variables (excluding \code{ID} and \code{Vejnavn}).}
  \label{fig:h:variable_overview_all_9}
  \vspace{\abovecaptionskip}
\end{figure*}

\begin{figure*}
  \centering
  \subfloat{\qquad}
  \includegraphics[width=0.45\textwidth, page=109, trim=15 0 15 0, clip]{figures/housing/overview_fig.pdf}\hfil
  \subfloat{\qquad}
  \includegraphics[width=0.45\textwidth, page=110, trim=15 0 15 0, clip]{figures/housing/overview_fig.pdf}
  \subfloat{\qquad}
  \includegraphics[width=0.45\textwidth, page=111, trim=15 0 15 0, clip]{figures/housing/overview_fig.pdf}\hfil
  \subfloat{\qquad}
  \includegraphics[width=0.45\textwidth, page=112, trim=15 0 15 0, clip]{figures/housing/overview_fig.pdf}
  \subfloat{\qquad}
  \includegraphics[width=0.45\textwidth, page=113, trim=15 0 15 0, clip]{figures/housing/overview_fig.pdf}\hfil
  \subfloat{\qquad}
  \includegraphics[width=0.45\textwidth, page=114, trim=15 0 15 0, clip]{figures/housing/overview_fig.pdf}
  \subfloat{\qquad}
  \includegraphics[width=0.45\textwidth, page=115, trim=15 0 15 0, clip]{figures/housing/overview_fig.pdf}\hfil
  \subfloat{\qquad}
  \includegraphics[width=0.45\textwidth, page=116, trim=15 0 15 0, clip]{figures/housing/overview_fig.pdf}
  \subfloat{\qquad}
  \includegraphics[width=0.45\textwidth, page=117, trim=15 0 15 0, clip]{figures/housing/overview_fig.pdf}\hfil
  \subfloat{\qquad}
  \includegraphics[width=0.45\textwidth, page=118, trim=15 0 15 0, clip]{figures/housing/overview_fig.pdf}
  \subfloat{\qquad}
  \includegraphics[width=0.45\textwidth, page=119, trim=15 0 15 0, clip]{figures/housing/overview_fig.pdf}\hfil
  \subfloat{\qquad}
  \includegraphics[width=0.45\textwidth, page=120, trim=15 0 15 0, clip]{figures/housing/overview_fig.pdf}
  \caption[Distributions for the housing price dataset]{Distributions the 168 input variables (excluding \code{ID} and \code{Vejnavn}).}
  \label{fig:h:variable_overview_all_10}
  \vspace{\abovecaptionskip}
\end{figure*}

\begin{figure*}
  \centering
  \subfloat{\qquad}
  \includegraphics[width=0.45\textwidth, page=121, trim=15 0 15 0, clip]{figures/housing/overview_fig.pdf}\hfil
  \subfloat{\qquad}
  \includegraphics[width=0.45\textwidth, page=122, trim=15 0 15 0, clip]{figures/housing/overview_fig.pdf}
  \subfloat{\qquad}
  \includegraphics[width=0.45\textwidth, page=123, trim=15 0 15 0, clip]{figures/housing/overview_fig.pdf}\hfil
  \subfloat{\qquad}
  \includegraphics[width=0.45\textwidth, page=124, trim=15 0 15 0, clip]{figures/housing/overview_fig.pdf}
  \subfloat{\qquad}
  \includegraphics[width=0.45\textwidth, page=125, trim=15 0 15 0, clip]{figures/housing/overview_fig.pdf}\hfil
  \subfloat{\qquad}
  \includegraphics[width=0.45\textwidth, page=126, trim=15 0 15 0, clip]{figures/housing/overview_fig.pdf}
  \subfloat{\qquad}
  \includegraphics[width=0.45\textwidth, page=127, trim=15 0 15 0, clip]{figures/housing/overview_fig.pdf}\hfil
  \subfloat{\qquad}
  \includegraphics[width=0.45\textwidth, page=128, trim=15 0 15 0, clip]{figures/housing/overview_fig.pdf}
  \subfloat{\qquad}
  \includegraphics[width=0.45\textwidth, page=129, trim=15 0 15 0, clip]{figures/housing/overview_fig.pdf}\hfil
  \subfloat{\qquad}
  \includegraphics[width=0.45\textwidth, page=130, trim=15 0 15 0, clip]{figures/housing/overview_fig.pdf}
  \subfloat{\qquad}
  \includegraphics[width=0.45\textwidth, page=131, trim=15 0 15 0, clip]{figures/housing/overview_fig.pdf}\hfil
  \subfloat{\qquad}
  \includegraphics[width=0.45\textwidth, page=132, trim=15 0 15 0, clip]{figures/housing/overview_fig.pdf}
  \caption[Distributions for the housing price dataset]{Distributions the 168 input variables (excluding \code{ID} and \code{Vejnavn}).}
  \label{fig:h:variable_overview_all_11}
  \vspace{\abovecaptionskip}
\end{figure*}

\begin{figure*}
  \centering
  \subfloat{\qquad}
  \includegraphics[width=0.45\textwidth, page=133, trim=15 0 15 0, clip]{figures/housing/overview_fig.pdf}\hfil
  \subfloat{\qquad}
  \includegraphics[width=0.45\textwidth, page=134, trim=15 0 15 0, clip]{figures/housing/overview_fig.pdf}
  \subfloat{\qquad}
  \includegraphics[width=0.45\textwidth, page=135, trim=15 0 15 0, clip]{figures/housing/overview_fig.pdf}\hfil
  \subfloat{\qquad}
  \includegraphics[width=0.45\textwidth, page=136, trim=15 0 15 0, clip]{figures/housing/overview_fig.pdf}
  \subfloat{\qquad}
  \includegraphics[width=0.45\textwidth, page=137, trim=15 0 15 0, clip]{figures/housing/overview_fig.pdf}\hfil
  \subfloat{\qquad}
  \includegraphics[width=0.45\textwidth, page=138, trim=15 0 15 0, clip]{figures/housing/overview_fig.pdf}
  \subfloat{\qquad}
  \includegraphics[width=0.45\textwidth, page=139, trim=15 0 15 0, clip]{figures/housing/overview_fig.pdf}\hfil
  \subfloat{\qquad}
  \includegraphics[width=0.45\textwidth, page=140, trim=15 0 15 0, clip]{figures/housing/overview_fig.pdf}
  \subfloat{\qquad}
  \includegraphics[width=0.45\textwidth, page=141, trim=15 0 15 0, clip]{figures/housing/overview_fig.pdf}\hfil
  \subfloat{\qquad}
  \includegraphics[width=0.45\textwidth, page=142, trim=15 0 15 0, clip]{figures/housing/overview_fig.pdf}
  \subfloat{\qquad}
  \includegraphics[width=0.45\textwidth, page=143, trim=15 0 15 0, clip]{figures/housing/overview_fig.pdf}\hfil
  \subfloat{\qquad}
  \includegraphics[width=0.45\textwidth, page=144, trim=15 0 15 0, clip]{figures/housing/overview_fig.pdf}
  \caption[Distributions for the housing price dataset]{Distributions the 168 input variables (excluding \code{ID} and \code{Vejnavn}).}
  \label{fig:h:variable_overview_all_12}
  \vspace{\abovecaptionskip}
\end{figure*}

\begin{figure*}
  \centering
  \subfloat{\qquad}
  \includegraphics[width=0.45\textwidth, page=145, trim=15 0 15 0, clip]{figures/housing/overview_fig.pdf}\hfil
  \subfloat{\qquad}
  \includegraphics[width=0.45\textwidth, page=146, trim=15 0 15 0, clip]{figures/housing/overview_fig.pdf}
  \subfloat{\qquad}
  \includegraphics[width=0.45\textwidth, page=147, trim=15 0 15 0, clip]{figures/housing/overview_fig.pdf}\hfil
  \subfloat{\qquad}
  \includegraphics[width=0.45\textwidth, page=148, trim=15 0 15 0, clip]{figures/housing/overview_fig.pdf}
  \subfloat{\qquad}
  \includegraphics[width=0.45\textwidth, page=149, trim=15 0 15 0, clip]{figures/housing/overview_fig.pdf}\hfil
  \subfloat{\qquad}
  \includegraphics[width=0.45\textwidth, page=150, trim=15 0 15 0, clip]{figures/housing/overview_fig.pdf}
  \subfloat{\qquad}
  \includegraphics[width=0.45\textwidth, page=151, trim=15 0 15 0, clip]{figures/housing/overview_fig.pdf}\hfil
  \subfloat{\qquad}
  \includegraphics[width=0.45\textwidth, page=152, trim=15 0 15 0, clip]{figures/housing/overview_fig.pdf}
  \subfloat{\qquad}
  \includegraphics[width=0.45\textwidth, page=153, trim=15 0 15 0, clip]{figures/housing/overview_fig.pdf}\hfil
  \subfloat{\qquad}
  \includegraphics[width=0.45\textwidth, page=154, trim=15 0 15 0, clip]{figures/housing/overview_fig.pdf}
  \subfloat{\qquad}
  \includegraphics[width=0.45\textwidth, page=155, trim=15 0 15 0, clip]{figures/housing/overview_fig.pdf}\hfil
  \subfloat{\qquad}
  \includegraphics[width=0.45\textwidth, page=156, trim=15 0 15 0, clip]{figures/housing/overview_fig.pdf}
  \caption[Distributions for the housing price dataset]{Distributions the 168 input variables (excluding \code{ID} and \code{Vejnavn}).}
  \label{fig:h:variable_overview_all_13}
  \vspace{\abovecaptionskip}
\end{figure*}

\begin{figure*}
  \centering
  \subfloat{\qquad}
  \includegraphics[width=0.45\textwidth, page=157, trim=15 0 15 0, clip]{figures/housing/overview_fig.pdf}\hfil
  \subfloat{\qquad}
  \includegraphics[width=0.45\textwidth, page=158, trim=15 0 15 0, clip]{figures/housing/overview_fig.pdf}
  \subfloat{\qquad}
  \includegraphics[width=0.45\textwidth, page=159, trim=15 0 15 0, clip]{figures/housing/overview_fig.pdf}\hfil
  \subfloat{\qquad}
  \includegraphics[width=0.45\textwidth, page=160, trim=15 0 15 0, clip]{figures/housing/overview_fig.pdf}
  \subfloat{\qquad}
  \includegraphics[width=0.45\textwidth, page=161, trim=15 0 15 0, clip]{figures/housing/overview_fig.pdf}\hfil
  \subfloat{\qquad}
  \includegraphics[width=0.45\textwidth, page=162, trim=15 0 15 0, clip]{figures/housing/overview_fig.pdf}
  \subfloat{\qquad}
  \includegraphics[width=0.45\textwidth, page=163, trim=15 0 15 0, clip]{figures/housing/overview_fig.pdf}\hfil
  \subfloat{\qquad}
  \includegraphics[width=0.45\textwidth, page=164, trim=15 0 15 0, clip]{figures/housing/overview_fig.pdf}
  \subfloat{\qquad}
  \includegraphics[width=0.45\textwidth, page=165, trim=15 0 15 0, clip]{figures/housing/overview_fig.pdf}\hfil
  \subfloat{\qquad}
  \includegraphics[width=0.45\textwidth, page=166, trim=15 0 15 0, clip]{figures/housing/overview_fig.pdf}
  \caption[Distributions for the housing price dataset]{Distributions the 168 input variables (excluding \code{ID} and \code{Vejnavn}).}
  \label{fig:h:variable_overview_all_14}
  \vspace{\abovecaptionskip}
\end{figure*}



\input{figures/housing/all_input_variables/all_demo2.tex}


\begin{table*}
  % \small
  \footnotesize
  % \centerfloat
  % \resizebox{\textwidth}{!}{%
  \begin{tabular}{lll}
     \code{OISSalgsType}                       & \code{GeoLandsdelNr}                    & \code{GeoRegionNr}                       \\
     \code{GeoKommuneNr}                       & \code{GeoPostNr}                        & \code{PostHovedNr}                       \\
     \code{Etage}                              & \code{SognKode}                         & \code{ZoneKode}                          \\
     \code{GisX_WGS84}                         & \code{GisY_WGS84}                       & \code{EjendomsNr}                        \\
     \code{ArealBolig}                         & \code{ArealKaelder}                     & \code{ArealGrund}                        \\
     \code{BeregnetAreal}                      & \code{AntalRum}                         & \code{AntalEtager}                       \\
     \code{ByggeAAr}                           & \code{OmbygningsAAr}                    & \code{EnergiLov}                         \\
     \code{UdenAnnoncering}                    & \code{ProjektSalg}                      & \code{LiggetidAktuel}                    \\
     \code{LiggetidSamlet}                     & \code{Ejerudgift}                       & \code{Bygning_GOP_OpfoerselesAAr}        \\
     \code{Bygning_GOP_OmbygningsAAr}          & \code{Bygning_GOP_EjerLavKode}          & \code{Bygning_GOP_EjerForholdKode}       \\
     \code{Bygning_GOP_AntalEtagerUKldTag}     & \code{Bygning_GOP_AntalBoligMedKoekken} & \code{Bygning_GOP_AntalBoligUdenKoekken} \\
     \code{Bygning_MOP_YdervaegKode}           & \code{Bygning_MOP_TagdaekningKode}      & \code{Bygning_MOP_KildeMatrKode}         \\
     \code{Bygning_IOP_VarmeinstalKode}        & \code{Bygning_IOP_OpvarmningKode}       & \code{Bygning_IOP_SuppVarmeKode}         \\
     \code{Bygning_VOP_VandforsyningKode}      & \code{Bygning_AGP_Bebygget}             & \code{Bygning_AGP_HerafAffaldsrum}       \\
     \code{Bygning_AGP_HerafIndbGarage}        & \code{Bygning_AGP_HerafIndbCarport}     & \code{Bygning_AGP_HerafIndbUdhus}        \\
     \code{Bygning_AGP_HerafUdstue}            & \code{Bygning_AGP_Overdaekket}          & \code{Bygning_AHB_SamletBygning}         \\
     \code{Bygning_AHB_HerafUdvIsolering}      & \code{Bygning_AHB_KaelderSamlet}        & \code{Bygning_AHB_KaelderU125}           \\
     \code{Bygning_AHB_TagSamlet}              & \code{Bygning_AHB_TagBeboelse}          & \code{Bygning_AHB_LukkedeOverdaekning}   \\
     \code{Bygning_AAV_SamletBolig}            & \code{Bygning_AAV_HerafKaelder}         & \code{Bygning_AAV_Adgangs}               \\
     \code{Bygning_AAV_Andet}                  & \code{Bygning_AAV_AAbenOverdaekning}    & \code{Enhed_Ejendomsnr}                  \\
     \code{Enhed_GOP_AnvendelseKode}           & \code{Enhed_GOP_AntVaerelseErv}         & \code{Enhed_GOP_AntToilet}               \\
     \code{Enhed_GOP_AntBad}                   & \code{Enhed_GOP_EnergiKode}             & \code{Enhed_AAV_Erhverv}                 \\
     \code{Enhed_AAV_Andet}                    & \code{Enhed_AAV_FeallesAdg}             & \code{Enhed_AAV_AabenOverdaek}           \\
     \code{Enhed_AAV_LukketOverdaek}           & \code{Historisk_SalgsType1}             & \code{Historisk_SalgsPris1}              \\
     \code{Historisk_SalgsType2}               & \code{Historisk_SalgsPris2}             & \code{Historisk_SalgsType3}              \\
     \code{Historisk_SalgsPris3}               & \code{EjdVurdering_VurderingAAr0}       & \code{EjdVurdering_EjendomsVaerdi0}      \\
     \code{EjdVurdering_GrundVaerdi0}          & \code{EjdVurdering_StuehusVaerdi0}      & \code{EjdVurdering_StueGrundVaerdi0}     \\
     \code{EjdVurdering_VurderingAAr1}         & \code{EjdVurdering_EjendomsVaerdi1}     & \code{EjdVurdering_GrundVaerdi1}         \\
     \code{EjdVurdering_StuehusVaerdi1}        & \code{EjdVurdering_StueGrundVaerdi1}    & \code{EjdVurdering_VurderingAAr2}        \\
     \code{EjdVurdering_EjendomsVaerdi2}       & \code{EjdVurdering_GrundVaerdi2}        & \code{EjdVurdering_StuehusVaerdi2}       \\
     \code{EjdVurdering_StueGrundVaerdi2}      & \code{EjdVurdering_VurderingAAr3}       & \code{EjdVurdering_EjendomsVaerdi3}      \\
     \code{EjdVurdering_GrundVaerdi3}          & \code{EjdVurdering_StuehusVaerdi3}      & \code{EjdVurdering_StueGrundVaerdi3}     \\
     \code{EjdVurdering_VurderingAAr4}         & \code{EjdVurdering_EjendomsVaerdi4}     & \code{EjdVurdering_GrundVaerdi4}         \\
     \code{EjdVurdering_StuehusVaerdi4}        & \code{EjdVurdering_StueGrundVaerdi4}    & \code{Tinglyst_AntEjere}                 \\
     \code{Tinglyst_MindsteAndel}              & \code{Tinglyst_StoersteAndel}           & \code{Afstand_Faengsel}                  \\
     \code{Afstand_Hede}                       & \code{Afstand_Hoejspaendingsledning}    & \code{Afstand_Industri}                  \\
     \code{Afstand_JernbaneSynlig}             & \code{Afstand_Kirke}                    & \code{Afstand_Kirkegaard}                \\
     \code{Afstand_Kyst}                       & \code{Afstand_Landingsbane}             & \code{Afstand_Motorvej}                  \\
     \code{Afstand_MotorvejTilFraKoersel}      & \code{Afstand_RekreativtOmraade}        & \code{Afstand_Rigsgraense}               \\
     \code{Afstand_Sportsanlaeg}               & \code{Afstand_Strand}                   & \code{Afstand_Vindmoelle}                \\
     \code{Kommune_Indbyggertal}               & \code{Kommune_SkatteProcent}            & \code{Kommune_Vuggestuer}                \\
     \code{Kommune_Boernehaver}                & \code{Kommune_IntegreredeInstitutioner} & \code{Kommune_FolkeSkoler}               \\
     \code{Kommune_Grundskyld}                 & \code{dag_i_maaned}                     & \code{maaned}                            \\
     \code{aar}                                & \code{SalgsDato_siden0}                 & \code{Historisk_SalgsDato1_siden0}       \\
     \code{Historisk_SalgsDato2_siden0}        & \code{Historisk_SalgsDato3_siden0}      & \code{HusNr_tal}                         \\
     \code{HusNr_bogstav}                      & \code{SidedoerNummer}                   & \code{Vej}                               \\
     \code{ArealVaegtet_same_as_BeregnetAreal} & \code{ByggeAAr_diff}                    & \code{OmbygningsAAr_diff}                \\
     \code{Energi}                             & \code{Prophet_index}                    &                   
    \end{tabular}
    \vspace{\abovecaptionskip}
    \caption[Final Variable List]{Final variable list.}
    \label{tab:h:all_variables}
\end{table*}

\begin{figure*}[h!]
  \centerfloat
  \includegraphics[draft, width=1.1\textwidth, trim=10 10 10 10, clip]{figures/housing/correlations_all.pdf}
  \caption[Linear Correlations]
          {XXXX \TODO.}
  \label{fig:h:correlations_all_lin}
\end{figure*}


\begin{figure}[h!]
  \centering
  \includegraphics[draft, width=0.99\textwidth, trim=0 0 0 40, clip]{figures/housing/MIC_test.pdf}
  \caption[MIC non-linear correlation]
          {MIC non-linear correlation.}
  \label{fig:h:MIC_example}
\end{figure}



\begin{figure}[h!]
  \includegraphics[draft, width=0.9\textwidth]{figures/housing/Villa_v18_cut_all_Ncols_all_prophet_forecast.png}
  \caption[Prophet Forecast for apartments]
          {The predictions of the Facebook Prophet model trained on square meter prices for owner-occupied apartments sold before January 1st, 2018. The data is down-sampled to weekly bins where the median of each week is used as in input to the Prophet model. This can be seen as black dots in the figure. The \textcolor{blue}{model's forecasts} for 2018 and 2019 are shown in blue with a light blue \textcolor{light-blue}{error band} showing the $1-\sigma$ confidence interval.
          }
  \label{fig:h:prophet_forecast_villa}
\end{figure}

\begin{figure}[h!]
  \includegraphics[draft, width=0.9\textwidth]{figures/housing/Villa_v18_cut_all_Ncols_all_prophet_trends.pdf}
  \caption[Prophet Trends]
          {The trends of the Facebook Prophet model trained on square meter prices for owner-occupied apartments sold before January 1st, 2018. In the top plot is the overall trend as a function of year and in the bottom plot is the yearly variation as a function of day of year. It can be seen that the square meter price is higher during the Summer months compared to the Winter months, however, compared to the overall trend this effect is minor ($<10\%$). 
          }
  \label{fig:h:prophet_trends_villa}
\end{figure}



\FloatBarrier


 %rmse:

\subsection*{Apartments}

 \begin{table}[h!]
  \centerfloat
  \begin{tabular}{@{}ccrrc@{}}
    %\toprule
    Halflife & $\log_{10}$ & $N_\mathrm{trees}$ & Time [$s$] & $f_\mathrm{eval}$ \\
    \midrule
    \num{2.5} & True & \num{226} & \num{141} & \num{0.1664} \\
    \num{2.5} & False & \num{201} & \num{115} & \num{0.1770} \\
    \num{5} & True & \num{301} & \num{90} & \num{0.1623} \\
    \num{5} & False & \num{375} & \num{82} & \num{0.1786} \\
    \num{10} & True & \num{318} & \num{97} & \num{0.1618} \\
    \num{10} & False & \num{226} & \num{56} & \num{0.1893} \\
    \num{20} & True & \num{265} & \num{81} & \num{0.1626} \\
    \num{20} & False & \num{687} & \num{124} & \num{0.1799} \\
    $\bm{\infty}$ & \textbf{True} & $\mathbf{405}$ & $\mathbf{110}$ & $\mathbf{0.1600}$ \\
    $\infty$ & False & \num{94} & \num{32} & \num{0.2036} \\
    %\bottomrule
  \end{tabular}
  % \caption{\label{tab:h:HPO_initial_Rmse-ejerlejlighed-appendix}Rmse-ejerlejlighed-appendix.}
  \caption[Initial Hyperparameter Optimization Results for Apartments -- SE Loss Function]{\label{tab:h:HPO_initial_Rmse-ejerlejlighed-appendix}Initial hyperparameter optimization results for apartments for the SE loss function $\ell_\mathrm{SE}$. The best hyperparameter is shown in bold.}
\end{table}

 %logcosh:

\begin{table}[h!]
  \centerfloat
  \begin{tabular}{@{}ccrrc@{}}
    %\toprule
    Halflife & $\log_{10}$ & $N_\mathrm{trees}$ & Time [$s$] & $f_\mathrm{eval}$ \\
    \midrule
    \num{2.5} & True & \num{333} & \num{75} & \num{0.1595} \\
    \num{2.5} & False & \num{496} & \num{57} & \num{0.1523} \\
    \num{5} & True & \num{280} & \num{66} & \num{0.1606} \\
    $\mathbf{5}$ & \textbf{False} & $\mathbf{734}$ & $\mathbf{96}$ & $\mathbf{0.1513}$ \\
    \num{10} & True & \num{367} & \num{83} & \num{0.1618} \\
    \num{10} & False & \num{351} & \num{52} & \num{0.1590} \\
    \num{20} & True & \num{269} & \num{62} & \num{0.1609} \\
    \num{20} & False & \num{333} & \num{49} & \num{0.1587} \\
    $\infty$ & True & \num{388} & \num{83} & \num{0.1595} \\
    $\infty$ & False & \num{268} & \num{42} & \num{0.1648} \\
    %\bottomrule
  \end{tabular}
  % \caption{\label{tab:h:HPO_initial_Logcosh-ejerlejlighed-appendix}Logcosh-ejerlejlighed-appendix.}
  \caption[Initial Hyperparameter Optimization Results for Apartments -- Log-Cosh Loss Function]{\label{tab:h:HPO_initial_Logcosh-ejerlejlighed-appendix}Initial hyperparameter optimization results for apartments for the log-cosh loss function $\ell_\mathrm{LogCosh}$. The best hyperparameter is shown in bold.}
\end{table}

 %cauchy:

\begin{table}[h!]
  \centerfloat
  \begin{tabular}{@{}ccrrc@{}}
    %\toprule
    Halflife & $\log_{10}$ & $N_\mathrm{trees}$ & Time [$s$] & $f_\mathrm{eval}$ \\
    \midrule
    \num{2.5} & True & \num{293} & \num{56} & \num{0.1598} \\
    \num{2.5} & False & \num{814} & \num{101} & \num{0.1466} \\
    \num{5} & True & \num{304} & \num{68} & \num{0.1610} \\
    \num{5} & False & \num{923} & \num{110} & \num{0.1468} \\
    \num{10} & True & \num{266} & \num{62} & \num{0.1610} \\
    $\mathbf{10}$ & \textbf{False} & $\mathbf{770}$ & $\mathbf{97}$ & $\mathbf{0.1450}$ \\
    \num{20} & True & \num{288} & \num{65} & \num{0.1613} \\
    \num{20} & False & \num{967} & \num{117} & \num{0.1467} \\
    $\infty$ & True & \num{340} & \num{72} & \num{0.1601} \\
    $\infty$ & False & \num{807} & \num{99} & \num{0.1480} \\
    %\bottomrule
  \end{tabular}
  % \caption{\label{tab:h:HPO_initial_Cauchy-ejerlejlighed-appendix}Cauchy-ejerlejlighed-appendix.}
  \caption[Initial Hyperparameter Optimization Results for Apartments -- Cauchy Loss Function]{\label{tab:h:HPO_initial_Cauchy-ejerlejlighed-appendix}Initial hyperparameter optimization results for apartments for the Cauchy loss function $\ell_\mathrm{Cauchy}$. The best hyperparameter is shown in bold.}
\end{table}

 %welsch:

\begin{table}[h!]
  \centerfloat
  \begin{tabular}{@{}ccrrc@{}}
    %\toprule
    Halflife & $\log_{10}$ & $N_\mathrm{trees}$ & Time [$s$] & $f_\mathrm{eval}$ \\
    \midrule
    \num{2.5} & True & \num{285} & \num{64} & \num{0.1628} \\
    \num{2.5} & False & \num{718} & \num{90} & \num{0.1517} \\
    \num{5} & True & \num{257} & \num{62} & \num{0.1600} \\
    \num{5} & False & \num{702} & \num{91} & \num{0.1499} \\
    \num{10} & True & \num{272} & \num{62} & \num{0.1601} \\
    \num{10} & False & \num{771} & \num{99} & \num{0.1466} \\
    \num{20} & True & \num{260} & \num{61} & \num{0.1603} \\
    \num{20} & False & \num{876} & \num{107} & \num{0.1486} \\
    $\infty$ & True & \num{310} & \num{69} & \num{0.1584} \\
    $\bm{\infty}$ & \textbf{False} & $\mathbf{973}$ & $\mathbf{115}$ & $\mathbf{0.1459}$ \\
    %\bottomrule
  \end{tabular}
  % \caption{\label{tab:h:HPO_initial_Welsch-ejerlejlighed-appendix}Welsch-ejerlejlighed-appendix.}
  \caption[Initial Hyperparameter Optimization Results for Apartments -- Welsch Loss Function]{\label{tab:h:HPO_initial_Welsch-ejerlejlighed-appendix}Initial hyperparameter optimization results for apartments for the Welsch loss function $\ell_\mathrm{Welsch}$. The best hyperparameter is shown in bold.}
\end{table}

 %fair:

\begin{table}[h!]
  \centerfloat
  \begin{tabular}{@{}ccrrc@{}}
    %\toprule
    Halflife & $\log_{10}$ & $N_\mathrm{trees}$ & Time [$s$] & $f_\mathrm{eval}$ \\
    \midrule
    \num{2.5} & True & \num{229} & \num{54} & \num{0.1601} \\
    \num{2.5} & False & \num{304} & \num{45} & \num{0.1577} \\
    \num{5} & True & \num{205} & \num{54} & \num{0.1629} \\
    \num{5} & False & \num{343} & \num{51} & \num{0.1549} \\
    \num{10} & True & \num{257} & \num{61} & \num{0.1596} \\
    \num{10} & False & \num{332} & \num{47} & \num{0.1573} \\
    \num{20} & True & \num{272} & \num{62} & \num{0.1608} \\
    \num{20} & False & \num{403} & \num{56} & \num{0.1537} \\
    $\infty$ & True & \num{344} & \num{74} & \num{0.1578} \\
    $\bm{\infty}$ & \textbf{False} & $\mathbf{453}$ & $\mathbf{59}$ & $\mathbf{0.1527}$ \\
    %\bottomrule
  \end{tabular}
  % \caption{\label{tab:h:HPO_initial_Fair-ejerlejlighed-appendix}Fair-ejerlejlighed-appendix.}
  \caption[Initial Hyperparameter Optimization Results for Apartments -- Fair Loss Function]{\label{tab:h:HPO_initial_Fair-ejerlejlighed-appendix}Initial hyperparameter optimization results for apartments for the Fair loss function $\ell_\mathrm{Fair}$. The best hyperparameter is shown in bold.}
\end{table}



\clearpage
\FloatBarrier
\subsection*{Houses}


 %rmse:

 \begin{table}[h!]
  \centerfloat
  \begin{tabular}{@{}ccrrc@{}}
    %\toprule
    Halflife & $\log_{10}$ & $N_\mathrm{trees}$ & Time [$s$] & $f_\mathrm{eval}$ \\
    \midrule
    \num{2.5} & True & \num{458} & \num{339} & \num{0.1983} \\
    \num{2.5} & False & \num{844} & \num{439} & \num{0.1913} \\
    \num{5} & True & \num{733} & \num{478} & \num{0.1968} \\
    \num{5} & False & \num{1126} & \num{541} & \num{0.1888} \\
    \num{10} & True & \num{434} & \num{310} & \num{0.1999} \\
    \num{10} & False & \num{917} & \num{444} & \num{0.1884} \\
    \num{20} & True & \num{398} & \num{286} & \num{0.2013} \\
    $\mathbf{20}$ & \textbf{False} & $\mathbf{1206}$ & $\mathbf{575}$ & $\mathbf{0.1867}$ \\
    $\infty$ & True & \num{730} & \num{505} & \num{0.1977} \\
    $\infty$ & False & \num{1264} & \num{625} & \num{0.1876} \\
    %\bottomrule
  \end{tabular}
  % \caption{\label{tab:h:HPO_initial_Rmse-villa-appendix}Rmse-villa-appendix.}
  \caption[Initial Hyperparameter Optimization Results for Houses -- SE Loss Function]{\label{tab:h:HPO_initial_Rmse-villa-appendix}Initial hyperparameter optimization results for houses for the SE loss function $\ell_\mathrm{SE}$. The best hyperparameter is shown in bold.}
\end{table}

 %logcosh:

\begin{table}[h!]
  \centerfloat
  \begin{tabular}{@{}ccrrc@{}}
    %\toprule
    Halflife & $\log_{10}$ & $N_\mathrm{trees}$ & Time [$s$] & $f_\mathrm{eval}$ \\
    \midrule
    \num{2.5} & True & \num{346} & \num{223} & \num{0.2018} \\
    \num{2.5} & False & \num{1095} & \num{415} & \num{0.1877} \\
    \num{5} & True & \num{618} & \num{331} & \num{0.1976} \\
    \num{5} & False & \num{1601} & \num{546} & \num{0.1847} \\
    \num{10} & True & \num{506} & \num{280} & \num{0.1990} \\
    \num{10} & False & \num{1160} & \num{400} & \num{0.1873} \\
    \num{20} & True & \num{445} & \num{269} & \num{0.2011} \\
    \num{20} & False & \num{1313} & \num{497} & \num{0.1876} \\
    $\infty$ & True & \num{432} & \num{258} & \num{0.1982} \\
    $\bm{\infty}$ & \textbf{False} & $\mathbf{2151}$ & $\mathbf{739}$ & $\mathbf{0.1842}$ \\
    %\bottomrule
  \end{tabular}
  % \caption{\label{tab:h:HPO_initial_Logcosh-villa-appendix}Logcosh-villa-appendix.}
  \caption[Initial Hyperparameter Optimization Results for Houses -- Log-Cosh Loss Function]{\label{tab:h:HPO_initial_Logcosh-villa-appendix}Initial hyperparameter optimization results for houses for the Log-Cosh loss function $\ell_\mathrm{LogCosh}$. The best hyperparameter is shown in bold.}
\end{table}

 %cauchy:

\begin{table}[h!]
  \centerfloat
  \begin{tabular}{@{}ccrrc@{}}
    %\toprule
    Halflife & $\log_{10}$ & $N_\mathrm{trees}$ & Time [$s$] & $f_\mathrm{eval}$ \\
    \midrule
    \num{2.5} & True & \num{434} & \num{244} & \num{0.1991} \\
    \num{2.5} & False & \num{1007} & \num{356} & \num{0.1872} \\
    \num{5} & True & \num{350} & \num{208} & \num{0.1999} \\
    \num{5} & False & \num{1130} & \num{389} & \num{0.1858} \\
    \num{10} & True & \num{436} & \num{240} & \num{0.1992} \\
    \num{10} & False & \num{1183} & \num{394} & \num{0.1850} \\
    \num{20} & True & \num{397} & \num{242} & \num{0.2003} \\
    $\mathbf{20}$ & \textbf{False} & $\mathbf{1514}$ & $\mathbf{542}$ & $\mathbf{0.1833}$ \\
    $\infty$ & True & \num{449} & \num{257} & \num{0.1992} \\
    $\infty$ & False & \num{1351} & \num{470} & \num{0.1844} \\
    %\bottomrule
  \end{tabular}
  % \caption{\label{tab:h:HPO_initial_Cauchy-villa-appendix}Cauchy-villa-appendix.}
  \caption[Initial Hyperparameter Optimization Results for Houses -- Cauchy Loss Function]{\label{tab:h:HPO_initial_Cauchy-villa-appendix}Initial hyperparameter optimization results for houses for the Cauchy loss function $\ell_\mathrm{Cauchy}$. The best hyperparameter is shown in bold.}
\end{table}

 %welsch:

\begin{table}[h!]
  \centerfloat
  \begin{tabular}{@{}ccrrc@{}}
    %\toprule
    Halflife & $\log_{10}$ & $N_\mathrm{trees}$ & Time [$s$] & $f_\mathrm{eval}$ \\
    \midrule
    \num{2.5} & True & \num{867} & \num{440} & \num{0.1960} \\
    \num{2.5} & False & \num{835} & \num{300} & \num{0.1897} \\
    \num{5} & True & \num{301} & \num{184} & \num{0.2035} \\
    \num{5} & False & \num{893} & \num{312} & \num{0.1878} \\
    \num{10} & True & \num{341} & \num{200} & \num{0.2014} \\
    \num{10} & False & \num{1113} & \num{390} & \num{0.1869} \\
    \num{20} & True & \num{338} & \num{209} & \num{0.2022} \\
    \num{20} & False & \num{1212} & \num{424} & \num{0.1875} \\
    $\infty$ & True & \num{579} & \num{321} & \num{0.1970} \\
    $\bm{\infty}$ & \textbf{False} & $\mathbf{1497}$ & $\mathbf{509}$ & $\mathbf{0.1837}$ \\
    %\bottomrule
  \end{tabular}
  % \caption{\label{tab:h:HPO_initial_Welsch-villa-appendix}Welsch-villa-appendix.}
  \caption[Initial Hyperparameter Optimization Results for Houses -- Welsch Loss Function]{\label{tab:h:HPO_initial_Welsch-villa-appendix}Initial hyperparameter optimization results for houses for the Welsch loss function $\ell_\mathrm{Welsch}$. The best hyperparameter is shown in bold.}
\end{table}

 %fair:

\begin{table}[h!]
  \centerfloat
  \begin{tabular}{@{}ccrrc@{}}
    %\toprule
    Halflife & $\log_{10}$ & $N_\mathrm{trees}$ & Time [$s$] & $f_\mathrm{eval}$ \\
    \midrule
    \num{2.5} & True & \num{508} & \num{278} & \num{0.1956} \\
    \num{2.5} & False & \num{862} & \num{301} & \num{0.1882} \\
    \num{5} & True & \num{506} & \num{278} & \num{0.1957} \\
    $\mathbf{5}$ & \textbf{False} & $\mathbf{1357}$ & $\mathbf{462}$ & $\mathbf{0.1835}$ \\
    \num{10} & True & \num{875} & \num{436} & \num{0.1946} \\
    \num{10} & False & \num{954} & \num{325} & \num{0.1861} \\
    \num{20} & True & \num{763} & \num{402} & \num{0.1943} \\
    \num{20} & False & \num{1256} & \num{435} & \num{0.1840} \\
    $\infty$ & True & \num{535} & \num{303} & \num{0.1973} \\
    $\infty$ & False & \num{1337} & \num{456} & \num{0.1844} \\
    %\bottomrule
  \end{tabular}
  % \caption{\label{tab:h:HPO_initial_Fair-villa-appendix}Fair-villa-appendix.}
  \caption[Initial Hyperparameter Optimization Results for Houses -- Fair Loss Function]{\label{tab:h:HPO_initial_Fair-villa-appendix}Initial hyperparameter optimization results for houses for the Fair loss function $\ell_\mathrm{Fair}$. The best hyperparameter is shown in bold.}
\end{table}




\FloatBarrier

\begin{figure}
  \includegraphics[width=0.95\textwidth, trim=0 0 0 0, clip]{figures/housing/Villa_v18_cut_all_Ncols_all_CV_viz_initial_HPO.pdf}
  \caption[Overview of initial hyperparamater optimization of the housing model for houses]
          {Hyperparameter optimization results of the housing model for houses. The results are shown as parallel coordinates with each hyperparameter along the x-axis and the value of that parameter on the y-axis. Each line is an event in the 4-dimensional space colored according to the performance of that hyperparameter as measured by $\mathrm{MAD}_0$ from \textcolor{viridis-dark}{highest} MAD in dark blue to \textcolor{viridis-light}{lowest} AUC in yellow. The \textcolor{red}{single best hyperparameter} is shown in red. For the hyperparamater \code{log10} \code{0} means False and \code{1} means True, for \code{Halftime} $\infty$ is mapped to \code{30}, and for \code{objektive} the functions Cauchy (0), Fair (1), LogCosh (2) SquaredError (3), and Welsch (4) are mapped to the integers in the parentheses.
          } 
  \label{fig:h:initial_CV_res_parallel_coords_villa}
\end{figure}


\newpage

\begin{figure}
  \includegraphics[width=0.95\textwidth, trim=0 0 0 0, clip]{figures/housing/Ejerlejlighed_v19_cut_all_Ncols_all_CV_viz_HPO_RS.pdf}
  \caption[XXX]
          {Hyperparameter optimization results of XGBoost parameters of the housing model for apartments shown as parallel coordinates. Here shown for random search as hyperparameter optimization.
          } 
  \label{fig:h:CV_res_BO_parallel_coords_ejer}
\end{figure}

\begin{figure}
  \includegraphics[width=0.95\textwidth, trim=0 0 0 0, clip]{figures/housing/Ejerlejlighed_v19_cut_all_Ncols_all_CV_viz_HPO_BO.pdf}
  \caption[XXX]
          {Hyperparameter optimization results of XGBoost parameters of the housing model for apartments shown as parallel coordinates. Here shown for Bayesian optimization as hyperparameter optimization.
          } 
  \label{fig:h:CV_res_BO_parallel_coords_ejer}
\end{figure}


\begin{figure}
  \includegraphics[width=0.95\textwidth, trim=0 0 0 0, clip]{figures/housing/Villa_v18_cut_all_Ncols_all_CV_viz_HPO_RS.pdf}
  \caption[XXX]
          {Hyperparameter optimization results of XGBoost parameters of the housing model for houses shown as parallel coordinates. Here shown for random search as hyperparameter optimization.
          } 
  \label{fig:h:CV_res_RS_parallel_coords_villa}
\end{figure}

\begin{figure}
  \includegraphics[width=0.95\textwidth, trim=0 0 0 0, clip]{figures/housing/Villa_v18_cut_all_Ncols_all_CV_viz_HPO_BO.pdf}
  \caption[XXX]
          {Hyperparameter optimization results of XGBoost parameters of the housing model for houses shown as parallel coordinates. Here shown for Bayesian optimization as hyperparameter optimization.
          } 
  \label{fig:h:CV_res_BO_parallel_coords_villa}
\end{figure}

\newpage



\begin{figure}
  \includegraphics[width=0.95\textwidth, trim=0 0 0 0, clip]{figures/housing/Ejerlejlighed_v19_cut_all_Ncols_all_xgb_score_over_time_random.pdf}
  \caption[XXX]
          {XXX of the housing model for apartments shown as parallel coordinates. Here shown for random search as hyperparameter optimization.
          } 
  \label{fig:h:CV_res_RS_uncertainties_ejer2}
\end{figure}

\begin{figure}
  \includegraphics[width=0.95\textwidth, trim=0 0 0 0, clip]{figures/housing/Ejerlejlighed_v19_cut_all_Ncols_all_xgb_score_over_time_BO.pdf}
  \caption[XXX]
          {XXX of the housing model for apartments shown as parallel coordinates. Here shown for Bayesian optimization as hyperparameter optimization.
          } 
  \label{fig:h:CV_res_BO_uncertainties_ejer}
\end{figure}


% TODO: add villa here
% \begin{figure}
%   \includegraphics[width=0.95\textwidth, trim=0 0 0 0, clip]{figures/housing/Ejerlejlighed_v18_cut_all_Ncols_all_xgb_score_over_time_random.pdf}
%   \caption[XXX]
%           {XXX of the housing model for apartments shown as parallel coordinates. Here shown for random search as hyperparameter optimization.
%           } 
%   \label{fig:h:CV_res_RS_uncertainties_ejer2}
% \end{figure}

% \begin{figure}
%   \includegraphics[width=0.95\textwidth, trim=0 0 0 0, clip]{figures/housing/Ejerlejlighed_v18_cut_all_Ncols_all_xgb_score_over_time_BO.pdf}
%   \caption[XXX]
%           {XXX of the housing model for apartments shown as parallel coordinates. Here shown for Bayesian optimization as hyperparameter optimization.
%           } 
%   \label{fig:h:CV_res_BO_uncertainties_ejer}
% \end{figure}

\newpage

\begin{figure}
  \includegraphics[width=0.95\textwidth, trim=0 0 0 0, clip]{figures/housing/Ejerlejlighed_v19_cut_all_Ncols_all_xgb_z_hist_metrics_tight.pdf}
  \caption[Performance of XGB-model on apartment prices]
          {Histogram of $z$-values of the XGB-model trained on apartments. The performance after hyperparameter optimization (HPO) using \textcolor{blue}{Random Search} (RS) is shown in blue, for \textcolor{red}{Bayesian Optimization} (BO) in red. After finding the best model, BO in this case, the model is retrained using \textcolor{green}{early stopping}, the performance of which is shown in green.} 
  \label{fig:h:CV_res_performance_ejer_tight}
\end{figure}


\newpage
\FloatBarrier


\begin{table}
  \centerfloat
  \begin{tabular}{@{}lccccc@{}}
    {} &      MAD (\%) & $\leq 5\% (\%)$ &  $\leq 10\% (\%)$ &   $\leq 20\% (\%)$ & $\mu$              \\
    \midrule
    Train & \num{6.35} & \num{56.22} & \num{83.41} & \num{97.08} &   $0.00902 \pm 0.00068$ \\
    Test  & \num{8.38} & \num{45.06} & \num{74.32} & \num{95.58} &  $-0.00820 \pm 0.00115$ \\
    2019  & \num{9.12} & \num{42.63} & \num{71.36} & \num{93.65} &   $0.00297 \pm 0.00235$ 
    \end{tabular}
  \vspace{\abovecaptionskip}
  \caption{XXX ejer tight}
  \label{tab:h:results_ejer_tight}
\end{table}


\begin{table}
  \centerfloat
  \begin{tabular}{@{}lccccc@{}}
    {} &      MAD (\%) & $\leq 5\% (\%)$ &  $\leq 10\% (\%)$ &   $\leq 20\% (\%)$ & $\mu$              \\
    \midrule
    Train & \num{15.63} & \num{25.65} & \num{47.89} & \num{75.82} &  $0.04543 \pm 0.00080$ \\
    Test  & \num{16.49} & \num{24.30} & \num{45.77} & \num{75.19} &  $0.01686 \pm 0.00194$ \\
    2019  & \num{17.17} & \num{23.67} & \num{44.25} & \num{73.54} &  $0.02056 \pm 0.00279$ 
    \end{tabular}
  \vspace{\abovecaptionskip}
  \caption{XXX villa tight}
  \label{tab:h:results_villa_tight}
\end{table}


\newpage
\FloatBarrier




\chapter{Quarks vs. Gluons Appendix}


\backmatter

\bibliography{My_Library}
% \bibliographystyle{plainnat}


\printindex

\end{document}

