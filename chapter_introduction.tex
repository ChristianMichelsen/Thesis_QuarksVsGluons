
\chapter{Introduction}
\label{ch:introduction}
\epigraph{\textit{``Begin at the beginning,'' the King said, gravely, ``and go on till you
come to an end; then stop.''}}{--- Lewis Carroll, \textit{Alice in Wonderland}}


\newthought{Not only is the title} of this project fairly broad, so are the subjects covered in this thesis. The overall goal of this project is to apply machine learning to different datasets and see how well these comparatively new tools might improve classical statistical methods. The project have dealt with two (seemingly) very different datasets: Danish housing prices and Quark-Gluon discrimination in particle physics. The aim of this section is to provide an initial overview of the scope and relationship of the two sub-projects.

The first part of the thesis deals with the problem of estimating housing prices as precisely and accurately as possible. This was the sub-project that was worked on in the beginning of the overall project and worked as an initial introduction to the application of machine learning to real-life datasets. The housing prices dataset thus became the playground in which the subtleties of these new modern tools were examined, where the difference between real life datasets with all its quirks, outliers and bad formatting, and curated toy datasets that works out of the box (such as the famous Iris dataset \citep{andersonSpeciesProblemIris1936a,fisherUseMultipleMeasurements1936}) were experienced first hand. Since the project started the dataset changed due to a new collaboration with the Danish housing agency \href{www.boligsiden.dk}{Boligsiden} where the agreement was, stated shortly, that we would get their data and they would get our results. Boligsiden is a natural collaborator since they are the biggest on the market\sidenote{Due to being owned by the \q{Dansk Ejendomsmæglerforening}, The Danish Association of Chartered Estate Agents.} and have been very helpful in the continuos process of providing data. It should also be noted that they have had no say on the results presented in this thesis. 
During this initial stage, the author sparred with Simon Gudiksen\sidenote{Who afterwards went on to get a job at Boligsiden.} who also worked on the same dataset, however, both projects were done independently. Where Gudiksen focussed on the prediction of the time evolution of the housing prices using Recurrent Neural Networks (RNN), my work was mostly on the different levels and methods of hyperparameter optimization with some smaller detours into Natural Language Processing (NLP) as an example. 

\clearpage


The second part, the Quark-Gluon discrimination in particle physics, was the main part of the project. Not only was most of the time focussed on this sub-project, it was also the work that generated the highest academic output; an article based on this is in the making. This part dealt with data from the Large Electron Positron collider (LEP) which was an underground particle accelerator at CERN built in \num{1989} and was discontinued in \num{2000}, where the first phase (LEP1), from \num{1989}-\num{1995}, is the sole source of data. As the name suggests it collided electrons and positrons together in what is still the largest electron-positron accelerator ever built \citep{LargeElectronPositronCollider}. During LEP1 it was primarily the decay channels of the $Z$-boson that were probed where especially the $Z\rightarrow q\bar{q}g$ and $Z \rightarrow q\bar{q}gg$ were examined in this thesis. The distributions of these gluon jets and the difference between Data\sidenote{Where \q{Data} with capital D refers to the actual, measured data and \q{data} refers to any arbitrary selection of data.} and Monte-Carlo (MC) that are of interests to the theoreticians that develop the MC-models. At first an improved $b$-tagging algorithm was developed. Here methods and code developed in the hyperparameter optimization process from the housing prices part were used. After the improvement in the $b$-tagging model, an event-based $g$-tag model -- in comparison to the jet-based $b$-tagging model -- was implemented which allows one to extract useful events of interest. Having found these useful events, one can start looking at how the distributions in the relevant variables differ between Data and MC. Finally XXX \TODO.

The thesis is structured such that \autoref{ch:ML_theory} introduces the needed theoretical Machine Learning (ML) background needed for understanding the methods used throughout the thesis and \autoref{ch:housing_price_analysis} describes the housing prices project as mentioned above. These two chapters constitutes the Part \RNum{1} of this thesis. Part \RNum{2} starts with \autoref{ch:hep:particle_physcis_LEP} that introduces the basic physics in the standard model and the Lund string model which is used in the quark gluon project in \autoref{ch:quark_gluon_analysis}. This chapter explains the analysis of the main project in this thesis: the quark gluon analysis.

The work presented in this thesis is split up into two parts as presented above, however, it should be noted that during the analysis part of the project they were treated not as two different projects but rather as two complementary instances of same underlying problem: teaching computers how to find patterns automatically in high-dimensional data and should thus not be seen as two independent projects. This also highlights another key aspect of this project, that the author does not have any background in particle physics other than rudimentary knowledge stemming from an undergraduate education in general physics. 

All of the work presented here is performed by the author unless otherwise noted.

